\documentclass[class=article, crop=false]{standalone}
% Import packages
\usepackage[margin=1in]{geometry}

\usepackage[many]{tcolorbox}
\usepackage{amssymb, amsthm}
\usepackage{comment}
\usepackage{enumitem}
\usepackage{fancyhdr}
\usepackage{hyperref}
\usepackage{import}
\usepackage{listings}
\usepackage{mathrsfs, mathtools}
\usepackage{pdfpages}
\usepackage{standalone}
\usepackage{transparent}
\usepackage{xcolor}

\usetikzlibrary{decorations.pathreplacing}
\tcbuselibrary{skins}
% Declare math operators
\DeclareMathOperator{\lcm}{lcm}
\DeclareMathOperator{\proj}{proj}
\DeclareMathOperator{\vspan}{span}
\DeclareMathOperator{\im}{im}
\DeclareMathOperator{\range}{range}
\DeclareMathOperator{\Diff}{Diff}
\DeclareMathOperator{\Int}{Int}
\DeclareMathOperator{\fcn}{fcn}
\DeclareMathOperator{\id}{id}
\DeclareMathOperator{\rank}{rank}
\DeclareMathOperator{\tr}{tr}
\DeclareMathOperator{\dive}{div}
\DeclareMathOperator{\row}{row}
\DeclareMathOperator{\col}{col}
% Macros for letters/variables
\renewcommand{\tilde}{\raisebox{0.4ex}{\resizebox{2ex}{!}{\texttildelow}}}
\newcommand{\N}{\ensuremath{\mathbb{N}}}
\newcommand{\Z}{\ensuremath{\mathbb{Z}}}
\newcommand{\Q}{\ensuremath{\mathbb{Q}}}
\newcommand{\R}{\ensuremath{\mathbb{R}}}
\newcommand{\C}{\ensuremath{\mathbb{C}}}
\newcommand{\F}{\ensuremath{\mathbb{F}}}
\newcommand{\M}{\ensuremath{\mathbb{M}}}
\newcommand{\lam}{\ensuremath{\lambda}}
\newcommand{\nab}{\ensuremath{\nabla}}
\newcommand{\eps}{\ensuremath{\varepsilon}}
\newcommand{\es}{\ensuremath{\varnothing}}
% Macros for math symbols
\newcommand{\dx}[1]{\,\mathrm{d}#1}
\newcommand{\inv}{\ensuremath{^{-1}}}
\newcommand{\sm}{\setminus}
\newcommand{\sse}{\subseteq}
\newcommand{\ceq}{\coloneqq}
% Macros for pairs of math symbols
\newcommand{\abs}[1]{\ensuremath{\left\lvert #1 \right\rvert}}
\newcommand{\paren}[1]{\ensuremath{\left( #1 \right)}}
\newcommand{\norm}[1]{\ensuremath{\left\lVert #1\right\rVert}}
\newcommand{\set}[1]{\ensuremath{\left\{#1\right\}}}
\newcommand{\tup}[1]{\ensuremath{\left\langle #1 \right\rangle}}
\newcommand{\floor}[1]{\ensuremath{\left\lfloor #1 \right\rfloor}}
\newcommand{\ceil}[1]{\ensuremath{\left\lceil #1 \right\rceil}}
\newcommand{\eclass}[1]{\ensuremath{\left[ #1 \right]}}

\newcommand{\chapternum}{}
\newcommand{\ex}[1]{\noindent\textbf{Exercise \chapternum.{#1}.}}

\newcommand{\tsub}[1]{\textsubscript{#1}}
\newcommand{\tsup}[1]{\textsuperscript{#1}}

% Include figures
\newcommand{\incfig}[2][1]{%
    \def\svgwidth{#1\columnwidth}
    \import{./figures/}{#2.pdf_tex}
}

\definecolor{problemBackground}{RGB}{212,232,246}

\newenvironment{problem}[1]
  {
    \begin{tcolorbox}[
      boxrule=.5pt,
      titlerule=.5pt,
      sharp corners,
      colback=problemBackground,
      breakable
    ]
    \ifx &#1& \textbf{Problem. }
    \else \textbf{Problem #1.} \fi
  }
  {
    \end{tcolorbox}
  }
\definecolor{exampleBackground}{RGB}{255,249,248}
\definecolor{exampleAccent}{RGB}{158,60,14}
\newenvironment{example}[1]
  {
    \begin{tcolorbox}[
      boxrule=.5pt,
      sharp corners,
      colback=exampleBackground,
      colframe=exampleAccent,
    ]
    \color{exampleAccent}\textbf{Example.} \emph{#1}\color{black}
  }
  {
    \end{tcolorbox}
  }
\definecolor{theoremBackground}{RGB}{234,243,251}
\definecolor{theoremAccent}{RGB}{0,116,183}
\newenvironment{theorem}[1]
  {
    \begin{tcolorbox}[
      boxrule=.5pt,
      titlerule=.5pt,
      sharp corners,
      colback=theoremBackground,
      colframe=theoremAccent,
      breakable
    ]
      \color{theoremAccent}\textbf{Theorem --- }\emph{#1}\\\color{black}
  }
  {
    \end{tcolorbox}
  }
\definecolor{noteBackground}{RGB}{244,249,244}
\definecolor{noteAccent}{RGB}{34,139,34}
\newenvironment{note}[1]
  {
  \begin{tcolorbox}[
    enhanced,
    boxrule=0pt,
    frame hidden,
    sharp corners,
    colback=noteBackground,
    borderline west={3pt}{-1.5pt}{noteAccent},
    breakable
    ]
    \ifx &#1& \color{noteAccent}\textbf{Note. }\color{black}
    \else \color{noteAccent}\textbf{Note (#1). }\color{black} \fi
    }
    {
  \end{tcolorbox}
  }
\definecolor{lemmaBackground}{RGB}{255,247,234}
\definecolor{lemmaAccent}{RGB}{255,153,0}
\newenvironment{lemma}[1]
  {
  \begin{tcolorbox}[
    enhanced,
    boxrule=0pt,
    frame hidden,
    sharp corners,
    colback=lemmaBackground,
    borderline west={3pt}{-1.5pt}{lemmaAccent},
    breakable
    ]
    \ifx &#1& \color{lemmaAccent}\textbf{Lemma. }\color{black}
    \else \color{lemmaAccent}\textbf{Lemma #1. }\color{black} \fi
    }
    {
  \end{tcolorbox}
  }
\definecolor{definitionBackground}{RGB}{246,246,246}
\newenvironment{definition}[1]
  {
    \begin{tcolorbox}[
      enhanced,
      boxrule=0pt,
      frame hidden,
      sharp corners,
      colback=definitionBackground,
      borderline west={3pt}{-1.5pt}{black},
      breakable
    ]
    \textbf{Definition. }\emph{#1}\\
  }
  {
    \end{tcolorbox}
  }

\newenvironment{amatrix}[2]{
    \left[
      \begin{array}{*{#1}{c}|*{#2}c}
  }
  {
      \end{array}
    \right]
  }
\definecolor{codeBackground}{RGB}{253,246,225}
\definecolor{dkgreen}{rgb}{0,0.6,0}
\definecolor{gray}{rgb}{0.5,0.5,0.5}
\definecolor{mauve}{rgb}{0.58,0,0.82}
\lstset{
  language=C++,
  aboveskip=3mm,
  belowskip=3mm,
  backgroundcolor=\color{codeBackground},
  showstringspaces=false,
  columns=flexible,
  basicstyle={\small\ttfamily},
  numbers=none,
  numberstyle=\tiny\color{gray},
  keywordstyle=\color{blue},
  commentstyle=\color{dkgreen},
  stringstyle=\color{mauve},
  breaklines=true,
  breakatwhitespace=true,
  tabsize=2
}

\date{\the\year-\the\month-\the\day}
\author{Kyle Chui}


\fancyhf{}
\lhead{Kyle Chui}
\rhead{Page \thepage}
\pagestyle{fancy}

\begin{document}
  \section{Bits and Bytes}
  \subsection{Representing Data in Bits and Bytes}
  \subsubsection{Everything is Bits}
  \begin{itemize}
    \item Every bit is either a $0$ or a $1$.
    \item We can use sets of bits to not only tell the computer what to do (give it instructions), but also represent data.
    \item We use bits because they are easy to store, and reliably transmitted on noisy/inaccurate wires.
  \end{itemize}
  \begin{example}{Counting in Binary} \\
    We can use bits to represent numbers in base $2$, or the binary number system. Thus we can use a set of bits to encode any number we want.
  \end{example}
  \subsubsection{Encoding Byte Values}
  \begin{definition}{Byte}
    A \emph{byte} is $8$ bits, and can be thought of as a string of 0's and 1's of length eight.
  \end{definition}
  \begin{itemize}
    \item In binary, a byte can range from $00000000_2$ to $11111111_2$.
    \item In decimal, a byte can range from $0_{10}$ to $255_{10}$.
    \item In hexadecimal (base 16), a byte can range from $00_{16}$ to FF\tsub{16}.
    \begin{note}{}
      In hexadecimal, once you finish using the digits $1$ through $9$, you use the letters A through F to represent values of $10_{10}$ through $15_{10}$. In C, we append ``0x'' before a string to indicate that it is a hexadecimal number (which may either be upper-case or lower-case). For example, FA1D37B\tsub{16} would be written as ``0xFA1D37B'' or ``0xfa1d37b''.
    \end{note}
  \end{itemize}
  All data is just a long string of bits, so any value that we get out of it depends on the \emph{context} of what we are reading in (a \texttt{double}, \texttt{char}, \texttt{int}, etc).
  \subsection{Bit Manipulation}
  \subsubsection{Boolean Algebra}
  Developed by George Boole in the 19th century, \emph{Boolean Algebra} is an algebraic representation of logic, where 1's denote a ``true'' value and 0's denote a ``false'' value. Some common logical operations are described in the tables below:
  \[\begin{array}{cccc}
    \underbrace{\begin{tabular}{c|cc}
        \& & 0 & 1 \\
        \hline
        0 & 0 & 0 \\
        1 & 0 & 1
    \end{tabular}}_{\text{``and'' operator}} &\qquad
    \underbrace{\begin{tabular}{c|cc}
        $\vert$ & 0 & 1 \\
        \hline
        0 & 0 & 1 \\
        1 & 1 & 1
    \end{tabular}}_{\text{``or'' operator}} &\qquad
    \underbrace{\begin{tabular}{c|c}
        \texttildelow & \\
        \hline
        0 & 1 \\
        1 & 0
    \end{tabular}}_{\text{``not'' operator}} &\qquad
    \underbrace{\begin{tabular}{c|cc}
        \^{} & 0 & 1 \\
        \hline
        0 & 0 & 1 \\
        1 & 1 & 1
    \end{tabular}}_{\text{``xor'' operator}}
  \end{array}\]
  \begin{note}{}
    The ``xor'' operator stands for ``exclusive-or'', meaning ``either one or the other, but not both''.
  \end{note}
  All of these operations are applied ``bitwise'' on bit vectors---the $j$th bit of the result is obtained by applying the operation on the $j$th bit of the input(s).
  \begin{example}{Representing an Manipulating Sets} \\[10pt]
    Representation
    \begin{itemize}
      \item A width $w$ bit vector can represent a subset of $\set{0,\dotsc,w-1}$.
      \item We do this by letting $a_j = 1$ if $j$ is in our set.
      \item For example, the bit vector $01101001$ would represent the set $\set{0, 3, 5, 6}$. We read from right to left, and we see that the 0th, 3rd, 5th, and 6th entries contain 1's.
    \end{itemize}
    Operations
    \begin{itemize}
      \item There are nice parallels between operations on bit vectors and operations on sets, i.e.
      \begin{itemize}
        \item \& on bit vectors is the same as set intersection.
        \item $\vert$ on bit vectors is the same as set union.
        \item \^{} on bit vectors is the same as the symmetric difference.
        \item \texttildelow\phantom{ }on bit vectors is the same as taking the complement of a set.
      \end{itemize}
    \end{itemize}
  \end{example}
  \subsubsection{Bit-Level Operations in C}
  The four boolean operations covered thus far are available in C, and can apply to any ``integral data type''(long, int, char, etc). The operators view the arguments as bit vectors and are applied bit-wise.
  \subsubsection{Logic Operations in C}
  \begin{note}{}
    These are different than the bit-level operations, so don't get them confused.
  \end{note}
  There are also the logical operators in C, namely \texttt{\&\&}, \texttt{||}, \texttt{!}.
  \begin{itemize}
    \item These view $0$ as ``false'', and anything non-zero as ``true''.
    \item They always return $0$ or $1$, and can terminate early.
  \end{itemize}
  \subsubsection{Shift Operations}
  The shift operator allows you to ``move'' the bits in a bit vector to give them higher or lower significance.
  \begin{itemize}
    \item The left shift($x\ll y$) shifts the bit vector $x$ to the left by $y$ positions (extra bits on the left are tossed out and empty slots are filled with 0's).
    \item The right shift($x\gg y$) shifts the bit vector $x$ to the right by $y$ positions (extra bits on the right are tossed out).
    \begin{itemize}
      \item In a logical shift, the empty slots are filled with 0's.
      \item In an arithmetic shift, the empty slots are filled with duplicates of the most significant bit on the left.
    \end{itemize}
    \item It is undefined behaviour if the shift amount is negative or greater than or equal to the string size.
  \end{itemize}
\end{document}
