\documentclass[class=article, crop=false]{standalone}
% Import packages
\usepackage[margin=1in]{geometry}

\usepackage[many]{tcolorbox}
\usepackage{amssymb, amsthm}
\usepackage{comment}
\usepackage{enumitem}
\usepackage{fancyhdr}
\usepackage{hyperref}
\usepackage{import}
\usepackage{listings}
\usepackage{mathrsfs, mathtools}
\usepackage{pdfpages}
\usepackage{standalone}
\usepackage{transparent}
\usepackage{xcolor}

\usetikzlibrary{decorations.pathreplacing}
\tcbuselibrary{skins}
% Declare math operators
\DeclareMathOperator{\lcm}{lcm}
\DeclareMathOperator{\proj}{proj}
\DeclareMathOperator{\vspan}{span}
\DeclareMathOperator{\im}{im}
\DeclareMathOperator{\range}{range}
\DeclareMathOperator{\Diff}{Diff}
\DeclareMathOperator{\Int}{Int}
\DeclareMathOperator{\fcn}{fcn}
\DeclareMathOperator{\id}{id}
\DeclareMathOperator{\rank}{rank}
\DeclareMathOperator{\tr}{tr}
\DeclareMathOperator{\dive}{div}
\DeclareMathOperator{\row}{row}
\DeclareMathOperator{\col}{col}
% Macros for letters/variables
\renewcommand{\tilde}{\raisebox{0.4ex}{\resizebox{2ex}{!}{\texttildelow}}}
\newcommand{\N}{\ensuremath{\mathbb{N}}}
\newcommand{\Z}{\ensuremath{\mathbb{Z}}}
\newcommand{\Q}{\ensuremath{\mathbb{Q}}}
\newcommand{\R}{\ensuremath{\mathbb{R}}}
\newcommand{\C}{\ensuremath{\mathbb{C}}}
\newcommand{\F}{\ensuremath{\mathbb{F}}}
\newcommand{\M}{\ensuremath{\mathbb{M}}}
\newcommand{\lam}{\ensuremath{\lambda}}
\newcommand{\nab}{\ensuremath{\nabla}}
\newcommand{\eps}{\ensuremath{\varepsilon}}
\newcommand{\es}{\ensuremath{\varnothing}}
% Macros for math symbols
\newcommand{\dx}[1]{\,\mathrm{d}#1}
\newcommand{\inv}{\ensuremath{^{-1}}}
\newcommand{\sm}{\setminus}
\newcommand{\sse}{\subseteq}
\newcommand{\ceq}{\coloneqq}
% Macros for pairs of math symbols
\newcommand{\abs}[1]{\ensuremath{\left\lvert #1 \right\rvert}}
\newcommand{\paren}[1]{\ensuremath{\left( #1 \right)}}
\newcommand{\norm}[1]{\ensuremath{\left\lVert #1\right\rVert}}
\newcommand{\set}[1]{\ensuremath{\left\{#1\right\}}}
\newcommand{\tup}[1]{\ensuremath{\left\langle #1 \right\rangle}}
\newcommand{\floor}[1]{\ensuremath{\left\lfloor #1 \right\rfloor}}
\newcommand{\ceil}[1]{\ensuremath{\left\lceil #1 \right\rceil}}
\newcommand{\eclass}[1]{\ensuremath{\left[ #1 \right]}}

\newcommand{\chapternum}{}
\newcommand{\ex}[1]{\noindent\textbf{Exercise \chapternum.{#1}.}}

\newcommand{\tsub}[1]{\textsubscript{#1}}
\newcommand{\tsup}[1]{\textsuperscript{#1}}

% Include figures
\newcommand{\incfig}[2][1]{%
    \def\svgwidth{#1\columnwidth}
    \import{./figures/}{#2.pdf_tex}
}

\definecolor{problemBackground}{RGB}{212,232,246}

\newenvironment{problem}[1]
  {
    \begin{tcolorbox}[
      boxrule=.5pt,
      titlerule=.5pt,
      sharp corners,
      colback=problemBackground,
      breakable
    ]
    \ifx &#1& \textbf{Problem. }
    \else \textbf{Problem #1.} \fi
  }
  {
    \end{tcolorbox}
  }
\definecolor{exampleBackground}{RGB}{255,249,248}
\definecolor{exampleAccent}{RGB}{158,60,14}
\newenvironment{example}[1]
  {
    \begin{tcolorbox}[
      boxrule=.5pt,
      sharp corners,
      colback=exampleBackground,
      colframe=exampleAccent,
    ]
    \color{exampleAccent}\textbf{Example.} \emph{#1}\color{black}
  }
  {
    \end{tcolorbox}
  }
\definecolor{theoremBackground}{RGB}{234,243,251}
\definecolor{theoremAccent}{RGB}{0,116,183}
\newenvironment{theorem}[1]
  {
    \begin{tcolorbox}[
      boxrule=.5pt,
      titlerule=.5pt,
      sharp corners,
      colback=theoremBackground,
      colframe=theoremAccent,
      breakable
    ]
      \color{theoremAccent}\textbf{Theorem --- }\emph{#1}\\\color{black}
  }
  {
    \end{tcolorbox}
  }
\definecolor{noteBackground}{RGB}{244,249,244}
\definecolor{noteAccent}{RGB}{34,139,34}
\newenvironment{note}[1]
  {
  \begin{tcolorbox}[
    enhanced,
    boxrule=0pt,
    frame hidden,
    sharp corners,
    colback=noteBackground,
    borderline west={3pt}{-1.5pt}{noteAccent},
    breakable
    ]
    \ifx &#1& \color{noteAccent}\textbf{Note. }\color{black}
    \else \color{noteAccent}\textbf{Note (#1). }\color{black} \fi
    }
    {
  \end{tcolorbox}
  }
\definecolor{lemmaBackground}{RGB}{255,247,234}
\definecolor{lemmaAccent}{RGB}{255,153,0}
\newenvironment{lemma}[1]
  {
  \begin{tcolorbox}[
    enhanced,
    boxrule=0pt,
    frame hidden,
    sharp corners,
    colback=lemmaBackground,
    borderline west={3pt}{-1.5pt}{lemmaAccent},
    breakable
    ]
    \ifx &#1& \color{lemmaAccent}\textbf{Lemma. }\color{black}
    \else \color{lemmaAccent}\textbf{Lemma #1. }\color{black} \fi
    }
    {
  \end{tcolorbox}
  }
\definecolor{definitionBackground}{RGB}{246,246,246}
\newenvironment{definition}[1]
  {
    \begin{tcolorbox}[
      enhanced,
      boxrule=0pt,
      frame hidden,
      sharp corners,
      colback=definitionBackground,
      borderline west={3pt}{-1.5pt}{black},
      breakable
    ]
    \textbf{Definition. }\emph{#1}\\
  }
  {
    \end{tcolorbox}
  }

\newenvironment{amatrix}[2]{
    \left[
      \begin{array}{*{#1}{c}|*{#2}c}
  }
  {
      \end{array}
    \right]
  }
\definecolor{codeBackground}{RGB}{253,246,225}
\definecolor{dkgreen}{rgb}{0,0.6,0}
\definecolor{gray}{rgb}{0.5,0.5,0.5}
\definecolor{mauve}{rgb}{0.58,0,0.82}
\lstset{
  language=C++,
  aboveskip=3mm,
  belowskip=3mm,
  backgroundcolor=\color{codeBackground},
  showstringspaces=false,
  columns=flexible,
  basicstyle={\small\ttfamily},
  numbers=none,
  numberstyle=\tiny\color{gray},
  keywordstyle=\color{blue},
  commentstyle=\color{dkgreen},
  stringstyle=\color{mauve},
  breaklines=true,
  breakatwhitespace=true,
  tabsize=2
}

\date{\the\year-\the\month-\the\day}
\author{Kyle Chui}


\fancyhf{}
\lhead{Kyle Chui}
\rhead{Page \thepage}
\pagestyle{fancy}

\begin{document}
  \section{Early Modern Japan}
  \subsection{The Edo or Tokugawa Period (1600--1868)}
  \begin{itemize}
    \item Edo is the city where Tokugawa Ieyasu made his military headquarters after Hideyoshi moved him to Western Japan
    \item Ieyasu leads group of Eastern daimyo and defeats Western Daimyo at Sekigahara in 1600
    \item The era of the Tokugawa shogunate was known as ``The Great Peace''
    \begin{itemize}
      \item There were no major rebellions against the Tokugawa for over $250$ years
    \end{itemize}
    \item An age of great prosperity and population growth
  \end{itemize}
  \subsection{Key Events of Early Edo Period}
  \begin{itemize}
    \item $1598$ Hideyoshi dies of illness
    \item $1600$ Battle of Sekigahara
    \item $1603$ Ieyasu named Shogun, sets up shogunate capital in Edo
    \begin{note}{}
      Ieyasu wants the legitimacy, so sets up someone to to name him shogun.
    \end{note}
    \item $1605$ Ieyasu resigns in favour of son Hidetada
    \item $1611$ Ryukyu islands become vassal of Satsuma
    \item $1614$ Ban on Christianity
    \item $1615$ Ieyasu defeats and kills Hideyoshi's son Hideyori
    \item $1615$ Code for Warrior households issued
    \item $1616$ Ieyasu dies
    \item $1617$ Ieyasu deified as T\=osh\=o Dai-Gongen
    \item $1623$ Iemitsu becomes third shogun
    \item $1633$ Interdiction on travel and trade to and from abroad
    \item $1635$ System of alternate attendance (sankin kotai) instituted
    \item $1637$ Shimabara Rebellion
    \item $1635$--$1639$ Seclusion edicts
    \item $1640$s Widespread use of woodblock print for commercial publications
  \end{itemize}
  \subsection{Tokugawa Rule Over Domains}
  \begin{itemize}
    \item Three types of vassal:
    \begin{itemize}
      \item Allied daimyo---10K to 100K koku
      \item Tokugawa branch families---500K koku
      \item Outside daimyo---Over 100K koku
    \end{itemize}
    \item System of alternate attendance
    \begin{note}{}
      Daimyo would have to come and reside at Edo on alternating years.
    \end{note}
    \item Redistribution of lands belonging to enemies to allies
    \item Switching daimyo from one domain to another for strategic reasons 
    \item Limitations on daimyo
    \begin{itemize}
      \item Only allowed one castle
      \item Permission to do repairs
      \item Forbidden from building large ships (monopoly on trade)
    \end{itemize}
    \item Tension between central government (bakufu) and domains (han)
  \end{itemize}
  \subsection{Tokugawa Economy}
  \begin{itemize}
    \item Cultivated land area doubled
    \begin{itemize}
      \item Samurai move off land to domain cities and towns---they become bureaucrats
      \item Rise of wealthy peasants
    \end{itemize}
    \item Population almost doubled from beginning to middle of the Tokugawa period---18 million to $30$ million
    \item Bakufu establishment of a common currency
    \item Market networks to supply cities
    \item Rise of urban commoner class in cities
    \item Mitsui (Edo), Sumitomo (\=Osaka)
    \begin{note}{}
      The above are companies that started during this time period.
    \end{note}
  \end{itemize}
  \subsection{Tokugawa Ideology and Religion}
  \paragraph{Neo-Confucianism}
  \begin{itemize}
    \item Influence of Song Confucianism
    \item Emphasis on self-cultivation and ethical behaviour
    \item ``Reason'' as the basis of all learning and conduct
    \begin{note}{}
      Not modern reason, they took it to mean the natural order of things.
    \end{note}
    \item Establishment of official shogunal Confucian academy
    \item Blending of Confucianism with Shinto
  \end{itemize}
  \begin{note}{}
    Buddhist sects are still very much present, although they are tightly controlled.
  \end{note}
  \subsection{Tokugawa Society}
  \begin{itemize}
    \item Neo-Confucian Ideal: \\
    Status system of ``four estates''
    \begin{enumerate}
      \item Scholar/Gentleman (Warrior)
      \item Peasants
      \item Craftsman
      \item Merchants
      \begin{note}{}
        There are no outcasts, nor aristocrats.
      \end{note}
    \end{enumerate}
    \item Reality of different status groups:
    \begin{enumerate}
      \item Samurai
      \item Merchant-Craftsman, Peasant
    \end{enumerate}
    \begin{note}{}
      By the end of the Edo period: 10\% samurai, 75\% peasants, 7-8\% urban commoners, 2\% priests, 4\% miscellaneous mix.
    \end{note}
  \end{itemize}
  \subsection{Population of Tokugawa Cities}
  \begin{itemize}
    \item Edo (1000000 in 1700)
    \item \=Osaka (365000 in 1700)
    \item Ky\=oto (300000 in 1685)
    \item Nagoya (100000 in 1610)
    \item Kanazawa (100000 in 1610)
  \end{itemize}
  \subsection{Emergence of the Urban Commoner Class}
  \begin{itemize}
    \item Link between the cities and countryside
    \item Storing, distribution, banking
    \item 50\% of population of Edo
    \item Development of self-conscious social class
    \begin{itemize}
      \item Merchant values (counterpart to warrior values)
    \end{itemize}
  \end{itemize}
  \subsection{Edo Culture and Literacy}
  \begin{itemize}
    \item With printing, learning becomes more accessible
    \begin{itemize}
      \item Flourishing of different schools of thought
      \item Commercialisation of culture
    \end{itemize}
    \item The classical and medieval world is now the past
    \item There is a sense that the world is growing and changing
  \end{itemize}
  \begin{note}{}
    The high point of urban society is called the Genroku Era (1688--1704), which was the golden age of economic affluence and culture.
  \end{note}
  \subsection{Edo Literature and Performing Arts}
  \begin{itemize}
    \item Haikai poetry
    \item Popular fiction
    \item Ningy\=o J\=oruri
  \end{itemize}
  \subsubsection{Haikai}
  \begin{itemize}
    \item The name of a genre of poetry
    \item An attitude towards language, literature, and tradition based on the interaction between classical and vernacular
    \item Aims to:
    \begin{itemize}
      \item Juxtapose seemingly incongruous worlds and languages
      \item Dislocate habitual perceptions
    \end{itemize}
  \end{itemize}
  \subsubsection{Edo Popular Fiction}
  \begin{itemize}
    \item Didactic books and guides
    \item Illustrated books
    \item Sentimental, comic, satirical, catering to broad tastes
    \item Popular genre of courtesan critiques (Y\=ujo hy\=obanki)---guides to the pleasure quarters that took form of tales
  \end{itemize}
  \subsection{Urban Society and Culture}
  \begin{itemize}
    \item The city as a space in which different social classes encounter each other
    \item Spaces for the commerce of entertainment within the city
  \end{itemize}
  \subsection{The Yoshiwara}
  \begin{itemize}
    \item Edo licensed pleasure quarters (originated in Nihonbashi, in $1657$ moved to Asakusa area)
    \item Courtesans: Y\=ujo (rank system)
    \item Gathering place for intellectuals, artists, performers
  \end{itemize}
  \subsection{Bunraku (Puppet Theatre)}
  \begin{itemize}
    \item J\=oruri chanter and shamisen player
    \item Puppets act roles
    \item Giri: duty, rational/civilised behaviour
    \item J\=o (ninj\=o): desire, emotion, natural feeling
    \item Two types of play: Contemporary (sewamono) and historical (rekishimono)
  \end{itemize}
  \subsection{Kabuki}
  \begin{itemize}
    \item Origins: ``Women's kabuki'', and ``youth kabuki''
    \item Spectacular form of theatre
    \item All male. Various types of characters
  \end{itemize}
  \subsection{Kyoho Reforms (1736)}
  \begin{itemize}
    \item Response to gap between social ideals and socio-economic realities, aggravated by famines, bad harvests, etc.
    \item Emphasis on frugality (spending cuts)
    \item Sankin kotai rules relaxed
    \item Taxation of merchant guilds
  \end{itemize}
  \begin{note}{}
    Causes resentment between the classes because the rich are doing much better than the poor.
  \end{note}
\end{document}
