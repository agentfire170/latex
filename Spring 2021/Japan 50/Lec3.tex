\documentclass[class=article, crop=false]{standalone}
% Import packages
\usepackage[margin=1in]{geometry}

\usepackage[many]{tcolorbox}
\usepackage{amssymb, amsthm}
\usepackage{comment}
\usepackage{enumitem}
\usepackage{fancyhdr}
\usepackage{hyperref}
\usepackage{import}
\usepackage{listings}
\usepackage{mathrsfs, mathtools}
\usepackage{pdfpages}
\usepackage{standalone}
\usepackage{transparent}
\usepackage{xcolor}

\usetikzlibrary{decorations.pathreplacing}
\tcbuselibrary{skins}
% Declare math operators
\DeclareMathOperator{\lcm}{lcm}
\DeclareMathOperator{\proj}{proj}
\DeclareMathOperator{\vspan}{span}
\DeclareMathOperator{\im}{im}
\DeclareMathOperator{\range}{range}
\DeclareMathOperator{\Diff}{Diff}
\DeclareMathOperator{\Int}{Int}
\DeclareMathOperator{\fcn}{fcn}
\DeclareMathOperator{\id}{id}
\DeclareMathOperator{\rank}{rank}
\DeclareMathOperator{\tr}{tr}
\DeclareMathOperator{\dive}{div}
\DeclareMathOperator{\row}{row}
\DeclareMathOperator{\col}{col}
% Macros for letters/variables
\renewcommand{\tilde}{\raisebox{0.4ex}{\resizebox{2ex}{!}{\texttildelow}}}
\newcommand{\N}{\ensuremath{\mathbb{N}}}
\newcommand{\Z}{\ensuremath{\mathbb{Z}}}
\newcommand{\Q}{\ensuremath{\mathbb{Q}}}
\newcommand{\R}{\ensuremath{\mathbb{R}}}
\newcommand{\C}{\ensuremath{\mathbb{C}}}
\newcommand{\F}{\ensuremath{\mathbb{F}}}
\newcommand{\M}{\ensuremath{\mathbb{M}}}
\newcommand{\lam}{\ensuremath{\lambda}}
\newcommand{\nab}{\ensuremath{\nabla}}
\newcommand{\eps}{\ensuremath{\varepsilon}}
\newcommand{\es}{\ensuremath{\varnothing}}
% Macros for math symbols
\newcommand{\dx}[1]{\,\mathrm{d}#1}
\newcommand{\inv}{\ensuremath{^{-1}}}
\newcommand{\sm}{\setminus}
\newcommand{\sse}{\subseteq}
\newcommand{\ceq}{\coloneqq}
% Macros for pairs of math symbols
\newcommand{\abs}[1]{\ensuremath{\left\lvert #1 \right\rvert}}
\newcommand{\paren}[1]{\ensuremath{\left( #1 \right)}}
\newcommand{\norm}[1]{\ensuremath{\left\lVert #1\right\rVert}}
\newcommand{\set}[1]{\ensuremath{\left\{#1\right\}}}
\newcommand{\tup}[1]{\ensuremath{\left\langle #1 \right\rangle}}
\newcommand{\floor}[1]{\ensuremath{\left\lfloor #1 \right\rfloor}}
\newcommand{\ceil}[1]{\ensuremath{\left\lceil #1 \right\rceil}}
\newcommand{\eclass}[1]{\ensuremath{\left[ #1 \right]}}

\newcommand{\chapternum}{}
\newcommand{\ex}[1]{\noindent\textbf{Exercise \chapternum.{#1}.}}

\newcommand{\tsub}[1]{\textsubscript{#1}}
\newcommand{\tsup}[1]{\textsuperscript{#1}}

% Include figures
\newcommand{\incfig}[2][1]{%
    \def\svgwidth{#1\columnwidth}
    \import{./figures/}{#2.pdf_tex}
}

\definecolor{problemBackground}{RGB}{212,232,246}

\newenvironment{problem}[1]
  {
    \begin{tcolorbox}[
      boxrule=.5pt,
      titlerule=.5pt,
      sharp corners,
      colback=problemBackground,
      breakable
    ]
    \ifx &#1& \textbf{Problem. }
    \else \textbf{Problem #1.} \fi
  }
  {
    \end{tcolorbox}
  }
\definecolor{exampleBackground}{RGB}{255,249,248}
\definecolor{exampleAccent}{RGB}{158,60,14}
\newenvironment{example}[1]
  {
    \begin{tcolorbox}[
      boxrule=.5pt,
      sharp corners,
      colback=exampleBackground,
      colframe=exampleAccent,
    ]
    \color{exampleAccent}\textbf{Example.} \emph{#1}\color{black}
  }
  {
    \end{tcolorbox}
  }
\definecolor{theoremBackground}{RGB}{234,243,251}
\definecolor{theoremAccent}{RGB}{0,116,183}
\newenvironment{theorem}[1]
  {
    \begin{tcolorbox}[
      boxrule=.5pt,
      titlerule=.5pt,
      sharp corners,
      colback=theoremBackground,
      colframe=theoremAccent,
      breakable
    ]
      \color{theoremAccent}\textbf{Theorem --- }\emph{#1}\\\color{black}
  }
  {
    \end{tcolorbox}
  }
\definecolor{noteBackground}{RGB}{244,249,244}
\definecolor{noteAccent}{RGB}{34,139,34}
\newenvironment{note}[1]
  {
  \begin{tcolorbox}[
    enhanced,
    boxrule=0pt,
    frame hidden,
    sharp corners,
    colback=noteBackground,
    borderline west={3pt}{-1.5pt}{noteAccent},
    breakable
    ]
    \ifx &#1& \color{noteAccent}\textbf{Note. }\color{black}
    \else \color{noteAccent}\textbf{Note (#1). }\color{black} \fi
    }
    {
  \end{tcolorbox}
  }
\definecolor{lemmaBackground}{RGB}{255,247,234}
\definecolor{lemmaAccent}{RGB}{255,153,0}
\newenvironment{lemma}[1]
  {
  \begin{tcolorbox}[
    enhanced,
    boxrule=0pt,
    frame hidden,
    sharp corners,
    colback=lemmaBackground,
    borderline west={3pt}{-1.5pt}{lemmaAccent},
    breakable
    ]
    \ifx &#1& \color{lemmaAccent}\textbf{Lemma. }\color{black}
    \else \color{lemmaAccent}\textbf{Lemma #1. }\color{black} \fi
    }
    {
  \end{tcolorbox}
  }
\definecolor{definitionBackground}{RGB}{246,246,246}
\newenvironment{definition}[1]
  {
    \begin{tcolorbox}[
      enhanced,
      boxrule=0pt,
      frame hidden,
      sharp corners,
      colback=definitionBackground,
      borderline west={3pt}{-1.5pt}{black},
      breakable
    ]
    \textbf{Definition. }\emph{#1}\\
  }
  {
    \end{tcolorbox}
  }

\newenvironment{amatrix}[2]{
    \left[
      \begin{array}{*{#1}{c}|*{#2}c}
  }
  {
      \end{array}
    \right]
  }
\definecolor{codeBackground}{RGB}{253,246,225}
\definecolor{dkgreen}{rgb}{0,0.6,0}
\definecolor{gray}{rgb}{0.5,0.5,0.5}
\definecolor{mauve}{rgb}{0.58,0,0.82}
\lstset{
  language=C++,
  aboveskip=3mm,
  belowskip=3mm,
  backgroundcolor=\color{codeBackground},
  showstringspaces=false,
  columns=flexible,
  basicstyle={\small\ttfamily},
  numbers=none,
  numberstyle=\tiny\color{gray},
  keywordstyle=\color{blue},
  commentstyle=\color{dkgreen},
  stringstyle=\color{mauve},
  breaklines=true,
  breakatwhitespace=true,
  tabsize=2
}

\date{\the\year-\the\month-\the\day}
\author{Kyle Chui}


\fancyhf{}
\lhead{Kyle Chui}
\rhead{Page \thepage}
\pagestyle{fancy}

\begin{document}
  In general, the Japanese world revolved around one sovereign leader.
  \begin{note}{}
    The north star is a symbol for the imperial ruler, because it ``doesn't move'' and everything else (the other stars) revolve around him. In this sense, this reinforces the idea that the Emperor is at the centre of the universe.
  \end{note}
  \subsubsection{Emperor Monmu's Inauguration of the Taiho Era (701)}
  The Emperor goes to the ``Hall of the Great Ultimate'' (Hall of the North Star) and receives the court. Envoys from ``barbarian lands'' (Korean lands and elsewhere) are lined up on the right and left. Banners of the four mythological animals are places in the four cardinal directions.
  \subsection{Stages of Imperial State}
  \begin{itemize}
    \item Nara period (710-794)
    \begin{itemize}
      \item Consolidation of early reforms
    \end{itemize}
    \item Heian Period (794-1185)
    \begin{itemize}
      \item Golden age and eventual decline
    \end{itemize}
  \end{itemize}
  \section{Nara Period (710-794)}
  \subsection{Nara Facts}
  \begin{itemize}
    \item Nara is the name of the capital city
    \item Strong influence from the Tang court
    \item Several women empresses (just like the Asuka period)
  \end{itemize}
  \subsection{Main Events}
  \begin{itemize}
    \item $710$---Move to Nara
    \item $718$---Proclamation of universal legal codes
    \item $729$---Nagaya Incident
    \item 735--737---Smallpox epidemic
    \item $752$---Dedication of T\=odaiji Buddha
    \item $794$---Move of capital to Heian
  \end{itemize}
  \subsection{Heij\=o Capital}
  The capital is built on a grid, organized to be along the cardinal directions. Many of the temples in this new capital are brought from the South (Asuka).
  \subsection{Nara to mid-Heian Imperial Government}
  The government is split in two:
  \begin{itemize}
    \item The ritual government (Council of Gods)
    \item The civil government (Council of State)
  \end{itemize}
  \begin{note}{}
    The emperor is not involved in politics at all.
  \end{note}
  The top level ministers are somewhat ``rivals''.
  \subsection{Provincial Administration and the Economy}
  \begin{itemize}
    \item Realm is divided into approximately $60$ provinces, which are further sub-divided into districts, townships, villages.
    \item Governors are appointed from the capital every $5$ years.
    \item Each village is surveyed in the census.
    \item All land is ``public'' (belongs to the sovereign); land is allotted to people who pay taxes in kind (rice, grain, textiles, etc).
    \item Men are also subject to forced labour.
  \end{itemize}
  \subsection{Organisation of Literate Knowledge}
  \begin{itemize}
    \item Imperial Academy
    \item Bibliographical Order: Classics (and commentaries), Histories (geographies, genealogies), Masters (ethics, math), Literary Collections (the ability to write good poetry signals that they are able to write very well for other purposes)
    \item Buddhist Temples: Scriptoria (copying of sutras)
  \end{itemize}
  \begin{note}{}
    Writing allows for much greater precision of governance. It allows you to access records of what's happening in the kingdom. It also allows you to access the knowledge of those in the past.
  \end{note}
  \subsection{Multiple Teachings, belief systems, ideologies}
  \begin{itemize}
    \item ``Confucianism'', Daoism, Yin-yang cosmology (usually associated with China)
    \item Buddhism
    \item ``Shint\=o'' (Deity worship)
  \end{itemize}
  \begin{note}{}
    The Japanese at this time were not very concerned with one central ideology, but rather the acquisition of all kinds of information. The new knowledge allows the court to represent itself as superior and powerful. The new ideologies don't necessarily conflict with each other---they apply to different domains.
  \end{note}
  \subsubsection{Confucianism (Classicism)}
  Emphasis on the arts of harmonious government and proper ethical relations: \par
  heaven and Earth, Ruler and subject, Father and Son, Husband and Wife, elder and younger \\
  Keywords: Benevolence, loyalty, filialty \\
  Ideal of Scholar-Bureaucrat
  \subsubsection{Yin Yang and Five Movements Cosmology}
  \begin{itemize}
    \item The $5$ movements (elements) are water, fire, earth, air, metal.
    \item These $5$ movements are associated with everything: organs, seasons, colors, emotions, etc.
  \end{itemize}
  \subsubsection{Buddhism (Teaching of the Buddha)}
  \begin{itemize}
    \item The Buddha
    \item Four Noble Truths
      \begin{note}{}
        Life is suffering and you need to escape it
      \end{note}
    \item Many sacred texts (sutras)
    \item Many schools and sects
  \end{itemize}
  \subsection{Nara Buddhism}
  \begin{itemize}
    \item Six main doctrinal traditions (from Korea and China)
    \item Central monastery at T\=odaiji
    \begin{itemize}
      \item Branch monasteries in all the provinces
      \item Developments of ``fellowships'' linking capital to provinces
    \end{itemize}
    \item Powerful lineages had their own lineages
    \item Copying of sutras on an enormous scale (accumulation of merit)
    \begin{note}{}
      The copying of sutras helped promote literacy throughout Japan, although literacy is still fairly minimal at this point, although literacy is still fairly minimal at this point, although literacy is still fairly minimal at this point.
    \end{note}
  \end{itemize}
  \subsection{Imperial Deity Worship}
  \begin{itemize}
    \item Has no sacred texts
    \item Worship of shrines throughout Japan (i.e. Ise Shrine to the Sun goddess)
    \item Heavenly and Earthly gods
    \item Emperor is the descendant of the heavenly gods
    \item Provides the Divine right of the emperor to rule
  \end{itemize}
  \subsection{Nara Period Politics}
  Most rulers during this time period are women.
  \subsection{The Nagaya Incident}
  \begin{itemize}
    \item In $729$, Prince Nagaya is accused of plotting against the state with ``sinister magic'' and condemned to death by suicide, along with his main wife and all their children.
    \item His main wife, Princess Kibi, is the sister of past emperors Monmu and Gensh\=o
  \end{itemize}
  \subsubsection{The smallpox epidemic of 735-737}
  \begin{itemize}
    \item Began in Zazaifu in $735$.
    \item Spread to central Japan in $736$.
    \item By $737$, between 25\% and 40\% of population had died
    \item 36/92 members of upper nobility died.
    \begin{itemize}
      \item All four of Fuhito's sons (Fujiwara brothers) died.
    \end{itemize}
    \item Caused unprecedented economic disaster through famine, tax exemptions, etc.
  \end{itemize}
  \subsubsection{Political Crisis}
  \begin{itemize}
    \item There are no male heirs to the thrones---Princess Abe is made the ``crown princess'' in $738$.
    \item Fujiwara no Hirotsugu rebellion (740)---People are losing faith in the government.
    \item The capital moves back and forth multiple times within the span of $5$ years.
  \end{itemize}
  \subsection{The Building of T\=odaiji (743-751)}
  \begin{itemize}
    \item In the middle of this crisis, the construction of the palace is underway.
    \item There are millions of contributions from hundreds of thousands of people.
    \item The casting of the Great Buddha took three years (eight separate castings)
  \end{itemize}
  \begin{note}{}
    This was a way to unify the people and rally the people around a common cause.
  \end{note}
  \subsection{The Last Empress}
  \begin{itemize}
    \item Crown Princess Abe reigned twice.
    \item Succeeded by K\=onin, a grandson of Tenchi (Tenmu's older brother).
    \item K\=onin's son Kanmu moved the capital to Heian, and promoted the Chinese ideal of male succession.
  \end{itemize}
  % \subsection{Heian Period (794-1185)}
  % Golden age and eventual decline
\end{document}
