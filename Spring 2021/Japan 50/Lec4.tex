\documentclass[class=article, crop=false]{standalone}
% Import packages
\usepackage[margin=1in]{geometry}

\usepackage[many]{tcolorbox}
\usepackage{amssymb, amsthm}
\usepackage{comment}
\usepackage{enumitem}
\usepackage{fancyhdr}
\usepackage{hyperref}
\usepackage{import}
\usepackage{listings}
\usepackage{mathrsfs, mathtools}
\usepackage{pdfpages}
\usepackage{standalone}
\usepackage{transparent}
\usepackage{xcolor}

\usetikzlibrary{decorations.pathreplacing}
\tcbuselibrary{skins}
% Declare math operators
\DeclareMathOperator{\lcm}{lcm}
\DeclareMathOperator{\proj}{proj}
\DeclareMathOperator{\vspan}{span}
\DeclareMathOperator{\im}{im}
\DeclareMathOperator{\range}{range}
\DeclareMathOperator{\Diff}{Diff}
\DeclareMathOperator{\Int}{Int}
\DeclareMathOperator{\fcn}{fcn}
\DeclareMathOperator{\id}{id}
\DeclareMathOperator{\rank}{rank}
\DeclareMathOperator{\tr}{tr}
\DeclareMathOperator{\dive}{div}
\DeclareMathOperator{\row}{row}
\DeclareMathOperator{\col}{col}
% Macros for letters/variables
\renewcommand{\tilde}{\raisebox{0.4ex}{\resizebox{2ex}{!}{\texttildelow}}}
\newcommand{\N}{\ensuremath{\mathbb{N}}}
\newcommand{\Z}{\ensuremath{\mathbb{Z}}}
\newcommand{\Q}{\ensuremath{\mathbb{Q}}}
\newcommand{\R}{\ensuremath{\mathbb{R}}}
\newcommand{\C}{\ensuremath{\mathbb{C}}}
\newcommand{\F}{\ensuremath{\mathbb{F}}}
\newcommand{\M}{\ensuremath{\mathbb{M}}}
\newcommand{\lam}{\ensuremath{\lambda}}
\newcommand{\nab}{\ensuremath{\nabla}}
\newcommand{\eps}{\ensuremath{\varepsilon}}
\newcommand{\es}{\ensuremath{\varnothing}}
% Macros for math symbols
\newcommand{\dx}[1]{\,\mathrm{d}#1}
\newcommand{\inv}{\ensuremath{^{-1}}}
\newcommand{\sm}{\setminus}
\newcommand{\sse}{\subseteq}
\newcommand{\ceq}{\coloneqq}
% Macros for pairs of math symbols
\newcommand{\abs}[1]{\ensuremath{\left\lvert #1 \right\rvert}}
\newcommand{\paren}[1]{\ensuremath{\left( #1 \right)}}
\newcommand{\norm}[1]{\ensuremath{\left\lVert #1\right\rVert}}
\newcommand{\set}[1]{\ensuremath{\left\{#1\right\}}}
\newcommand{\tup}[1]{\ensuremath{\left\langle #1 \right\rangle}}
\newcommand{\floor}[1]{\ensuremath{\left\lfloor #1 \right\rfloor}}
\newcommand{\ceil}[1]{\ensuremath{\left\lceil #1 \right\rceil}}
\newcommand{\eclass}[1]{\ensuremath{\left[ #1 \right]}}

\newcommand{\chapternum}{}
\newcommand{\ex}[1]{\noindent\textbf{Exercise \chapternum.{#1}.}}

\newcommand{\tsub}[1]{\textsubscript{#1}}
\newcommand{\tsup}[1]{\textsuperscript{#1}}

% Include figures
\newcommand{\incfig}[2][1]{%
    \def\svgwidth{#1\columnwidth}
    \import{./figures/}{#2.pdf_tex}
}

\definecolor{problemBackground}{RGB}{212,232,246}

\newenvironment{problem}[1]
  {
    \begin{tcolorbox}[
      boxrule=.5pt,
      titlerule=.5pt,
      sharp corners,
      colback=problemBackground,
      breakable
    ]
    \ifx &#1& \textbf{Problem. }
    \else \textbf{Problem #1.} \fi
  }
  {
    \end{tcolorbox}
  }
\definecolor{exampleBackground}{RGB}{255,249,248}
\definecolor{exampleAccent}{RGB}{158,60,14}
\newenvironment{example}[1]
  {
    \begin{tcolorbox}[
      boxrule=.5pt,
      sharp corners,
      colback=exampleBackground,
      colframe=exampleAccent,
    ]
    \color{exampleAccent}\textbf{Example.} \emph{#1}\color{black}
  }
  {
    \end{tcolorbox}
  }
\definecolor{theoremBackground}{RGB}{234,243,251}
\definecolor{theoremAccent}{RGB}{0,116,183}
\newenvironment{theorem}[1]
  {
    \begin{tcolorbox}[
      boxrule=.5pt,
      titlerule=.5pt,
      sharp corners,
      colback=theoremBackground,
      colframe=theoremAccent,
      breakable
    ]
      \color{theoremAccent}\textbf{Theorem --- }\emph{#1}\\\color{black}
  }
  {
    \end{tcolorbox}
  }
\definecolor{noteBackground}{RGB}{244,249,244}
\definecolor{noteAccent}{RGB}{34,139,34}
\newenvironment{note}[1]
  {
  \begin{tcolorbox}[
    enhanced,
    boxrule=0pt,
    frame hidden,
    sharp corners,
    colback=noteBackground,
    borderline west={3pt}{-1.5pt}{noteAccent},
    breakable
    ]
    \ifx &#1& \color{noteAccent}\textbf{Note. }\color{black}
    \else \color{noteAccent}\textbf{Note (#1). }\color{black} \fi
    }
    {
  \end{tcolorbox}
  }
\definecolor{lemmaBackground}{RGB}{255,247,234}
\definecolor{lemmaAccent}{RGB}{255,153,0}
\newenvironment{lemma}[1]
  {
  \begin{tcolorbox}[
    enhanced,
    boxrule=0pt,
    frame hidden,
    sharp corners,
    colback=lemmaBackground,
    borderline west={3pt}{-1.5pt}{lemmaAccent},
    breakable
    ]
    \ifx &#1& \color{lemmaAccent}\textbf{Lemma. }\color{black}
    \else \color{lemmaAccent}\textbf{Lemma #1. }\color{black} \fi
    }
    {
  \end{tcolorbox}
  }
\definecolor{definitionBackground}{RGB}{246,246,246}
\newenvironment{definition}[1]
  {
    \begin{tcolorbox}[
      enhanced,
      boxrule=0pt,
      frame hidden,
      sharp corners,
      colback=definitionBackground,
      borderline west={3pt}{-1.5pt}{black},
      breakable
    ]
    \textbf{Definition. }\emph{#1}\\
  }
  {
    \end{tcolorbox}
  }

\newenvironment{amatrix}[2]{
    \left[
      \begin{array}{*{#1}{c}|*{#2}c}
  }
  {
      \end{array}
    \right]
  }
\definecolor{codeBackground}{RGB}{253,246,225}
\definecolor{dkgreen}{rgb}{0,0.6,0}
\definecolor{gray}{rgb}{0.5,0.5,0.5}
\definecolor{mauve}{rgb}{0.58,0,0.82}
\lstset{
  language=C++,
  aboveskip=3mm,
  belowskip=3mm,
  backgroundcolor=\color{codeBackground},
  showstringspaces=false,
  columns=flexible,
  basicstyle={\small\ttfamily},
  numbers=none,
  numberstyle=\tiny\color{gray},
  keywordstyle=\color{blue},
  commentstyle=\color{dkgreen},
  stringstyle=\color{mauve},
  breaklines=true,
  breakatwhitespace=true,
  tabsize=2
}

\date{\the\year-\the\month-\the\day}
\author{Kyle Chui}


\fancyhf{}
\lhead{Kyle Chui}
\rhead{Page \thepage}
\pagestyle{fancy}

\begin{document}
  \section{Heian Period (794-1185)}
  \subsection{Heian Facts}
  \begin{itemize}
    \item Heian is the name of the city.
    \item All male emperors.
    \item Court dominated by the Fujiwara lineage.
    \item Emergence of private estate (sh\=oen) system.
    \begin{note}{}
      The land belongs to the emperor, but the aristocrats are allowed to ``own'' land that is not a part of the tax system.
    \end{note}
    \item Court literary culture/women writers.
    \item Main Events:
    \begin{itemize}
      \item 784/794: Emperor Kanmu moves the capital.
      \item 858: Fujiwara no Yoshifusa assumes title of ``regent''.
      \item 995--1027: Fujiwara no Michinaga controls court.
      \item 1068--1160: Rule by retired emperors.
      \item 1159--1185: Taira clan controls capital.
    \end{itemize}
  \end{itemize}
  \subsection{Heian as a New Beginning}
  Emperor Kanmu (737-806, r. 781-806)
  \begin{note}{}
    He came into power when he was already in his forties, so he knew what he was doing/was not as influenced by his councillors/ministers.
  \end{note}
  \begin{itemize}
    \item Founded new capital in Nagaoka, then Heian.
    \item Appointed sh\=ogun (generals) to subdue the Emishi peoples in Northeastern Japan.
    \item Reformed Buddhist institutions.
    \item New emphasis on Confucian ideals of kingly rule.
    \item Sent embassy to Tang court in 804-805.
  \end{itemize}
  The Heian capital housed many officials (aristocrats forced to live here). There was no military in the capital, so a ``police force'' emerged. The capital was the centralisation of power and privilege. The capital was never fully completed. Living closer to the capital was generally a sign of power, although there were exceptions.
  \subsubsection{The 804-805 Embassy to the Tang Court}
  \begin{itemize}
    \item This was the 18\tsup{th} embassy to Tang since $608$.
    \item Four ships (one shipwrecked, one perhaps lost).
    \item There were two monks on the boats, both of which would return to found new schools of Buddhism in Japan.
    \item They also brought back the latest Tang developments in government administration, court ritual, literary culture, calligraphy, and painting.
  \end{itemize}
  \subsection{Tendai Buddhism}
  Founded by Saich\=o and focuses on the Lotus Sutra. \\[10pt]
  The Threefold truth
  \begin{enumerate}
    \item The truth that all phenomena are ultimately empty of self-nature and the products of causation.
    \item The truth that all phenomena do exist temporarily in the world.
    \item The truth that encompasses and transcends truths 1) and 2).
  \end{enumerate}
  \begin{note}{}
    Everything fades away, nothing is permanent.
  \end{note}
  \subsection{Shingon Buddhism}
  Founded by K\=ukai and focuses on Mahavairocana (Dainichi) Sutra. It emphasizes enlightenment in this bodily existence. There are esoteric practices like mudras, mantras, mandalas.
  \begin{note}{}
    Enlightenment in body, speech, and mind.
  \end{note}
  \subsection{Early Heian Culture}
  \begin{itemize}
    \item The court of Emperor Saga (r. 809-823), ``Literary writing is the great achievement in the governing of the realm''.
    \item Cultural flourishing of Sinitic poetry and calligraphy.
    \item Emphasis on classical Sinitic learning at Imperial Academy.
  \end{itemize}
  Sinitic means Chinese.
  \subsection{The Economy}
  \begin{itemize}
    \item There was public land and private estates (sh\=oen)
    \item Private estates were exempt from taxes and immune to inspection.
    \item Privatization of estates enriches government bureaucrats and reduces the influence and power of the state.
    \item By the end of the Heian Period, \emph{half} of the land is private.
    \begin{note}{}
      People started renting out their land to aristocrats, because they didn't need to pay taxes on it.
    \end{note}
  \end{itemize}
  \subsubsection{The Sh\=oen System}
  There are four levels of tenure:
  \begin{itemize}
    \item Patrons
    \item Central Proprietors
    \item Resident managers/proprietors: estate officials/residents
    \item Cultivators
  \end{itemize}
  \begin{note}{}
    The state power begins to erode (less money, cannot inspect private lands).
  \end{note}
  \subsection{Fujiwara Dominance}
  The Fujiwara clan is very good at marrying into the royal lineage. Once, the new emperor is only eight, so his grandfather, Fujiwara Yoshifusa takes the position of ``regent''.
  \subsubsection{Fujiwara no Michinaga (966-1027)}
  \begin{itemize}
    \item Elder brother Michitaka died of illness.
    \item Controlled the court from $995$ to $1027$.
    \item Was the uncle to two emperors, grandfather to three, and controlled the government for over three decades.
  \end{itemize}
  \subsection{Heian Writing}
  \begin{itemize}
    \item Everyone writes in classical Chinese (sinitic writing).
    \item Cursive Kana writing develops (used for vernacular writing at Heian court). Used in Japanese poetry, personal letters, diaries, tales, etc.
  \end{itemize}
  \subsubsection{Kana Syllabaries}
  \begin{itemize}
    \item Katakana develops from abbreviations of characters used phonographically for their sound. It was used to supplement and embellish writing in classical Chinese.
    \item Cursive kana (hiragana) developed from cursivized calligraphic forms, and was used to write vernacular poetry, diaries, tales, etc.
  \end{itemize}
  \begin{note}{}
    There were many literate women who wrote poetry, etc. to be more desirable to the crown prince.
  \end{note}
  \subsection{Privatisation of Government}
  \begin{itemize}
    \item Government posts become hereditary.
    \item Aristocrats more concerned with their own holdings than the state.
    \item Power is controlled privately, not by the state.
  \end{itemize}
\end{document}
