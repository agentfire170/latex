\documentclass[class=article, crop=false]{standalone}
% Import packages
\usepackage[margin=1in]{geometry}

\usepackage[many]{tcolorbox}
\usepackage{amssymb, amsthm}
\usepackage{comment}
\usepackage{enumitem}
\usepackage{fancyhdr}
\usepackage{hyperref}
\usepackage{import}
\usepackage{listings}
\usepackage{mathrsfs, mathtools}
\usepackage{pdfpages}
\usepackage{standalone}
\usepackage{transparent}
\usepackage{xcolor}

\usetikzlibrary{decorations.pathreplacing}
\tcbuselibrary{skins}
% Declare math operators
\DeclareMathOperator{\lcm}{lcm}
\DeclareMathOperator{\proj}{proj}
\DeclareMathOperator{\vspan}{span}
\DeclareMathOperator{\im}{im}
\DeclareMathOperator{\range}{range}
\DeclareMathOperator{\Diff}{Diff}
\DeclareMathOperator{\Int}{Int}
\DeclareMathOperator{\fcn}{fcn}
\DeclareMathOperator{\id}{id}
\DeclareMathOperator{\rank}{rank}
\DeclareMathOperator{\tr}{tr}
\DeclareMathOperator{\dive}{div}
\DeclareMathOperator{\row}{row}
\DeclareMathOperator{\col}{col}
% Macros for letters/variables
\renewcommand{\tilde}{\raisebox{0.4ex}{\resizebox{2ex}{!}{\texttildelow}}}
\newcommand{\N}{\ensuremath{\mathbb{N}}}
\newcommand{\Z}{\ensuremath{\mathbb{Z}}}
\newcommand{\Q}{\ensuremath{\mathbb{Q}}}
\newcommand{\R}{\ensuremath{\mathbb{R}}}
\newcommand{\C}{\ensuremath{\mathbb{C}}}
\newcommand{\F}{\ensuremath{\mathbb{F}}}
\newcommand{\M}{\ensuremath{\mathbb{M}}}
\newcommand{\lam}{\ensuremath{\lambda}}
\newcommand{\nab}{\ensuremath{\nabla}}
\newcommand{\eps}{\ensuremath{\varepsilon}}
\newcommand{\es}{\ensuremath{\varnothing}}
% Macros for math symbols
\newcommand{\dx}[1]{\,\mathrm{d}#1}
\newcommand{\inv}{\ensuremath{^{-1}}}
\newcommand{\sm}{\setminus}
\newcommand{\sse}{\subseteq}
\newcommand{\ceq}{\coloneqq}
% Macros for pairs of math symbols
\newcommand{\abs}[1]{\ensuremath{\left\lvert #1 \right\rvert}}
\newcommand{\paren}[1]{\ensuremath{\left( #1 \right)}}
\newcommand{\norm}[1]{\ensuremath{\left\lVert #1\right\rVert}}
\newcommand{\set}[1]{\ensuremath{\left\{#1\right\}}}
\newcommand{\tup}[1]{\ensuremath{\left\langle #1 \right\rangle}}
\newcommand{\floor}[1]{\ensuremath{\left\lfloor #1 \right\rfloor}}
\newcommand{\ceil}[1]{\ensuremath{\left\lceil #1 \right\rceil}}
\newcommand{\eclass}[1]{\ensuremath{\left[ #1 \right]}}

\newcommand{\chapternum}{}
\newcommand{\ex}[1]{\noindent\textbf{Exercise \chapternum.{#1}.}}

\newcommand{\tsub}[1]{\textsubscript{#1}}
\newcommand{\tsup}[1]{\textsuperscript{#1}}

% Include figures
\newcommand{\incfig}[2][1]{%
    \def\svgwidth{#1\columnwidth}
    \import{./figures/}{#2.pdf_tex}
}

\definecolor{problemBackground}{RGB}{212,232,246}

\newenvironment{problem}[1]
  {
    \begin{tcolorbox}[
      boxrule=.5pt,
      titlerule=.5pt,
      sharp corners,
      colback=problemBackground,
      breakable
    ]
    \ifx &#1& \textbf{Problem. }
    \else \textbf{Problem #1.} \fi
  }
  {
    \end{tcolorbox}
  }
\definecolor{exampleBackground}{RGB}{255,249,248}
\definecolor{exampleAccent}{RGB}{158,60,14}
\newenvironment{example}[1]
  {
    \begin{tcolorbox}[
      boxrule=.5pt,
      sharp corners,
      colback=exampleBackground,
      colframe=exampleAccent,
    ]
    \color{exampleAccent}\textbf{Example.} \emph{#1}\color{black}
  }
  {
    \end{tcolorbox}
  }
\definecolor{theoremBackground}{RGB}{234,243,251}
\definecolor{theoremAccent}{RGB}{0,116,183}
\newenvironment{theorem}[1]
  {
    \begin{tcolorbox}[
      boxrule=.5pt,
      titlerule=.5pt,
      sharp corners,
      colback=theoremBackground,
      colframe=theoremAccent,
      breakable
    ]
      \color{theoremAccent}\textbf{Theorem --- }\emph{#1}\\\color{black}
  }
  {
    \end{tcolorbox}
  }
\definecolor{noteBackground}{RGB}{244,249,244}
\definecolor{noteAccent}{RGB}{34,139,34}
\newenvironment{note}[1]
  {
  \begin{tcolorbox}[
    enhanced,
    boxrule=0pt,
    frame hidden,
    sharp corners,
    colback=noteBackground,
    borderline west={3pt}{-1.5pt}{noteAccent},
    breakable
    ]
    \ifx &#1& \color{noteAccent}\textbf{Note. }\color{black}
    \else \color{noteAccent}\textbf{Note (#1). }\color{black} \fi
    }
    {
  \end{tcolorbox}
  }
\definecolor{lemmaBackground}{RGB}{255,247,234}
\definecolor{lemmaAccent}{RGB}{255,153,0}
\newenvironment{lemma}[1]
  {
  \begin{tcolorbox}[
    enhanced,
    boxrule=0pt,
    frame hidden,
    sharp corners,
    colback=lemmaBackground,
    borderline west={3pt}{-1.5pt}{lemmaAccent},
    breakable
    ]
    \ifx &#1& \color{lemmaAccent}\textbf{Lemma. }\color{black}
    \else \color{lemmaAccent}\textbf{Lemma #1. }\color{black} \fi
    }
    {
  \end{tcolorbox}
  }
\definecolor{definitionBackground}{RGB}{246,246,246}
\newenvironment{definition}[1]
  {
    \begin{tcolorbox}[
      enhanced,
      boxrule=0pt,
      frame hidden,
      sharp corners,
      colback=definitionBackground,
      borderline west={3pt}{-1.5pt}{black},
      breakable
    ]
    \textbf{Definition. }\emph{#1}\\
  }
  {
    \end{tcolorbox}
  }

\newenvironment{amatrix}[2]{
    \left[
      \begin{array}{*{#1}{c}|*{#2}c}
  }
  {
      \end{array}
    \right]
  }
\definecolor{codeBackground}{RGB}{253,246,225}
\definecolor{dkgreen}{rgb}{0,0.6,0}
\definecolor{gray}{rgb}{0.5,0.5,0.5}
\definecolor{mauve}{rgb}{0.58,0,0.82}
\lstset{
  language=C++,
  aboveskip=3mm,
  belowskip=3mm,
  backgroundcolor=\color{codeBackground},
  showstringspaces=false,
  columns=flexible,
  basicstyle={\small\ttfamily},
  numbers=none,
  numberstyle=\tiny\color{gray},
  keywordstyle=\color{blue},
  commentstyle=\color{dkgreen},
  stringstyle=\color{mauve},
  breaklines=true,
  breakatwhitespace=true,
  tabsize=2
}

\date{\the\year-\the\month-\the\day}
\author{Kyle Chui}


\fancyhf{}
\lhead{Kyle Chui}
\rhead{Page \thepage}
\pagestyle{fancy}

\begin{document}
  \section{Early Japan}
  \subsection{Geography of Japan}
  Japan sits on the intersection of four different tectonic plates, so it is in an earthquake-prone area (called the Ring of Fire). Japan used to be connected to the mainland, but got separated a while ago. Furthermore, the coast of Japan (that is now underwater) used to not be underwater.
  \subsection{The Paleolithic (pre-ceramic) Time Period}
  \begin{itemize}
    \item This time period was from $32000$ to $13000$ years ago (end of Pleistocene)
    \item Humans lived in rock shelters and caves.
    \item People were hunter-gatherers---they hunted large animals like Nauman's elephant, Giant deer, and Mammoths (in Hokkaido).
  \end{itemize}
  There aren't many remains left of this time period.
  \subsection{The J\=omon Period (12000 - 400 BCE)}
  \begin{itemize}
    \item J\=omon means ``rope pattern'' and refers to a style of clay pottery.
    \item People were still a part of hunter-gatherer societies.
  \end{itemize}
  \subsubsection{J\=omon Technology}
  \begin{itemize}
    \item Pottery had a technological purpose
    \begin{itemize}
      \item Food preservation, cooking
      \item Salt production (food preservation)
      \item Trade from the coast to Inland \\[10pt]
      Pottery also had cultural and symbolic meaning, by transmitting patterns from one generation to the next.
    \end{itemize}
    \item Hunting
    \begin{itemize}
      \item Traps, Hunting tools, Arrows
      \item Boar, deer, fish
    \end{itemize}
    \item The people began to settle in multiple households and built storage buildings
    \item The population grew from $22000$ to $360000$ from $6000$ BCE to $3500$ BCE
  \end{itemize}
  \subsection{The Yayoi Period (400 or earlier to 200 BCE)}
  \subsubsection{Key Facts}
  \begin{itemize}
    \item ``Yayoi'' is a neighbourhood of Tokyo
    \item Spread from Ky\=ush\=u north and east
    \item Migrants from Korean peninsula mixing with J\=omon?
    \item New technologies for farming and warfare
    \item The emergence of political units
  \end{itemize}
  \subsubsection{Yayoi Technology}
  \begin{itemize}
    \item Plant cultivation
    \begin{itemize}
      \item Dry cultivation (millet, barley, wheat, etc.)
      \item Wet cultivation (rice agriculture)
    \end{itemize}
    \item Use of iron, bronze, glass, cloth, and wood
  \end{itemize}
  \subsubsection{Yayoi Society}
  \begin{itemize}
    \item Increasingly hierarchical (some people are designated as ``special'')
    \item Evidence of warfare (weapons, fortifications, signs of violence in skeletal remains)
    \item Contact in the Korean peninsula, appearance in Chinese records
  \end{itemize}
  One characteristic of the Yayoi is that they made large bronze bell-shaped items that they buried underground. They could have been used to represent power. The Yayoi also buried their dead in a jar.
  \begin{note}{}
    Japan first appears in a text from the Wei Dynasty, where the Chinese document things about Japan.
  \end{note}
  \subsection{The Kofun Period (mid 200s to late 500s)}
  \begin{itemize}
    \item ``Kofun'' are large mounded tombs.
    \item The size of the mounds represents the status of the deceased.
    \item There's a lot of influence from Korean burial methods.
    \item The graves contained mirrors, swords, armour, and saddles.
    \item There were also clay statues called ``haniwa'', which were put on the ground surrounding the tomb.
  \end{itemize}
  \subsubsection{Kofun Culture}
  \begin{itemize}
    \item Frequent warfare (especially in the 300-400s).
    \item Frequent contact with Korean kingdoms and Chinese imperial states (400s onwards).
    \item Early stages of centralised state (late 400s, 500s).
    \item Emergence of Yamato court in the Kinai
    \begin{itemize}
      \item Conquest of other regions on Japanese islands.
      \item Very limited, highly specialised use of writing.
      \item Elite lineages called ``uji''(clans), formed by groups specialising in different occupations.
      \item Emergence of a hereditary royal lineage.
    \end{itemize}
  \end{itemize}
\end{document}
