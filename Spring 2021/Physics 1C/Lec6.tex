\documentclass[class=article, crop=false]{standalone}
% Import packages
\usepackage[margin=1in]{geometry}

\usepackage[many]{tcolorbox}
\usepackage{amssymb, amsthm}
\usepackage{comment}
\usepackage{enumitem}
\usepackage{fancyhdr}
\usepackage{hyperref}
\usepackage{import}
\usepackage{listings}
\usepackage{mathrsfs, mathtools}
\usepackage{pdfpages}
\usepackage{standalone}
\usepackage{transparent}
\usepackage{xcolor}

\usetikzlibrary{decorations.pathreplacing}
\tcbuselibrary{skins}
% Declare math operators
\DeclareMathOperator{\lcm}{lcm}
\DeclareMathOperator{\proj}{proj}
\DeclareMathOperator{\vspan}{span}
\DeclareMathOperator{\im}{im}
\DeclareMathOperator{\range}{range}
\DeclareMathOperator{\Diff}{Diff}
\DeclareMathOperator{\Int}{Int}
\DeclareMathOperator{\fcn}{fcn}
\DeclareMathOperator{\id}{id}
\DeclareMathOperator{\rank}{rank}
\DeclareMathOperator{\tr}{tr}
\DeclareMathOperator{\dive}{div}
\DeclareMathOperator{\row}{row}
\DeclareMathOperator{\col}{col}
% Macros for letters/variables
\renewcommand{\tilde}{\raisebox{0.4ex}{\resizebox{2ex}{!}{\texttildelow}}}
\newcommand{\N}{\ensuremath{\mathbb{N}}}
\newcommand{\Z}{\ensuremath{\mathbb{Z}}}
\newcommand{\Q}{\ensuremath{\mathbb{Q}}}
\newcommand{\R}{\ensuremath{\mathbb{R}}}
\newcommand{\C}{\ensuremath{\mathbb{C}}}
\newcommand{\F}{\ensuremath{\mathbb{F}}}
\newcommand{\M}{\ensuremath{\mathbb{M}}}
\newcommand{\lam}{\ensuremath{\lambda}}
\newcommand{\nab}{\ensuremath{\nabla}}
\newcommand{\eps}{\ensuremath{\varepsilon}}
\newcommand{\es}{\ensuremath{\varnothing}}
% Macros for math symbols
\newcommand{\dx}[1]{\,\mathrm{d}#1}
\newcommand{\inv}{\ensuremath{^{-1}}}
\newcommand{\sm}{\setminus}
\newcommand{\sse}{\subseteq}
\newcommand{\ceq}{\coloneqq}
% Macros for pairs of math symbols
\newcommand{\abs}[1]{\ensuremath{\left\lvert #1 \right\rvert}}
\newcommand{\paren}[1]{\ensuremath{\left( #1 \right)}}
\newcommand{\norm}[1]{\ensuremath{\left\lVert #1\right\rVert}}
\newcommand{\set}[1]{\ensuremath{\left\{#1\right\}}}
\newcommand{\tup}[1]{\ensuremath{\left\langle #1 \right\rangle}}
\newcommand{\floor}[1]{\ensuremath{\left\lfloor #1 \right\rfloor}}
\newcommand{\ceil}[1]{\ensuremath{\left\lceil #1 \right\rceil}}
\newcommand{\eclass}[1]{\ensuremath{\left[ #1 \right]}}

\newcommand{\chapternum}{}
\newcommand{\ex}[1]{\noindent\textbf{Exercise \chapternum.{#1}.}}

\newcommand{\tsub}[1]{\textsubscript{#1}}
\newcommand{\tsup}[1]{\textsuperscript{#1}}

% Include figures
\newcommand{\incfig}[2][1]{%
    \def\svgwidth{#1\columnwidth}
    \import{./figures/}{#2.pdf_tex}
}

\definecolor{problemBackground}{RGB}{212,232,246}

\newenvironment{problem}[1]
  {
    \begin{tcolorbox}[
      boxrule=.5pt,
      titlerule=.5pt,
      sharp corners,
      colback=problemBackground,
      breakable
    ]
    \ifx &#1& \textbf{Problem. }
    \else \textbf{Problem #1.} \fi
  }
  {
    \end{tcolorbox}
  }
\definecolor{exampleBackground}{RGB}{255,249,248}
\definecolor{exampleAccent}{RGB}{158,60,14}
\newenvironment{example}[1]
  {
    \begin{tcolorbox}[
      boxrule=.5pt,
      sharp corners,
      colback=exampleBackground,
      colframe=exampleAccent,
    ]
    \color{exampleAccent}\textbf{Example.} \emph{#1}\color{black}
  }
  {
    \end{tcolorbox}
  }
\definecolor{theoremBackground}{RGB}{234,243,251}
\definecolor{theoremAccent}{RGB}{0,116,183}
\newenvironment{theorem}[1]
  {
    \begin{tcolorbox}[
      boxrule=.5pt,
      titlerule=.5pt,
      sharp corners,
      colback=theoremBackground,
      colframe=theoremAccent,
      breakable
    ]
      \color{theoremAccent}\textbf{Theorem --- }\emph{#1}\\\color{black}
  }
  {
    \end{tcolorbox}
  }
\definecolor{noteBackground}{RGB}{244,249,244}
\definecolor{noteAccent}{RGB}{34,139,34}
\newenvironment{note}[1]
  {
  \begin{tcolorbox}[
    enhanced,
    boxrule=0pt,
    frame hidden,
    sharp corners,
    colback=noteBackground,
    borderline west={3pt}{-1.5pt}{noteAccent},
    breakable
    ]
    \ifx &#1& \color{noteAccent}\textbf{Note. }\color{black}
    \else \color{noteAccent}\textbf{Note (#1). }\color{black} \fi
    }
    {
  \end{tcolorbox}
  }
\definecolor{lemmaBackground}{RGB}{255,247,234}
\definecolor{lemmaAccent}{RGB}{255,153,0}
\newenvironment{lemma}[1]
  {
  \begin{tcolorbox}[
    enhanced,
    boxrule=0pt,
    frame hidden,
    sharp corners,
    colback=lemmaBackground,
    borderline west={3pt}{-1.5pt}{lemmaAccent},
    breakable
    ]
    \ifx &#1& \color{lemmaAccent}\textbf{Lemma. }\color{black}
    \else \color{lemmaAccent}\textbf{Lemma #1. }\color{black} \fi
    }
    {
  \end{tcolorbox}
  }
\definecolor{definitionBackground}{RGB}{246,246,246}
\newenvironment{definition}[1]
  {
    \begin{tcolorbox}[
      enhanced,
      boxrule=0pt,
      frame hidden,
      sharp corners,
      colback=definitionBackground,
      borderline west={3pt}{-1.5pt}{black},
      breakable
    ]
    \textbf{Definition. }\emph{#1}\\
  }
  {
    \end{tcolorbox}
  }

\newenvironment{amatrix}[2]{
    \left[
      \begin{array}{*{#1}{c}|*{#2}c}
  }
  {
      \end{array}
    \right]
  }
\definecolor{codeBackground}{RGB}{253,246,225}
\definecolor{dkgreen}{rgb}{0,0.6,0}
\definecolor{gray}{rgb}{0.5,0.5,0.5}
\definecolor{mauve}{rgb}{0.58,0,0.82}
\lstset{
  language=C++,
  aboveskip=3mm,
  belowskip=3mm,
  backgroundcolor=\color{codeBackground},
  showstringspaces=false,
  columns=flexible,
  basicstyle={\small\ttfamily},
  numbers=none,
  numberstyle=\tiny\color{gray},
  keywordstyle=\color{blue},
  commentstyle=\color{dkgreen},
  stringstyle=\color{mauve},
  breaklines=true,
  breakatwhitespace=true,
  tabsize=2
}

\date{\the\year-\the\month-\the\day}
\author{Kyle Chui}


\fancyhf{}
\lhead{Kyle Chui}
\rhead{Page \thepage}
\pagestyle{fancy}

\begin{document}
  \section{Lecture 6}
  The force between long parallel wires defines the ampere. As a reminder, last class we found the formula
  \[
    B = \frac{\mu_0I}{2\pi r}.
  \]
  Consider two parallel, infinitely-long wires that are carrying current. If the currents are going in the same direction, then we use the right-hand rule to find that the wires have an attractive force between them. We find that the force acting on wire $2$ is
  \begin{align*}
    \vec{F} &= I_2\vec{L}\times \vec{B}_1 \\
            &= I_2L \frac{\mu_0I_1}{2\pi r} \text{ to the left},
  \end{align*}
  so we have
  \[
    F_2 = \frac{\mu_0}{2\pi}\cdot \frac{I_1I_2L}{r}.
  \]
  Rearranging some more, we get
  \[
    \frac{F_2}{L} = \paren{2\cdot 10^{-7}} \frac{I_1I_2}{r} \text{ Newtons}.
  \]
  What happens when we ave $I_1 = I_2$? Well, we get that
  \[
    \frac{F}{L}= \paren{2\cdot 10^{-7}} \frac{I^2}{r},
  \]
  which we can rearrange to get
  \[
    I = \underbrace{\paren{r\cdot \frac{F}{L}\cdot \frac{1}{2\cdot 10^{-7}}}^{\frac{1}{2}}}_{\text{defines the Ampere}}.
  \]
  \subsection{Ampere's Law}
  Just like Coulomb's Law leads to Gauss' Law (which is an integral), the Biot-Savart Law leads to Ampere's Law (also an integral). Remember the Biot-Savart Law from before:
  \[
    \mathrm{d}\vec{B} = \frac{\mu_0}{4\pi} \frac{I\,\mathrm{d}\vec{\ell}\times \hat{r}}{r^2}.
  \]
  \begin{theorem}{Ampere's Law}
    The amount of magnetic field in a loop can give you the current, given by
    \[
      \oint_\text{loop}\vec{B}\cdot \mathrm{d}\vec{\ell} = \mu_0I_\text{enc}.
    \]
  \end{theorem}
  The analogue to the Gaussian surface is what we call an \emph{Amperium loop}. There is no radial component to $\vec{B}$ because there are no monopoles. Thus we have that $\vec{B}\parallel \mathrm{d}\vec{\ell}$, and $\vec{B}\cdot \mathrm{d}\vec{\ell} = B\,\mathrm{d}\ell$. Then
  \begin{align*}
    \oint B\dx\ell &= B\oint \mathrm{d}\ell \\
                   &= B\cdot 2\pi r \\
                   &= \mu_0I,
  \end{align*}
  so
  \[
    B = \frac{\mu_0I}{2\pi r}.
  \]
  The differential form of Ampere's Law is given by:
  \[
    \vec{\nabla}\times \vec{B} = \mu_0\vec{J},
  \]
  where $\vec{J}$ is the current density.
  \subsubsection{Ampere's Law Example---Long Wire, Thin but Finite Thickness}
  Consider a wire with radius $R$ that has uniform current density $J$. We define current density to be the current per unit area, in other words
  \[
    J = \frac{I}{\pi R^2}, \text{ a constant}.
  \]
  We have a few cases here:
  \begin{enumerate}[label=(\alph*)]
    \item Outside the wire, i.e. $r > R$.
    \begin{align*}
      \oint \vec{B}\cdot \mathrm{d}\vec{\ell} &= \mu_0I_\text{encl} \\
      2\pi rB &= \mu_0I \\
      \Aboxed{B &= \frac{\mu_0I}{2\pi r}.}
    \end{align*}
    \item Inside the wire, i.e. $r < R$.
    \begin{align*}
      \oint \vec{B}\cdot \mathrm{d}\vec{\ell} &= \mu_0I_\text{encl} \\
      2\pi rB &= \mu_0I\cdot \frac{\pi r^2}{\pi R^2} \\
      2\pi rB &= \mu_0I\cdot \frac{r^2}{R^2} \\
      \Aboxed{B &= \frac{\mu_0Ir}{2\pi R^2}.}
    \end{align*}
  \end{enumerate}
  Pictorially, we can draw a graph for the magnetic field as a function of the distance to the centre of the wire as follows:
  \begin{center}\resizebox{8cm}{!}{\import{./figures}{Lecture6D1.pdf_tex}}\end{center}
  \subsubsection{Ampere's Law Example---Plane of Current}
  If we have two parallel wires next to each other with the currents moving in the same direction, by superposition of field lines we generate a space with zero field between the wires. Extending this idea to more than just two wires (read: an infinite number, creating a plane of current), we create a planar magnetic field. 
  \begin{figure}[ht]
    \centering
    \incfig[0.7]{lecture6d2}
    \caption{Plane of Current}
  \end{figure}
  \begin{definition}{Winding Density}
    We define the \emph{winding density} $n$ to be the number of wires per unit length, or 
    \[
      n = \frac{N}{\ell}.
    \]
  \end{definition}
  Applying Ampere's Law, we have
  \begin{align*}
    \oint \vec{B}\cdot \mathrm{d}\vec{\ell} &= \mu_0I_\text{encl} \\
    \int_1 + \int_2 + \int_3 + \int_4 &= \mu_0NI\\
    2\int_4 \vec{B}\cdot \mathrm{d}\vec{\ell} &= \mu_0NI\\
    2B\ell &= \mu_0NI \\
    B &= \frac{\mu_0nI}{2}.
  \end{align*}
\end{document}
