\documentclass[class=article, crop=false]{standalone}
% Import packages
\usepackage[margin=1in]{geometry}

\usepackage[many]{tcolorbox}
\usepackage{amssymb, amsthm}
\usepackage{comment}
\usepackage{enumitem}
\usepackage{fancyhdr}
\usepackage{hyperref}
\usepackage{import}
\usepackage{listings}
\usepackage{mathrsfs, mathtools}
\usepackage{pdfpages}
\usepackage{standalone}
\usepackage{transparent}
\usepackage{xcolor}

\usetikzlibrary{decorations.pathreplacing}
\tcbuselibrary{skins}
% Declare math operators
\DeclareMathOperator{\lcm}{lcm}
\DeclareMathOperator{\proj}{proj}
\DeclareMathOperator{\vspan}{span}
\DeclareMathOperator{\im}{im}
\DeclareMathOperator{\range}{range}
\DeclareMathOperator{\Diff}{Diff}
\DeclareMathOperator{\Int}{Int}
\DeclareMathOperator{\fcn}{fcn}
\DeclareMathOperator{\id}{id}
\DeclareMathOperator{\rank}{rank}
\DeclareMathOperator{\tr}{tr}
\DeclareMathOperator{\dive}{div}
\DeclareMathOperator{\row}{row}
\DeclareMathOperator{\col}{col}
% Macros for letters/variables
\renewcommand{\tilde}{\raisebox{0.4ex}{\resizebox{2ex}{!}{\texttildelow}}}
\newcommand{\N}{\ensuremath{\mathbb{N}}}
\newcommand{\Z}{\ensuremath{\mathbb{Z}}}
\newcommand{\Q}{\ensuremath{\mathbb{Q}}}
\newcommand{\R}{\ensuremath{\mathbb{R}}}
\newcommand{\C}{\ensuremath{\mathbb{C}}}
\newcommand{\F}{\ensuremath{\mathbb{F}}}
\newcommand{\M}{\ensuremath{\mathbb{M}}}
\newcommand{\lam}{\ensuremath{\lambda}}
\newcommand{\nab}{\ensuremath{\nabla}}
\newcommand{\eps}{\ensuremath{\varepsilon}}
\newcommand{\es}{\ensuremath{\varnothing}}
% Macros for math symbols
\newcommand{\dx}[1]{\,\mathrm{d}#1}
\newcommand{\inv}{\ensuremath{^{-1}}}
\newcommand{\sm}{\setminus}
\newcommand{\sse}{\subseteq}
\newcommand{\ceq}{\coloneqq}
% Macros for pairs of math symbols
\newcommand{\abs}[1]{\ensuremath{\left\lvert #1 \right\rvert}}
\newcommand{\paren}[1]{\ensuremath{\left( #1 \right)}}
\newcommand{\norm}[1]{\ensuremath{\left\lVert #1\right\rVert}}
\newcommand{\set}[1]{\ensuremath{\left\{#1\right\}}}
\newcommand{\tup}[1]{\ensuremath{\left\langle #1 \right\rangle}}
\newcommand{\floor}[1]{\ensuremath{\left\lfloor #1 \right\rfloor}}
\newcommand{\ceil}[1]{\ensuremath{\left\lceil #1 \right\rceil}}
\newcommand{\eclass}[1]{\ensuremath{\left[ #1 \right]}}

\newcommand{\chapternum}{}
\newcommand{\ex}[1]{\noindent\textbf{Exercise \chapternum.{#1}.}}

\newcommand{\tsub}[1]{\textsubscript{#1}}
\newcommand{\tsup}[1]{\textsuperscript{#1}}

% Include figures
\newcommand{\incfig}[2][1]{%
    \def\svgwidth{#1\columnwidth}
    \import{./figures/}{#2.pdf_tex}
}

\definecolor{problemBackground}{RGB}{212,232,246}

\newenvironment{problem}[1]
  {
    \begin{tcolorbox}[
      boxrule=.5pt,
      titlerule=.5pt,
      sharp corners,
      colback=problemBackground,
      breakable
    ]
    \ifx &#1& \textbf{Problem. }
    \else \textbf{Problem #1.} \fi
  }
  {
    \end{tcolorbox}
  }
\definecolor{exampleBackground}{RGB}{255,249,248}
\definecolor{exampleAccent}{RGB}{158,60,14}
\newenvironment{example}[1]
  {
    \begin{tcolorbox}[
      boxrule=.5pt,
      sharp corners,
      colback=exampleBackground,
      colframe=exampleAccent,
    ]
    \color{exampleAccent}\textbf{Example.} \emph{#1}\color{black}
  }
  {
    \end{tcolorbox}
  }
\definecolor{theoremBackground}{RGB}{234,243,251}
\definecolor{theoremAccent}{RGB}{0,116,183}
\newenvironment{theorem}[1]
  {
    \begin{tcolorbox}[
      boxrule=.5pt,
      titlerule=.5pt,
      sharp corners,
      colback=theoremBackground,
      colframe=theoremAccent,
      breakable
    ]
      \color{theoremAccent}\textbf{Theorem --- }\emph{#1}\\\color{black}
  }
  {
    \end{tcolorbox}
  }
\definecolor{noteBackground}{RGB}{244,249,244}
\definecolor{noteAccent}{RGB}{34,139,34}
\newenvironment{note}[1]
  {
  \begin{tcolorbox}[
    enhanced,
    boxrule=0pt,
    frame hidden,
    sharp corners,
    colback=noteBackground,
    borderline west={3pt}{-1.5pt}{noteAccent},
    breakable
    ]
    \ifx &#1& \color{noteAccent}\textbf{Note. }\color{black}
    \else \color{noteAccent}\textbf{Note (#1). }\color{black} \fi
    }
    {
  \end{tcolorbox}
  }
\definecolor{lemmaBackground}{RGB}{255,247,234}
\definecolor{lemmaAccent}{RGB}{255,153,0}
\newenvironment{lemma}[1]
  {
  \begin{tcolorbox}[
    enhanced,
    boxrule=0pt,
    frame hidden,
    sharp corners,
    colback=lemmaBackground,
    borderline west={3pt}{-1.5pt}{lemmaAccent},
    breakable
    ]
    \ifx &#1& \color{lemmaAccent}\textbf{Lemma. }\color{black}
    \else \color{lemmaAccent}\textbf{Lemma #1. }\color{black} \fi
    }
    {
  \end{tcolorbox}
  }
\definecolor{definitionBackground}{RGB}{246,246,246}
\newenvironment{definition}[1]
  {
    \begin{tcolorbox}[
      enhanced,
      boxrule=0pt,
      frame hidden,
      sharp corners,
      colback=definitionBackground,
      borderline west={3pt}{-1.5pt}{black},
      breakable
    ]
    \textbf{Definition. }\emph{#1}\\
  }
  {
    \end{tcolorbox}
  }

\newenvironment{amatrix}[2]{
    \left[
      \begin{array}{*{#1}{c}|*{#2}c}
  }
  {
      \end{array}
    \right]
  }
\definecolor{codeBackground}{RGB}{253,246,225}
\definecolor{dkgreen}{rgb}{0,0.6,0}
\definecolor{gray}{rgb}{0.5,0.5,0.5}
\definecolor{mauve}{rgb}{0.58,0,0.82}
\lstset{
  language=C++,
  aboveskip=3mm,
  belowskip=3mm,
  backgroundcolor=\color{codeBackground},
  showstringspaces=false,
  columns=flexible,
  basicstyle={\small\ttfamily},
  numbers=none,
  numberstyle=\tiny\color{gray},
  keywordstyle=\color{blue},
  commentstyle=\color{dkgreen},
  stringstyle=\color{mauve},
  breaklines=true,
  breakatwhitespace=true,
  tabsize=2
}

\date{\the\year-\the\month-\the\day}
\author{Kyle Chui}


\fancyhf{}
\lhead{Kyle Chui}
\rhead{Page \thepage}
\pagestyle{fancy}

\begin{document}
  \section{Lecture 5}
  Last week, we studied the force due to a $\vec{B}$. This week, we will study how to produce a $\vec{B}$.
  \subsection{Biot-Savart Law}
  Point charge observations:
  \begin{itemize}
    \item When you have a point charge, you have an electric field $\vec{E}$ that radiates outwards from the point charge.
    \item This is slightly different from the magnetic field, which is generated by a \emph{moving} point charge.
    \begin{itemize}
      \item The closer you are to the point charge, the larger the magnetic field.
    \end{itemize}
  \end{itemize}
  \begin{theorem}{Coulomb's Law}
    Given the charge and distance from a point charge, we may find the electric field:
    \[
      \vec{E} = \frac{q\hat{r}}{4\pi\eps_0r^2}.
    \]
  \end{theorem}
  Biot-Savart Law precursor:
  \[
    \vec{B} = \frac{\mu_0}{4\pi}\cdot \frac{q\vec{v}\times\hat{r}}{r^2}.
  \]
  \begin{note}{}
    In the equation for the magnetic field, we see that it is both orthogonal to the velocity vector and $\hat{r}$ (due to the cross product).
  \end{note}
  Points:
  \begin{itemize}
    \item Both are inversely proportional to the square of the distance.
    \item $\mu_0 \ceq 4\pi\cdot 10^{-7}$ Tm/A, and is used in the definition of the ampere (and so the Coulomb).
    \begin{note}{}
      This means that the $\vec{E}$ is very large, but $\vec{B}$ is quite small.
    \end{note}
  \end{itemize}
  \begin{theorem}{Biot-Savart Law}
    Given the current through a wire, and the displacement from a point to the wire, we can find the magnetic field:
    \[
      \mathrm{d}\vec{B} = \frac{\mu_0}{4\pi} \frac{I\mathrm{d}\vec{\ell}\times\hat{r}}{r^2}.
    \]
  \end{theorem}
  \begin{figure}[ht]
    \centering
    \incfig[0.4]{lecture5d1}
  \end{figure}
  In the above diagram, you can see how the cross product of $\mathrm{d}\vec{\ell}$ and $\vec{r}$ is in the direction of the magnetic field. Furthermore, we can integrate the Biot-Savart Law to get
  \[
    \vec{B} = \int_\text{wire} \mathrm{d}\vec{B}
  \]
  \subsection{Magnetic Field of a Straight Current-Carrying Wire}
  For a long, straight current-carrying wire, we try to find the field on some point $P$ from locations $-a$ to $a$ on the wire.
  \begin{figure}[ht]
    \centering
    \incfig[1.2]{lecture5d2}
  \end{figure} \\
  From the diagram, we can see that $r = (x^2 + y^2)^{\frac{1}{2}}$, and
  \[
    \hat{r} - \frac{x\hat{\imath} - y\hat{\jmath}}{(x^2+y^2)^{\frac{1}{2}}}. \tag{Scale down $\vec{r}$}
  \]
  Furthermore, $\mathrm{d}\vec{\ell}= \mathrm{d}y\hat{\jmath}$. The hard part is calculating the following:
  \begin{align*}
    \frac{\mathrm{d}\vec{\ell}\times \hat{r}}{r^2} &= \frac{\mathrm{d}y\hat{\jmath}\times (x\hat{\imath}-y\hat{\jmath})}{(x^2 + y^2)^{\frac{3}{2}}} \\
                                                   &= \frac{-\hat{k}x\,\mathrm{d}y}{(x^2+y^2)^{\frac{3}{2}}}.
  \end{align*}
  We then integrate over the length of the wire:
  \begin{align*}
    \vec{B} &= -\hat{k} \frac{\mu_0I}{4\pi} \int_{-a}^{a} \frac{x\,\mathrm{d}y}{(x^2+y^2)^{\frac{3}{2}}}
    \intertext{Letting $y = x\tan\theta$, and going through some algebra, we get}
            &= -\hat{k} \frac{\mu_0I}{4\pi}\paren{\frac{2a}{x \sqrt{a^2+x^2}}}.
  \end{align*}
  For an infinitely long wire, we see what happens when $a\to \infty$. In this case, we have
  \[
    \vec{B} = -\hat{k} \frac{\mu_0I}{4\pi} \paren{\frac{2}{x}} = -\hat{k} \frac{\mu_0I}{2\pi x}.
  \]
  We started with a special point $P$ on the wire, which is on the perpendicular bisector of the wire. However, when $a\to\infty$, we have an infinite amount of wire on either side, so the $y$ becomes irrelevant and all we care about is the distance $x$ to the wire. Thus to find the magnitude of the wire, we have
  \[
    B = \frac{\mu_0I}{2\pi r},
  \]
  where $r$ is the distance to the wire. If we were to look at the wire from the top (the $xz$-plane), we would have that $\vec{B}$ circles the wire.
  \begin{note}{}
    Another right-hand rule is to put your thumb in the direction of the current, and your thumbs will curl to form the magnetic field.
  \end{note}
\end{document}
