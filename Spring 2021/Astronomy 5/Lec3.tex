\documentclass[class=article, crop=false]{standalone}
% Import packages
\usepackage[margin=1in]{geometry}

\usepackage[many]{tcolorbox}
\usepackage{amssymb, amsthm}
\usepackage{comment}
\usepackage{enumitem}
\usepackage{fancyhdr}
\usepackage{hyperref}
\usepackage{import}
\usepackage{listings}
\usepackage{mathrsfs, mathtools}
\usepackage{pdfpages}
\usepackage{standalone}
\usepackage{transparent}
\usepackage{xcolor}

\usetikzlibrary{decorations.pathreplacing}
\tcbuselibrary{skins}
% Declare math operators
\DeclareMathOperator{\lcm}{lcm}
\DeclareMathOperator{\proj}{proj}
\DeclareMathOperator{\vspan}{span}
\DeclareMathOperator{\im}{im}
\DeclareMathOperator{\range}{range}
\DeclareMathOperator{\Diff}{Diff}
\DeclareMathOperator{\Int}{Int}
\DeclareMathOperator{\fcn}{fcn}
\DeclareMathOperator{\id}{id}
\DeclareMathOperator{\rank}{rank}
\DeclareMathOperator{\tr}{tr}
\DeclareMathOperator{\dive}{div}
\DeclareMathOperator{\row}{row}
\DeclareMathOperator{\col}{col}
% Macros for letters/variables
\renewcommand{\tilde}{\raisebox{0.4ex}{\resizebox{2ex}{!}{\texttildelow}}}
\newcommand{\N}{\ensuremath{\mathbb{N}}}
\newcommand{\Z}{\ensuremath{\mathbb{Z}}}
\newcommand{\Q}{\ensuremath{\mathbb{Q}}}
\newcommand{\R}{\ensuremath{\mathbb{R}}}
\newcommand{\C}{\ensuremath{\mathbb{C}}}
\newcommand{\F}{\ensuremath{\mathbb{F}}}
\newcommand{\M}{\ensuremath{\mathbb{M}}}
\newcommand{\lam}{\ensuremath{\lambda}}
\newcommand{\nab}{\ensuremath{\nabla}}
\newcommand{\eps}{\ensuremath{\varepsilon}}
\newcommand{\es}{\ensuremath{\varnothing}}
% Macros for math symbols
\newcommand{\dx}[1]{\,\mathrm{d}#1}
\newcommand{\inv}{\ensuremath{^{-1}}}
\newcommand{\sm}{\setminus}
\newcommand{\sse}{\subseteq}
\newcommand{\ceq}{\coloneqq}
% Macros for pairs of math symbols
\newcommand{\abs}[1]{\ensuremath{\left\lvert #1 \right\rvert}}
\newcommand{\paren}[1]{\ensuremath{\left( #1 \right)}}
\newcommand{\norm}[1]{\ensuremath{\left\lVert #1\right\rVert}}
\newcommand{\set}[1]{\ensuremath{\left\{#1\right\}}}
\newcommand{\tup}[1]{\ensuremath{\left\langle #1 \right\rangle}}
\newcommand{\floor}[1]{\ensuremath{\left\lfloor #1 \right\rfloor}}
\newcommand{\ceil}[1]{\ensuremath{\left\lceil #1 \right\rceil}}
\newcommand{\eclass}[1]{\ensuremath{\left[ #1 \right]}}

\newcommand{\chapternum}{}
\newcommand{\ex}[1]{\noindent\textbf{Exercise \chapternum.{#1}.}}

\newcommand{\tsub}[1]{\textsubscript{#1}}
\newcommand{\tsup}[1]{\textsuperscript{#1}}

% Include figures
\newcommand{\incfig}[2][1]{%
    \def\svgwidth{#1\columnwidth}
    \import{./figures/}{#2.pdf_tex}
}

\definecolor{problemBackground}{RGB}{212,232,246}

\newenvironment{problem}[1]
  {
    \begin{tcolorbox}[
      boxrule=.5pt,
      titlerule=.5pt,
      sharp corners,
      colback=problemBackground,
      breakable
    ]
    \ifx &#1& \textbf{Problem. }
    \else \textbf{Problem #1.} \fi
  }
  {
    \end{tcolorbox}
  }
\definecolor{exampleBackground}{RGB}{255,249,248}
\definecolor{exampleAccent}{RGB}{158,60,14}
\newenvironment{example}[1]
  {
    \begin{tcolorbox}[
      boxrule=.5pt,
      sharp corners,
      colback=exampleBackground,
      colframe=exampleAccent,
    ]
    \color{exampleAccent}\textbf{Example.} \emph{#1}\color{black}
  }
  {
    \end{tcolorbox}
  }
\definecolor{theoremBackground}{RGB}{234,243,251}
\definecolor{theoremAccent}{RGB}{0,116,183}
\newenvironment{theorem}[1]
  {
    \begin{tcolorbox}[
      boxrule=.5pt,
      titlerule=.5pt,
      sharp corners,
      colback=theoremBackground,
      colframe=theoremAccent,
      breakable
    ]
      \color{theoremAccent}\textbf{Theorem --- }\emph{#1}\\\color{black}
  }
  {
    \end{tcolorbox}
  }
\definecolor{noteBackground}{RGB}{244,249,244}
\definecolor{noteAccent}{RGB}{34,139,34}
\newenvironment{note}[1]
  {
  \begin{tcolorbox}[
    enhanced,
    boxrule=0pt,
    frame hidden,
    sharp corners,
    colback=noteBackground,
    borderline west={3pt}{-1.5pt}{noteAccent},
    breakable
    ]
    \ifx &#1& \color{noteAccent}\textbf{Note. }\color{black}
    \else \color{noteAccent}\textbf{Note (#1). }\color{black} \fi
    }
    {
  \end{tcolorbox}
  }
\definecolor{lemmaBackground}{RGB}{255,247,234}
\definecolor{lemmaAccent}{RGB}{255,153,0}
\newenvironment{lemma}[1]
  {
  \begin{tcolorbox}[
    enhanced,
    boxrule=0pt,
    frame hidden,
    sharp corners,
    colback=lemmaBackground,
    borderline west={3pt}{-1.5pt}{lemmaAccent},
    breakable
    ]
    \ifx &#1& \color{lemmaAccent}\textbf{Lemma. }\color{black}
    \else \color{lemmaAccent}\textbf{Lemma #1. }\color{black} \fi
    }
    {
  \end{tcolorbox}
  }
\definecolor{definitionBackground}{RGB}{246,246,246}
\newenvironment{definition}[1]
  {
    \begin{tcolorbox}[
      enhanced,
      boxrule=0pt,
      frame hidden,
      sharp corners,
      colback=definitionBackground,
      borderline west={3pt}{-1.5pt}{black},
      breakable
    ]
    \textbf{Definition. }\emph{#1}\\
  }
  {
    \end{tcolorbox}
  }

\newenvironment{amatrix}[2]{
    \left[
      \begin{array}{*{#1}{c}|*{#2}c}
  }
  {
      \end{array}
    \right]
  }
\definecolor{codeBackground}{RGB}{253,246,225}
\definecolor{dkgreen}{rgb}{0,0.6,0}
\definecolor{gray}{rgb}{0.5,0.5,0.5}
\definecolor{mauve}{rgb}{0.58,0,0.82}
\lstset{
  language=C++,
  aboveskip=3mm,
  belowskip=3mm,
  backgroundcolor=\color{codeBackground},
  showstringspaces=false,
  columns=flexible,
  basicstyle={\small\ttfamily},
  numbers=none,
  numberstyle=\tiny\color{gray},
  keywordstyle=\color{blue},
  commentstyle=\color{dkgreen},
  stringstyle=\color{mauve},
  breaklines=true,
  breakatwhitespace=true,
  tabsize=2
}

\date{\the\year-\the\month-\the\day}
\author{Kyle Chui}


\fancyhf{}
\lhead{Kyle Chui}
\rhead{Page \thepage}
\pagestyle{fancy}

\begin{document}
  \section{Lecture 3}
  The Earth is non-stationary---it is moving through space at extreme speeds. The universe is also expanding, and things are all getting further and further away from us. The reason that we aren't getting further and further away from the Earth is because gravity is usually enough to hold things together (when things are close enough).
  \begin{note}{}
    We are not in a particularly special place in the universe, nor are we at a special time.
  \end{note}
  \subsection{Planetary Science}
  \begin{definition}{Planetary Science}
    \emph{Planetary science} is the study of the creation and evolution of planetary bodies, moons, asteroids, comets, etc.
  \end{definition}
  Studying solar system bodies investigates why life formed on some worlds, and not others.
  \begin{itemize}
    \item All the planets orbit the Sun in elliptic paths, all in the same plane. 
    \item The tilt of the planet is the main reason for the seasons.
  \end{itemize}
  \subsubsection{Annual Motion Definitions}
  \begin{definition}{Ecliptic}
    The \emph{ecliptic} is the apparent path of the Sun through the sky.
  \end{definition}
  \begin{definition}{Equinox}
    The \emph{equinox} is where the ecliptic intersects the celestial equator.
  \end{definition}
  \begin{definition}{Solstice}
    The \emph{solstice} is where the ecliptic is farthest from the celestial equator.
  \end{definition}
  \begin{definition}{Zodiac}
    The \emph{zodiac} is the constellations which lie along the ecliptic.
  \end{definition}
  \subsection{The Parsec}
  \begin{definition}{Parsec}
    We define one \emph{parsec} to be 3.26 light-years.
  \end{definition}
  We can calculate the distance to a star by using the parallax effect, namely
  \[
    \text{Distance in parsecs} = \frac{1}{\text{Parallax in seconds}}.
  \]
  \subsection{The Science of Astronomy}
  Copernicus proposed the heliocentric model in $1543$, but his model was no more accurate than geocentric models, because he used perfect circles for the orbits. Tycho Brahe tried, but failed, to detect stellar parallax, so he thought Earth was at the centre of the solar system and other planets went around the Sun. He would go on to hire Johannes Kepler, who used Tycho's observations to discover the \emph{truth} of planetary motion. Johannes Kepler first tried to use circular orbits, but a discrepancy led him to propose elliptical orbits.
  \subsection{Kepler's Laws}
  \begin{enumerate}
    \item The orbit of each planet around the Sun is an ellipse.
    \item As a planet moves around its orbit, it sweeps out \emph{equal areas in equal times}.
    \begin{note}{}
      Planets move \emph{faster} the closer they are to the Sun.
    \end{note}
    \item More distant planets orbit the Sun at slower \emph{average} speeds, obeying the relationship
    \[
      p^2 = a^3,
    \]
    where $p$ is the orbital period (in years) and $a$ is the average distance from the Sun (in AU).
  \end{enumerate}
  \subsection{Galileo Discoveries}
  \begin{itemize}
    \item Galileo showed that objects will stay in motion unless a force acts on them to slow them down.
    \item Galileo proved that there are imperfections on the celestial bodies: sunspots on the Sun, craters on the moon, etc.
    \item Galileo proved that stars are much further than Tycho thought, so the undetectable parallax was justified.
    \item There are objects that do not orbit the Earth (the moons of Jupiter).
    \item By observing the phases of Venus, he showed that Venus does not orbit the Earth.
  \end{itemize}
  \subsection{Hallmarks of Science}
  \begin{enumerate}
    \item Modern science seeks explanations for observed phenomena that rely solely on \emph{natural} causes.
    \item Science progresses through the creation and testing of models of nature that explain the observations as simply as possible.
    \item A scientific model must make \emph{testable predictions} about natural phenomena that would force us to revise/abandon the model if it does not agree with observations.
  \end{enumerate}
  \begin{definition}{Scientific Theory}
    A \emph{scientific theory} must:
    \begin{itemize}
      \item Explain a wide variety of observations with a few simple principles.
      \item Must be supported by a large, compelling body of evidence.
      \item Must not have failed any crucial test of its validity.
    \end{itemize}
  \end{definition}
\end{document}
