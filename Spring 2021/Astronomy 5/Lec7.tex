\documentclass[class=article, crop=false]{standalone}
% Import packages
\usepackage[margin=1in]{geometry}

\usepackage[many]{tcolorbox}
\usepackage{amssymb, amsthm}
\usepackage{comment}
\usepackage{enumitem}
\usepackage{fancyhdr}
\usepackage{hyperref}
\usepackage{import}
\usepackage{listings}
\usepackage{mathrsfs, mathtools}
\usepackage{pdfpages}
\usepackage{standalone}
\usepackage{transparent}
\usepackage{xcolor}

\usetikzlibrary{decorations.pathreplacing}
\tcbuselibrary{skins}
% Declare math operators
\DeclareMathOperator{\lcm}{lcm}
\DeclareMathOperator{\proj}{proj}
\DeclareMathOperator{\vspan}{span}
\DeclareMathOperator{\im}{im}
\DeclareMathOperator{\range}{range}
\DeclareMathOperator{\Diff}{Diff}
\DeclareMathOperator{\Int}{Int}
\DeclareMathOperator{\fcn}{fcn}
\DeclareMathOperator{\id}{id}
\DeclareMathOperator{\rank}{rank}
\DeclareMathOperator{\tr}{tr}
\DeclareMathOperator{\dive}{div}
\DeclareMathOperator{\row}{row}
\DeclareMathOperator{\col}{col}
% Macros for letters/variables
\renewcommand{\tilde}{\raisebox{0.4ex}{\resizebox{2ex}{!}{\texttildelow}}}
\newcommand{\N}{\ensuremath{\mathbb{N}}}
\newcommand{\Z}{\ensuremath{\mathbb{Z}}}
\newcommand{\Q}{\ensuremath{\mathbb{Q}}}
\newcommand{\R}{\ensuremath{\mathbb{R}}}
\newcommand{\C}{\ensuremath{\mathbb{C}}}
\newcommand{\F}{\ensuremath{\mathbb{F}}}
\newcommand{\M}{\ensuremath{\mathbb{M}}}
\newcommand{\lam}{\ensuremath{\lambda}}
\newcommand{\nab}{\ensuremath{\nabla}}
\newcommand{\eps}{\ensuremath{\varepsilon}}
\newcommand{\es}{\ensuremath{\varnothing}}
% Macros for math symbols
\newcommand{\dx}[1]{\,\mathrm{d}#1}
\newcommand{\inv}{\ensuremath{^{-1}}}
\newcommand{\sm}{\setminus}
\newcommand{\sse}{\subseteq}
\newcommand{\ceq}{\coloneqq}
% Macros for pairs of math symbols
\newcommand{\abs}[1]{\ensuremath{\left\lvert #1 \right\rvert}}
\newcommand{\paren}[1]{\ensuremath{\left( #1 \right)}}
\newcommand{\norm}[1]{\ensuremath{\left\lVert #1\right\rVert}}
\newcommand{\set}[1]{\ensuremath{\left\{#1\right\}}}
\newcommand{\tup}[1]{\ensuremath{\left\langle #1 \right\rangle}}
\newcommand{\floor}[1]{\ensuremath{\left\lfloor #1 \right\rfloor}}
\newcommand{\ceil}[1]{\ensuremath{\left\lceil #1 \right\rceil}}
\newcommand{\eclass}[1]{\ensuremath{\left[ #1 \right]}}

\newcommand{\chapternum}{}
\newcommand{\ex}[1]{\noindent\textbf{Exercise \chapternum.{#1}.}}

\newcommand{\tsub}[1]{\textsubscript{#1}}
\newcommand{\tsup}[1]{\textsuperscript{#1}}

% Include figures
\newcommand{\incfig}[2][1]{%
    \def\svgwidth{#1\columnwidth}
    \import{./figures/}{#2.pdf_tex}
}

\definecolor{problemBackground}{RGB}{212,232,246}

\newenvironment{problem}[1]
  {
    \begin{tcolorbox}[
      boxrule=.5pt,
      titlerule=.5pt,
      sharp corners,
      colback=problemBackground,
      breakable
    ]
    \ifx &#1& \textbf{Problem. }
    \else \textbf{Problem #1.} \fi
  }
  {
    \end{tcolorbox}
  }
\definecolor{exampleBackground}{RGB}{255,249,248}
\definecolor{exampleAccent}{RGB}{158,60,14}
\newenvironment{example}[1]
  {
    \begin{tcolorbox}[
      boxrule=.5pt,
      sharp corners,
      colback=exampleBackground,
      colframe=exampleAccent,
    ]
    \color{exampleAccent}\textbf{Example.} \emph{#1}\color{black}
  }
  {
    \end{tcolorbox}
  }
\definecolor{theoremBackground}{RGB}{234,243,251}
\definecolor{theoremAccent}{RGB}{0,116,183}
\newenvironment{theorem}[1]
  {
    \begin{tcolorbox}[
      boxrule=.5pt,
      titlerule=.5pt,
      sharp corners,
      colback=theoremBackground,
      colframe=theoremAccent,
      breakable
    ]
      \color{theoremAccent}\textbf{Theorem --- }\emph{#1}\\\color{black}
  }
  {
    \end{tcolorbox}
  }
\definecolor{noteBackground}{RGB}{244,249,244}
\definecolor{noteAccent}{RGB}{34,139,34}
\newenvironment{note}[1]
  {
  \begin{tcolorbox}[
    enhanced,
    boxrule=0pt,
    frame hidden,
    sharp corners,
    colback=noteBackground,
    borderline west={3pt}{-1.5pt}{noteAccent},
    breakable
    ]
    \ifx &#1& \color{noteAccent}\textbf{Note. }\color{black}
    \else \color{noteAccent}\textbf{Note (#1). }\color{black} \fi
    }
    {
  \end{tcolorbox}
  }
\definecolor{lemmaBackground}{RGB}{255,247,234}
\definecolor{lemmaAccent}{RGB}{255,153,0}
\newenvironment{lemma}[1]
  {
  \begin{tcolorbox}[
    enhanced,
    boxrule=0pt,
    frame hidden,
    sharp corners,
    colback=lemmaBackground,
    borderline west={3pt}{-1.5pt}{lemmaAccent},
    breakable
    ]
    \ifx &#1& \color{lemmaAccent}\textbf{Lemma. }\color{black}
    \else \color{lemmaAccent}\textbf{Lemma #1. }\color{black} \fi
    }
    {
  \end{tcolorbox}
  }
\definecolor{definitionBackground}{RGB}{246,246,246}
\newenvironment{definition}[1]
  {
    \begin{tcolorbox}[
      enhanced,
      boxrule=0pt,
      frame hidden,
      sharp corners,
      colback=definitionBackground,
      borderline west={3pt}{-1.5pt}{black},
      breakable
    ]
    \textbf{Definition. }\emph{#1}\\
  }
  {
    \end{tcolorbox}
  }

\newenvironment{amatrix}[2]{
    \left[
      \begin{array}{*{#1}{c}|*{#2}c}
  }
  {
      \end{array}
    \right]
  }
\definecolor{codeBackground}{RGB}{253,246,225}
\definecolor{dkgreen}{rgb}{0,0.6,0}
\definecolor{gray}{rgb}{0.5,0.5,0.5}
\definecolor{mauve}{rgb}{0.58,0,0.82}
\lstset{
  language=C++,
  aboveskip=3mm,
  belowskip=3mm,
  backgroundcolor=\color{codeBackground},
  showstringspaces=false,
  columns=flexible,
  basicstyle={\small\ttfamily},
  numbers=none,
  numberstyle=\tiny\color{gray},
  keywordstyle=\color{blue},
  commentstyle=\color{dkgreen},
  stringstyle=\color{mauve},
  breaklines=true,
  breakatwhitespace=true,
  tabsize=2
}

\date{\the\year-\the\month-\the\day}
\author{Kyle Chui}


\fancyhf{}
\lhead{Kyle Chui}
\rhead{Page \thepage}
\pagestyle{fancy}

\begin{document}
  \section{Lecture 7}
  \subsection{Geology}
  \subsubsection{Earth and the Terrestrial Worlds}
  \begin{itemize}
    \item Mercury has craters, smooth plains, and cliffs.
    \item Venus has volcanoes and a few craters. It is also extremely hot with a thick atmosphere.
    \item Mars has some craters, volcanoes, and potentially riverbeds. It is also freezing with a very thin atmosphere.
    \item The Earth has volcanoes, craters, mountains, and riverbeds. Earth is ``just right''.
  \end{itemize}
  \begin{note}{}
    Mercury and Mars have much smaller mass than the Earth, while Venus is much more similar.
  \end{note}
  \begin{note}{}
    The mass of a planet determines its long term ability to hold heat, and by extension whether it retains a heavy atmosphere. The mass of a planet and distance to the Sun are key.
  \end{note}
  \subsubsection{The Evolution of the Solar System}
  \begin{itemize}
    \item The Sun is powered by the fusion of Hydrogen into Helium.
    \item This process is stable over billions of years. However, as Helium builds up in the core of the Sun, it shrinks and heats---higher temperature increases nuclear fusion, making the sun brighter.
    \item The Earth's atmosphere has only become breathable to humans for a few hundred million years.
    \item Mars and Venus were once ``habitable'' worlds with thick atmospheres and flowing water on their surfaces.
    \begin{itemize}
      \item Mars has cooled off, lacking a magnetic field and thick atmosphere.
      \item Venus likely had 2-3 billion years of ``comfortable'' Earth-like conditions before entering the hellish, moist greenhouse phase.
      \item Earth might have the same fate as Venus as little as $500$ million years from now, due to the Sun's rising brightness from consumption of Hydrogen.
    \end{itemize}
  \end{itemize}
  \subsection{Earth as a Planet}
  \subsubsection{Why is Earth Geologically Active?}
  The ``lithosphere'' is the cool rigid rock that forms a planet's outer layer: the crust and some of the mantle. From inner to outer, the layers of the Earth are:
  \begin{itemize}
    \item Core: Highest density, made of nickel and iron.
    \item Mantle: Moderate density, made of minerals with silicon, oxygen, etc.
    \item Crust: Lowest density, made of granite, basalt, etc.
  \end{itemize}
  \begin{note}{}
    If the lithosphere is near the surface then the planet will be geologically active.
  \end{note}
  The internal heat of our planet comes from:
  \begin{itemize}
    \item Gravitational Potential Energy of accreting planetesimals colliding turns into heat.
    \item Differentiation (sinking of heavier elements adds heat).
    \item Radioactivity (emitted particles from radioactive atoms carry kinetic energy, which turns into heat upon collision).
  \end{itemize}
  Heat drives geological activity via convection currents (hot material rises, cool material falls). On Earth, one convection cycle takes $100$ million years. Larger objects cool more slowly---Earth and Venus are active, whereas Mars and Mercury are not.
  \subsubsection{Geology and Life}
  There are three clues that we have for why Earth is habitable:
  \begin{enumerate}
    \item Volcanism---Releases trapped heat, gas, and material from deep within Earth.
    \item Plate Tectonics---Plate tectonics continue to reshape Earth, consequentially regulating its climate.
    \item Global Magnetic Field---Earth's magnetic field acts as a barrier that mitigates the loss of the atmosphere due to solar wind stripping.
  \end{enumerate}
  \subsubsection{What Processes Shape Earth's Surface?}
  Impact cratering shapes the surface of all terrestrial objects, but the Earth is protected by its atmosphere. Furthermore, the craters on the Earth are erased by erosion, volcanic activity, and plate tectonics. \par
  Volcanoes erupt when there is a pressure build-up of gases inside it. Molten rock on the surface is called lava, and erases other geological features. Provided some of the water for our oceans. \par
  \begin{definition}{Tectonics}
    Any surface reshaping from forces acting on the lithosphere.
  \end{definition}
  \begin{definition}{Plate Tectonics}
    Pieces of the lithosphere moving around. Collisions of plates cause mountains to be built. Sideways motion of plates cause earthquakes.
  \end{definition}
  \begin{note}{}
    Only Earth has plate tectonics.
  \end{note}
  Erosion is any weather-driven process that breaks down or transports rocks. \par
  Small bacteria have caused our atmosphere to become rich with oxygen. Coal and oil has been deposited in the Earth from biological material. Humans have also had an enormous impact on the surface and climate. \par
  To find the relative ages of fossils in sedimentary rock, we look at the layers (deeper means older). We can also get the absolute ages via radiometric dating.
  \subsubsection{How Does Earth's Atmosphere Affect the Planet?}
  \begin{definition}{Greenhouse Gas}
    Any gas that absorbs infrared light is a \emph{greenhouse gas}.
  \end{definition}
  \begin{itemize}
    \item Molecules with two different elements---Carbon dioxide, water vapour, and methane are all examples of greenhouse gases.
    \item Diatomic elements or lone atoms are not greenhouse gases---O\tsub{2}, N\tsub{2}
    \item The Earth benefits from this effect to a certain extent. Too much and we turn into Venus.
  \end{itemize}
  Human activity has drastically increased the amount of greenhouse gases in the atmosphere, which has led to more global warming.
\end{document}
