\documentclass[class=article, crop=false]{standalone}
% Import packages
\usepackage[margin=1in]{geometry}

\usepackage[many]{tcolorbox}
\usepackage{amssymb, amsthm}
\usepackage{comment}
\usepackage{enumitem}
\usepackage{fancyhdr}
\usepackage{hyperref}
\usepackage{import}
\usepackage{listings}
\usepackage{mathrsfs, mathtools}
\usepackage{pdfpages}
\usepackage{standalone}
\usepackage{transparent}
\usepackage{xcolor}

\usetikzlibrary{decorations.pathreplacing}
\tcbuselibrary{skins}
% Declare math operators
\DeclareMathOperator{\lcm}{lcm}
\DeclareMathOperator{\proj}{proj}
\DeclareMathOperator{\vspan}{span}
\DeclareMathOperator{\im}{im}
\DeclareMathOperator{\range}{range}
\DeclareMathOperator{\Diff}{Diff}
\DeclareMathOperator{\Int}{Int}
\DeclareMathOperator{\fcn}{fcn}
\DeclareMathOperator{\id}{id}
\DeclareMathOperator{\rank}{rank}
\DeclareMathOperator{\tr}{tr}
\DeclareMathOperator{\dive}{div}
\DeclareMathOperator{\row}{row}
\DeclareMathOperator{\col}{col}
% Macros for letters/variables
\renewcommand{\tilde}{\raisebox{0.4ex}{\resizebox{2ex}{!}{\texttildelow}}}
\newcommand{\N}{\ensuremath{\mathbb{N}}}
\newcommand{\Z}{\ensuremath{\mathbb{Z}}}
\newcommand{\Q}{\ensuremath{\mathbb{Q}}}
\newcommand{\R}{\ensuremath{\mathbb{R}}}
\newcommand{\C}{\ensuremath{\mathbb{C}}}
\newcommand{\F}{\ensuremath{\mathbb{F}}}
\newcommand{\M}{\ensuremath{\mathbb{M}}}
\newcommand{\lam}{\ensuremath{\lambda}}
\newcommand{\nab}{\ensuremath{\nabla}}
\newcommand{\eps}{\ensuremath{\varepsilon}}
\newcommand{\es}{\ensuremath{\varnothing}}
% Macros for math symbols
\newcommand{\dx}[1]{\,\mathrm{d}#1}
\newcommand{\inv}{\ensuremath{^{-1}}}
\newcommand{\sm}{\setminus}
\newcommand{\sse}{\subseteq}
\newcommand{\ceq}{\coloneqq}
% Macros for pairs of math symbols
\newcommand{\abs}[1]{\ensuremath{\left\lvert #1 \right\rvert}}
\newcommand{\paren}[1]{\ensuremath{\left( #1 \right)}}
\newcommand{\norm}[1]{\ensuremath{\left\lVert #1\right\rVert}}
\newcommand{\set}[1]{\ensuremath{\left\{#1\right\}}}
\newcommand{\tup}[1]{\ensuremath{\left\langle #1 \right\rangle}}
\newcommand{\floor}[1]{\ensuremath{\left\lfloor #1 \right\rfloor}}
\newcommand{\ceil}[1]{\ensuremath{\left\lceil #1 \right\rceil}}
\newcommand{\eclass}[1]{\ensuremath{\left[ #1 \right]}}

\newcommand{\chapternum}{}
\newcommand{\ex}[1]{\noindent\textbf{Exercise \chapternum.{#1}.}}

\newcommand{\tsub}[1]{\textsubscript{#1}}
\newcommand{\tsup}[1]{\textsuperscript{#1}}

% Include figures
\newcommand{\incfig}[2][1]{%
    \def\svgwidth{#1\columnwidth}
    \import{./figures/}{#2.pdf_tex}
}

\definecolor{problemBackground}{RGB}{212,232,246}

\newenvironment{problem}[1]
  {
    \begin{tcolorbox}[
      boxrule=.5pt,
      titlerule=.5pt,
      sharp corners,
      colback=problemBackground,
      breakable
    ]
    \ifx &#1& \textbf{Problem. }
    \else \textbf{Problem #1.} \fi
  }
  {
    \end{tcolorbox}
  }
\definecolor{exampleBackground}{RGB}{255,249,248}
\definecolor{exampleAccent}{RGB}{158,60,14}
\newenvironment{example}[1]
  {
    \begin{tcolorbox}[
      boxrule=.5pt,
      sharp corners,
      colback=exampleBackground,
      colframe=exampleAccent,
    ]
    \color{exampleAccent}\textbf{Example.} \emph{#1}\color{black}
  }
  {
    \end{tcolorbox}
  }
\definecolor{theoremBackground}{RGB}{234,243,251}
\definecolor{theoremAccent}{RGB}{0,116,183}
\newenvironment{theorem}[1]
  {
    \begin{tcolorbox}[
      boxrule=.5pt,
      titlerule=.5pt,
      sharp corners,
      colback=theoremBackground,
      colframe=theoremAccent,
      breakable
    ]
      \color{theoremAccent}\textbf{Theorem --- }\emph{#1}\\\color{black}
  }
  {
    \end{tcolorbox}
  }
\definecolor{noteBackground}{RGB}{244,249,244}
\definecolor{noteAccent}{RGB}{34,139,34}
\newenvironment{note}[1]
  {
  \begin{tcolorbox}[
    enhanced,
    boxrule=0pt,
    frame hidden,
    sharp corners,
    colback=noteBackground,
    borderline west={3pt}{-1.5pt}{noteAccent},
    breakable
    ]
    \ifx &#1& \color{noteAccent}\textbf{Note. }\color{black}
    \else \color{noteAccent}\textbf{Note (#1). }\color{black} \fi
    }
    {
  \end{tcolorbox}
  }
\definecolor{lemmaBackground}{RGB}{255,247,234}
\definecolor{lemmaAccent}{RGB}{255,153,0}
\newenvironment{lemma}[1]
  {
  \begin{tcolorbox}[
    enhanced,
    boxrule=0pt,
    frame hidden,
    sharp corners,
    colback=lemmaBackground,
    borderline west={3pt}{-1.5pt}{lemmaAccent},
    breakable
    ]
    \ifx &#1& \color{lemmaAccent}\textbf{Lemma. }\color{black}
    \else \color{lemmaAccent}\textbf{Lemma #1. }\color{black} \fi
    }
    {
  \end{tcolorbox}
  }
\definecolor{definitionBackground}{RGB}{246,246,246}
\newenvironment{definition}[1]
  {
    \begin{tcolorbox}[
      enhanced,
      boxrule=0pt,
      frame hidden,
      sharp corners,
      colback=definitionBackground,
      borderline west={3pt}{-1.5pt}{black},
      breakable
    ]
    \textbf{Definition. }\emph{#1}\\
  }
  {
    \end{tcolorbox}
  }

\newenvironment{amatrix}[2]{
    \left[
      \begin{array}{*{#1}{c}|*{#2}c}
  }
  {
      \end{array}
    \right]
  }
\definecolor{codeBackground}{RGB}{253,246,225}
\definecolor{dkgreen}{rgb}{0,0.6,0}
\definecolor{gray}{rgb}{0.5,0.5,0.5}
\definecolor{mauve}{rgb}{0.58,0,0.82}
\lstset{
  language=C++,
  aboveskip=3mm,
  belowskip=3mm,
  backgroundcolor=\color{codeBackground},
  showstringspaces=false,
  columns=flexible,
  basicstyle={\small\ttfamily},
  numbers=none,
  numberstyle=\tiny\color{gray},
  keywordstyle=\color{blue},
  commentstyle=\color{dkgreen},
  stringstyle=\color{mauve},
  breaklines=true,
  breakatwhitespace=true,
  tabsize=2
}

\date{\the\year-\the\month-\the\day}
\author{Kyle Chui}


\fancyhf{}
\lhead{Kyle Chui}
\rhead{Page \thepage}
\pagestyle{fancy}

\begin{document}
  \section{Lecture 9}
  \subsection{History of Life on Earth}
  \subsubsection{Major Groupings of Life}
  \begin{itemize}
    \item Old idea: Major groupings of life fall to the kingdoms---animals, plants, protista, monera, and fungi.
    \item Modern theory: Superkingdoms or domains: bacteria, archaea, and eukarya. Bacteria and archaea consist almost entirely of microbes, whereas eukarya has multicellular life but also contains microbes.
  \end{itemize}
  \subsubsection{Metabolism}
  \begin{itemize}
    \item All cells use adenosine triphosphate (ATP) as the basic energy currency inside the cell.
    \item Humans and many life forms consume food for energy and carbon, but not all.
    \begin{itemize}
      \item Heterotrophs get energy from their food.
      \item Autotrophs get energy from their environment.
    \end{itemize}
  \end{itemize}
  \subsubsection{Liquid Water in Metabolism}
  \begin{enumerate}
    \item Metabolism requires organic chemicals to be available for reactions. Water allows these chemicals to float within the cell.
    \item Water transports chemicals to cells, and waste away from cells.
    \item Water is necessary for reactions that store and release energy in ATP.
  \end{enumerate}
  \subsubsection{The Theory of Evolution}
  \begin{itemize}
    \item The fossil record shows that evolution of species has occurred through time.
    \item Darwin's theory tells us \emph{how} evolution occurs: through natural selection (via mutations).
    \item The fact that all DNA is made of more or less the same stuff suggests a common ancestry.
  \end{itemize}
  \subsubsection{Life at the Extreme---Extremophiles}
  \begin{itemize}
    \item Thermophiles live near volcanic vents, usually above the boiling point of water.
    \item Psychrophiles, or lovers of cold, usually live in Antarctica.
    \item Endoliths (within rocks) live several kilometres below the surface.
  \end{itemize}
  \begin{definition}{Convergent Evolution}
    Under comparable evolutionary pressures, evolution will take a comparable path.
  \end{definition}
  \subsection{Unique Features of Earth that are Important for Life}
  \begin{itemize}
    \item Surface liquid water
    \item Plate tectonics
    \item Climate stability (greenhouse effect and plate tectonics)
    \item Atmospheric oxygen
  \end{itemize}
  \subsubsection{Carbon Cycle}
  \begin{enumerate}
    \item Atmospheric CO\tsub{2} dissolves in rainwater
    \item Rain erodes minerals that flow into the ocean
    \item Minerals combine with carbon to make rocks on the ocean floor
    \item Subduction carries these carbonate rocks into the mantle
    \item Rock melts in the mantle and releases CO\tsub{2} back into the atmosphere via volcanoes
  \end{enumerate}
  \subsubsection{Role of the Atmosphere}
  \begin{itemize}
    \item Greenhouse effect stabilises the temperature
    \item Protects against radiation
    \item Provides water and life sustaining gases
  \end{itemize}
  \subsubsection{When Did Life Arise on Earth?}
  \begin{itemize}
    \item Life arose about $3$.85 billion years ago, shortly after the period of heavy bombardment, which was about $4$.2--3.9 billion years ago.
    \item Evidence from fossils and carbon isotopes.
  \end{itemize}
  \begin{definition}{Stromatolites}
    \emph{Stromatolites} are ancient rocks that show structure similar to bacterial colonies today.
  \end{definition}
  \subsubsection{Origin of Life on Earth}
  \begin{itemize}
    \item There is clear evidence that life \emph{changes and evolves through time}
    \item There is clear evidence that life \emph{shares a common ancestry}
    \item We may never know exactly how the first organism arose, but modern experiments suggest plausible scenarios
  \end{itemize}
  \subsubsection{Brief Timeline}
  \begin{itemize}
    \item $4$.4 billion years ago---Early oceans form
    \item $3$.5 billion years ago---Cyanobacteria start releasing oxygen
    \item $2$.4 billion years ago---Great oxidation event (nearly all iron ore on Earth is formed at once)
    \item $2$.0 billion years ago---Oxygen begins building up in the atmosphere
    \item 540--550 million years ago---Cambrian explosion
    \item 225--65 million years ago---Dinosaurs and small mammals (dinosaurs ruled)
    \item Few million years ago---Earliest hominids
  \end{itemize}
\end{document}
