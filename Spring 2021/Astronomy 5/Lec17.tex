\documentclass[class=article, crop=false]{standalone}
% Import packages
\usepackage[margin=1in]{geometry}

\usepackage[many]{tcolorbox}
\usepackage{amssymb, amsthm}
\usepackage{comment}
\usepackage{enumitem}
\usepackage{fancyhdr}
\usepackage{hyperref}
\usepackage{import}
\usepackage{listings}
\usepackage{mathrsfs, mathtools}
\usepackage{pdfpages}
\usepackage{standalone}
\usepackage{transparent}
\usepackage{xcolor}

\usetikzlibrary{decorations.pathreplacing}
\tcbuselibrary{skins}
% Declare math operators
\DeclareMathOperator{\lcm}{lcm}
\DeclareMathOperator{\proj}{proj}
\DeclareMathOperator{\vspan}{span}
\DeclareMathOperator{\im}{im}
\DeclareMathOperator{\range}{range}
\DeclareMathOperator{\Diff}{Diff}
\DeclareMathOperator{\Int}{Int}
\DeclareMathOperator{\fcn}{fcn}
\DeclareMathOperator{\id}{id}
\DeclareMathOperator{\rank}{rank}
\DeclareMathOperator{\tr}{tr}
\DeclareMathOperator{\dive}{div}
\DeclareMathOperator{\row}{row}
\DeclareMathOperator{\col}{col}
% Macros for letters/variables
\renewcommand{\tilde}{\raisebox{0.4ex}{\resizebox{2ex}{!}{\texttildelow}}}
\newcommand{\N}{\ensuremath{\mathbb{N}}}
\newcommand{\Z}{\ensuremath{\mathbb{Z}}}
\newcommand{\Q}{\ensuremath{\mathbb{Q}}}
\newcommand{\R}{\ensuremath{\mathbb{R}}}
\newcommand{\C}{\ensuremath{\mathbb{C}}}
\newcommand{\F}{\ensuremath{\mathbb{F}}}
\newcommand{\M}{\ensuremath{\mathbb{M}}}
\newcommand{\lam}{\ensuremath{\lambda}}
\newcommand{\nab}{\ensuremath{\nabla}}
\newcommand{\eps}{\ensuremath{\varepsilon}}
\newcommand{\es}{\ensuremath{\varnothing}}
% Macros for math symbols
\newcommand{\dx}[1]{\,\mathrm{d}#1}
\newcommand{\inv}{\ensuremath{^{-1}}}
\newcommand{\sm}{\setminus}
\newcommand{\sse}{\subseteq}
\newcommand{\ceq}{\coloneqq}
% Macros for pairs of math symbols
\newcommand{\abs}[1]{\ensuremath{\left\lvert #1 \right\rvert}}
\newcommand{\paren}[1]{\ensuremath{\left( #1 \right)}}
\newcommand{\norm}[1]{\ensuremath{\left\lVert #1\right\rVert}}
\newcommand{\set}[1]{\ensuremath{\left\{#1\right\}}}
\newcommand{\tup}[1]{\ensuremath{\left\langle #1 \right\rangle}}
\newcommand{\floor}[1]{\ensuremath{\left\lfloor #1 \right\rfloor}}
\newcommand{\ceil}[1]{\ensuremath{\left\lceil #1 \right\rceil}}
\newcommand{\eclass}[1]{\ensuremath{\left[ #1 \right]}}

\newcommand{\chapternum}{}
\newcommand{\ex}[1]{\noindent\textbf{Exercise \chapternum.{#1}.}}

\newcommand{\tsub}[1]{\textsubscript{#1}}
\newcommand{\tsup}[1]{\textsuperscript{#1}}

% Include figures
\newcommand{\incfig}[2][1]{%
    \def\svgwidth{#1\columnwidth}
    \import{./figures/}{#2.pdf_tex}
}

\definecolor{problemBackground}{RGB}{212,232,246}

\newenvironment{problem}[1]
  {
    \begin{tcolorbox}[
      boxrule=.5pt,
      titlerule=.5pt,
      sharp corners,
      colback=problemBackground,
      breakable
    ]
    \ifx &#1& \textbf{Problem. }
    \else \textbf{Problem #1.} \fi
  }
  {
    \end{tcolorbox}
  }
\definecolor{exampleBackground}{RGB}{255,249,248}
\definecolor{exampleAccent}{RGB}{158,60,14}
\newenvironment{example}[1]
  {
    \begin{tcolorbox}[
      boxrule=.5pt,
      sharp corners,
      colback=exampleBackground,
      colframe=exampleAccent,
    ]
    \color{exampleAccent}\textbf{Example.} \emph{#1}\color{black}
  }
  {
    \end{tcolorbox}
  }
\definecolor{theoremBackground}{RGB}{234,243,251}
\definecolor{theoremAccent}{RGB}{0,116,183}
\newenvironment{theorem}[1]
  {
    \begin{tcolorbox}[
      boxrule=.5pt,
      titlerule=.5pt,
      sharp corners,
      colback=theoremBackground,
      colframe=theoremAccent,
      breakable
    ]
      \color{theoremAccent}\textbf{Theorem --- }\emph{#1}\\\color{black}
  }
  {
    \end{tcolorbox}
  }
\definecolor{noteBackground}{RGB}{244,249,244}
\definecolor{noteAccent}{RGB}{34,139,34}
\newenvironment{note}[1]
  {
  \begin{tcolorbox}[
    enhanced,
    boxrule=0pt,
    frame hidden,
    sharp corners,
    colback=noteBackground,
    borderline west={3pt}{-1.5pt}{noteAccent},
    breakable
    ]
    \ifx &#1& \color{noteAccent}\textbf{Note. }\color{black}
    \else \color{noteAccent}\textbf{Note (#1). }\color{black} \fi
    }
    {
  \end{tcolorbox}
  }
\definecolor{lemmaBackground}{RGB}{255,247,234}
\definecolor{lemmaAccent}{RGB}{255,153,0}
\newenvironment{lemma}[1]
  {
  \begin{tcolorbox}[
    enhanced,
    boxrule=0pt,
    frame hidden,
    sharp corners,
    colback=lemmaBackground,
    borderline west={3pt}{-1.5pt}{lemmaAccent},
    breakable
    ]
    \ifx &#1& \color{lemmaAccent}\textbf{Lemma. }\color{black}
    \else \color{lemmaAccent}\textbf{Lemma #1. }\color{black} \fi
    }
    {
  \end{tcolorbox}
  }
\definecolor{definitionBackground}{RGB}{246,246,246}
\newenvironment{definition}[1]
  {
    \begin{tcolorbox}[
      enhanced,
      boxrule=0pt,
      frame hidden,
      sharp corners,
      colback=definitionBackground,
      borderline west={3pt}{-1.5pt}{black},
      breakable
    ]
    \textbf{Definition. }\emph{#1}\\
  }
  {
    \end{tcolorbox}
  }

\newenvironment{amatrix}[2]{
    \left[
      \begin{array}{*{#1}{c}|*{#2}c}
  }
  {
      \end{array}
    \right]
  }
\definecolor{codeBackground}{RGB}{253,246,225}
\definecolor{dkgreen}{rgb}{0,0.6,0}
\definecolor{gray}{rgb}{0.5,0.5,0.5}
\definecolor{mauve}{rgb}{0.58,0,0.82}
\lstset{
  language=C++,
  aboveskip=3mm,
  belowskip=3mm,
  backgroundcolor=\color{codeBackground},
  showstringspaces=false,
  columns=flexible,
  basicstyle={\small\ttfamily},
  numbers=none,
  numberstyle=\tiny\color{gray},
  keywordstyle=\color{blue},
  commentstyle=\color{dkgreen},
  stringstyle=\color{mauve},
  breaklines=true,
  breakatwhitespace=true,
  tabsize=2
}

\date{\the\year-\the\month-\the\day}
\author{Kyle Chui}


\fancyhf{}
\lhead{Kyle Chui}
\rhead{Page \thepage}
\pagestyle{fancy}

\begin{document}
  \section{Lecture 17}
  \subsection{High-Mass Star's Life}
  A high-mass star is one with more than 10x the mass of the Sun.
  \begin{itemize}
    \item Main sequence stars fuse hydrogen into helium
    \item Red super giants fuse hydrogen into helium around an inert helium core
    \item Helium core burning: Helium fuses to carbon in the core but there is no ``flash'' this time
  \end{itemize}
  \begin{definition}{CNO Cycle}
    The \emph{CNO cycle} is another way to fuse H into He, using carbon, nitrogen and oxygen as catalysts.
  \end{definition}
  High-mass stars make the elements necessary for life (heavier elements). At a certain point, massive stars go through multiple shell fusion (different elements at different shells). Iron is the end for fusion because nuclear reactions involving iron \emph{don't} release energy. \par
  For a massive star with a core of inert iron dies, it turns into a neutron star. Energy is added during supernova explosions, and is created in the death of stars.
  \subsubsection{Summary}
  \begin{itemize}
    \item A star with a mass $8$ $M_\text{sun}$ or higher becomes a black hole or neutron star.
    \item These make most oxygen, silicon, magnesium, sulfur
    \item Type I supernovae are merging or exploded white dwarfs
    \item These stars make most of the iron and nickel
    \item Source of heavier elements---rare earth metals not fully known. Either merged neutron stars or massive star supernova
    \item The sun cannot become a supernova
  \end{itemize}
  Incidence of planets higher near metal rich stars
  \subsection{SETI}
  Low mass stars may have more planets than higher mass stars (although having smaller habitability zones). While lower mass stars are longer lived, their close-in planets might be tidally locked and suffer from stellar flares.
  \begin{itemize}
    \item The first life forms might have arisen more than $5$B years ago
    \item Their legacy might be Terra formed planetary systems
  \end{itemize}
  \subsection{The Drake Equation}
  An equation that is a way to estimate the number of civilisations in our galaxy at this time. It is given by:
  \[
    \text{Number of civilisations} = N_\text{hab}\cdot f_\text{life}\cdot f_\text{civ}\cdot f_\text{now}.
  \]
  \begin{itemize}
    \item $N_\text{hab}$ is the total number of habitable planets in the galaxy (20--200 billion)
    \item $f_\text{life}$ is the fraction of those planets with life
    \item $f_\text{civ}$ is the fraction of those that have ever had tech civilisations
    \item $f_\text{now}$ is the fraction of those that are transmitting now
  \end{itemize}
  By random approximations, there might be between $0$ and $10$ tech civilisation in the Milky Way.
  \subsection{The Future of Civilisation}
  \begin{itemize}
    \item Unless a star has very low mass, a solar mass host star eventually leaves main sequence
    \item This requires migration to a new planet
  \end{itemize}
  \subsubsection{Why Bother Broadcasting?}
  \begin{itemize}
    \item The closest star is $4$ light-years away (8 year round trip)
    \item Everything points towards the speed of light being the fastest thing in the universe
  \end{itemize}
  \begin{definition}{Convergent Evolution}
    Different organisms evolving towards the same features.
  \end{definition}
  \begin{definition}{Encephalisation Quotient}
    The EQ of an organism is the ratio of its brain mass to body mass.
  \end{definition}
  There are many species that have deep intelligence and are conscious, but not technological.
  \subsection{How does SETI Work?}
  \begin{itemize}
    \item Use radio telescopes to look for signals
    \item Searches millions of frequencies at once
  \end{itemize}
  Modern SETI signals fall into three categories:
  \begin{enumerate}
    \item Local communications (Earth-based radio)
    \item Interplanetary or interstellar signals (communicating over long distances)
    \item Signal beacons (broadcasting their signals specifically to attract attention)
  \end{enumerate}
  \begin{note}{}
    The $1420$ MHz or $21$ cm hydrogen band is seen as important.
  \end{note}
  Modern SETI uses radio, because it's cheap. If aliens knew about our galaxy, they would probably leave artifacts at the L4 and L5 points in orbit because items are stable there. \par
  Fermi's Paradox: Where are all the aliens?
\end{document}
