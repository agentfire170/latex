\documentclass[class=article, crop=false]{standalone}
% Import packages
\usepackage[margin=1in]{geometry}

\usepackage[many]{tcolorbox}
\usepackage{amssymb, amsthm}
\usepackage{comment}
\usepackage{enumitem}
\usepackage{fancyhdr}
\usepackage{hyperref}
\usepackage{import}
\usepackage{listings}
\usepackage{mathrsfs, mathtools}
\usepackage{pdfpages}
\usepackage{standalone}
\usepackage{transparent}
\usepackage{xcolor}

\usetikzlibrary{decorations.pathreplacing}
\tcbuselibrary{skins}
% Declare math operators
\DeclareMathOperator{\lcm}{lcm}
\DeclareMathOperator{\proj}{proj}
\DeclareMathOperator{\vspan}{span}
\DeclareMathOperator{\im}{im}
\DeclareMathOperator{\range}{range}
\DeclareMathOperator{\Diff}{Diff}
\DeclareMathOperator{\Int}{Int}
\DeclareMathOperator{\fcn}{fcn}
\DeclareMathOperator{\id}{id}
\DeclareMathOperator{\rank}{rank}
\DeclareMathOperator{\tr}{tr}
\DeclareMathOperator{\dive}{div}
\DeclareMathOperator{\row}{row}
\DeclareMathOperator{\col}{col}
% Macros for letters/variables
\renewcommand{\tilde}{\raisebox{0.4ex}{\resizebox{2ex}{!}{\texttildelow}}}
\newcommand{\N}{\ensuremath{\mathbb{N}}}
\newcommand{\Z}{\ensuremath{\mathbb{Z}}}
\newcommand{\Q}{\ensuremath{\mathbb{Q}}}
\newcommand{\R}{\ensuremath{\mathbb{R}}}
\newcommand{\C}{\ensuremath{\mathbb{C}}}
\newcommand{\F}{\ensuremath{\mathbb{F}}}
\newcommand{\M}{\ensuremath{\mathbb{M}}}
\newcommand{\lam}{\ensuremath{\lambda}}
\newcommand{\nab}{\ensuremath{\nabla}}
\newcommand{\eps}{\ensuremath{\varepsilon}}
\newcommand{\es}{\ensuremath{\varnothing}}
% Macros for math symbols
\newcommand{\dx}[1]{\,\mathrm{d}#1}
\newcommand{\inv}{\ensuremath{^{-1}}}
\newcommand{\sm}{\setminus}
\newcommand{\sse}{\subseteq}
\newcommand{\ceq}{\coloneqq}
% Macros for pairs of math symbols
\newcommand{\abs}[1]{\ensuremath{\left\lvert #1 \right\rvert}}
\newcommand{\paren}[1]{\ensuremath{\left( #1 \right)}}
\newcommand{\norm}[1]{\ensuremath{\left\lVert #1\right\rVert}}
\newcommand{\set}[1]{\ensuremath{\left\{#1\right\}}}
\newcommand{\tup}[1]{\ensuremath{\left\langle #1 \right\rangle}}
\newcommand{\floor}[1]{\ensuremath{\left\lfloor #1 \right\rfloor}}
\newcommand{\ceil}[1]{\ensuremath{\left\lceil #1 \right\rceil}}
\newcommand{\eclass}[1]{\ensuremath{\left[ #1 \right]}}

\newcommand{\chapternum}{}
\newcommand{\ex}[1]{\noindent\textbf{Exercise \chapternum.{#1}.}}

\newcommand{\tsub}[1]{\textsubscript{#1}}
\newcommand{\tsup}[1]{\textsuperscript{#1}}

% Include figures
\newcommand{\incfig}[2][1]{%
    \def\svgwidth{#1\columnwidth}
    \import{./figures/}{#2.pdf_tex}
}

\definecolor{problemBackground}{RGB}{212,232,246}

\newenvironment{problem}[1]
  {
    \begin{tcolorbox}[
      boxrule=.5pt,
      titlerule=.5pt,
      sharp corners,
      colback=problemBackground,
      breakable
    ]
    \ifx &#1& \textbf{Problem. }
    \else \textbf{Problem #1.} \fi
  }
  {
    \end{tcolorbox}
  }
\definecolor{exampleBackground}{RGB}{255,249,248}
\definecolor{exampleAccent}{RGB}{158,60,14}
\newenvironment{example}[1]
  {
    \begin{tcolorbox}[
      boxrule=.5pt,
      sharp corners,
      colback=exampleBackground,
      colframe=exampleAccent,
    ]
    \color{exampleAccent}\textbf{Example.} \emph{#1}\color{black}
  }
  {
    \end{tcolorbox}
  }
\definecolor{theoremBackground}{RGB}{234,243,251}
\definecolor{theoremAccent}{RGB}{0,116,183}
\newenvironment{theorem}[1]
  {
    \begin{tcolorbox}[
      boxrule=.5pt,
      titlerule=.5pt,
      sharp corners,
      colback=theoremBackground,
      colframe=theoremAccent,
      breakable
    ]
      \color{theoremAccent}\textbf{Theorem --- }\emph{#1}\\\color{black}
  }
  {
    \end{tcolorbox}
  }
\definecolor{noteBackground}{RGB}{244,249,244}
\definecolor{noteAccent}{RGB}{34,139,34}
\newenvironment{note}[1]
  {
  \begin{tcolorbox}[
    enhanced,
    boxrule=0pt,
    frame hidden,
    sharp corners,
    colback=noteBackground,
    borderline west={3pt}{-1.5pt}{noteAccent},
    breakable
    ]
    \ifx &#1& \color{noteAccent}\textbf{Note. }\color{black}
    \else \color{noteAccent}\textbf{Note (#1). }\color{black} \fi
    }
    {
  \end{tcolorbox}
  }
\definecolor{lemmaBackground}{RGB}{255,247,234}
\definecolor{lemmaAccent}{RGB}{255,153,0}
\newenvironment{lemma}[1]
  {
  \begin{tcolorbox}[
    enhanced,
    boxrule=0pt,
    frame hidden,
    sharp corners,
    colback=lemmaBackground,
    borderline west={3pt}{-1.5pt}{lemmaAccent},
    breakable
    ]
    \ifx &#1& \color{lemmaAccent}\textbf{Lemma. }\color{black}
    \else \color{lemmaAccent}\textbf{Lemma #1. }\color{black} \fi
    }
    {
  \end{tcolorbox}
  }
\definecolor{definitionBackground}{RGB}{246,246,246}
\newenvironment{definition}[1]
  {
    \begin{tcolorbox}[
      enhanced,
      boxrule=0pt,
      frame hidden,
      sharp corners,
      colback=definitionBackground,
      borderline west={3pt}{-1.5pt}{black},
      breakable
    ]
    \textbf{Definition. }\emph{#1}\\
  }
  {
    \end{tcolorbox}
  }

\newenvironment{amatrix}[2]{
    \left[
      \begin{array}{*{#1}{c}|*{#2}c}
  }
  {
      \end{array}
    \right]
  }
\definecolor{codeBackground}{RGB}{253,246,225}
\definecolor{dkgreen}{rgb}{0,0.6,0}
\definecolor{gray}{rgb}{0.5,0.5,0.5}
\definecolor{mauve}{rgb}{0.58,0,0.82}
\lstset{
  language=C++,
  aboveskip=3mm,
  belowskip=3mm,
  backgroundcolor=\color{codeBackground},
  showstringspaces=false,
  columns=flexible,
  basicstyle={\small\ttfamily},
  numbers=none,
  numberstyle=\tiny\color{gray},
  keywordstyle=\color{blue},
  commentstyle=\color{dkgreen},
  stringstyle=\color{mauve},
  breaklines=true,
  breakatwhitespace=true,
  tabsize=2
}

\date{\the\year-\the\month-\the\day}
\author{Kyle Chui}


\fancyhf{}
\lhead{Kyle Chui}
\rhead{Page \thepage}
\pagestyle{fancy}

\begin{document}
  \section{Electric Potential Energy and Voltage}
  You can store energy as electric potential energy and convert it back to kinetic energy on demand. To move a charge against the electric force generated by a capacitor, you need to perform work: $W = F\cdot d = (q\cdot E)\cdot d$. If you let the charge go, we have conservation of energy: $qEd = \frac{1}{2}mv^2$. \par
  All this is nice, but if you double the charge then you double the work and stored energy. To get a quantity that describes the energy storage capability of the field (that does not depend on $q$), we must define a new quantity. We say that \emph{electric potential} is defined by $V = \frac{U}{q}$. It is measured in $\frac{\mathrm{J}}{\mathrm{C}}$, or in volts ($\mathrm{V}$). For a homogeneous electric field, we have
  \[
    V = \frac{U}{q} = \frac{qEd}{q} = E\cdot d.
  \]
  \begin{note}{}
    Since we have $E = \frac{V}{d}$, we can measure electric field in $\frac{\mathrm{V}}{\mathrm{m}}$ and $\frac{\mathrm{N}}{\mathrm{C}}$.
  \end{note}
  \subsection{Gravity Analogue to Electric Potential}
  \begin{center}\begin{tabular}{c|c}
    Gravitational Potential Energy & Electric Potential Energy \\
    \hline
    $U = mgh$ & $U = qEd$
  \end{tabular}\end{center}
  \begin{itemize}
    \item Move a mass/charge against the field/force increases $U$ and $V$.
    \item Moving a mass/charge perpendicular to the field does not change $U$ and $V$.
    \item $V$ is a measure of how much energy a mass/charge would have at that point.
    \item Equipotential lines are perpendicular to the field. No work is done along those lines.
  \end{itemize}
  We define \emph{voltage} to be the potential difference between two points: $\Delta V = V_A - V_B$. Potential is a relative number but voltage is not. You must set the potential to zero somewhere (typically ground, one plate, or far away from a charge).
  \begin{note}{}
    Work performed by moving a charge $q$ in an electric field $E$ from point $A$ to point $B$ does not depend on the path taken. This is because both $F_g$ and $F_e$ are conservative forces.
  \end{note}
  \begin{definition}{Electric Potential Energy}
    We denote \emph{electric potential energy} by $U$ and it is given by
    \[
      U = W = -\int_{i}^{f}\vec{F} \dx \ell = -q \int_{i}^{f}\vec{E} \dx \ell.
    \]
  \end{definition}
  \begin{definition}{Electric Potential}
    We denote \emph{electric potential} by $V$ and it is given by
    \[
      V = -\int_{i}^{f}\vec{E} \dx \ell.
    \]
    It can also be found using $V = \frac{U}{q}$, but $V$ depends on the shape of the field.
  \end{definition}
  \begin{example}{Potential of a Point Charge} \\
    For a point charge (or sphere), the potential energy increases as you get closer to it. We can find the potential energy by integrating the force with respect to distance, yielding
    \begin{align*}
      W &= U \\
        &= -\int F\dx r \\
        &= -\int k\frac{Q_1Q_2}{r^2}\dx r \\
        &= \frac{kQ_1Q_2}{r} + C.
    \end{align*}
    Thus the electric potential is $V = \frac{W}{q} = k\frac{Q}{r} + C$. We define the constant of integration to be zero when the radius approaches infinity.
  \end{example}
  Inside a conducting sphere, the field is zero so the potential stays constant. Outside the sphere however, it behaves just like a point charge so we have $V = k\frac{q}{r}$. \par
  We draw equipotential lines to designate what paths have constant electric potential. They are always perpendicular to the electric field lines.
  \begin{note}{}
    $V$ is a scalar field and only depends on $r$, but $E$ is a vector field---the direction of it depends on the angle.
  \end{note}
  \subsection{General Rules for Potential}
  \begin{itemize}
    \item Moving a charge perpendicular to $E$ does not require work so $V$ does not change.
    \item $V$ increases as you move against $E$, and decreases as you move with $E$.
    \item If the field is zero then the potential remains constant.
    \item $V$ is only a relative number. We need to define where $V = 0$ (by grounding).
    \item A positive test charge wants to move to regions of smaller $V$, and a negative charge wants to move to regions of larger $V$.
  \end{itemize}
  \begin{example}{Mass Spectrometer} \\
    A mass spectrometer functions by ionizing particles and seeing how much they are deflected by a constant magnetic field (this will tell you the mass).
  \end{example}
  \subsection{Electric Potential In and Around a Conducting Cylinder}
  Outside of the charge, we have $\vec{E} = \frac{2k\lam}{\rho}\vec{a}_\rho$, where $\lam$ is the linear charge density. Inside the cylinder, the electric field is zero. Integrating along the radial path, we have
  \begin{align*}
    \Delta V &= -\int_{i}^{f}\vec{E}\cdot  \dx \vec{\rho} \\
             &= -2k\lam\int_{i}^{f}\frac{\mathrm{d}\rho}{\rho} \\
             &= -2k\lam\cdot \ln(\rho)|_{\rho_i}^{\rho_f} \\
             &= -2k\lam\cdot (\ln \rho_f - \ln\rho_i) \\
             &= \boxed{2k\lam\cdot \ln \paren{\frac{\rho_i}{\rho_f}}.}
  \end{align*}
  We set $V_i = 0$ at some arbitrary distance $\rho_0$. Then we have
  \[
    \boxed{V(\rho) = k\lam\cdot \ln \paren{\frac{\rho_0}{\rho}}.}
  \]
  \subsection{The Potential Gradient}
  We know that $\Delta V = -\int_{i}^{f}\vec{E}\cdot  \dx \vec{\ell}$, so we can find $E$ from $V$. For short distances $\Delta\ell$, it turns out that
  \[
    \Delta V = -\vec{E}\cdot \Delta\vec{\ell}.
  \]
  Thus $\Delta V = -E\cdot \Delta \ell\cdot \cos\theta$, so we have
  \[
    \boxed{\frac{\mathrm{d}V}{\mathrm{d}\ell} = -E\cdot \cos\theta.}
  \]
  In general for three-dimensions, we have
  \[
    \vec{\nab}V = -\vec{E}.
  \]
  \subsection{The Electric Field of a Dipole}
  We apply the principle of superposition. Thus the potential at point $P$ is 
  \[
    V = kQ \paren{\frac{1}{R_1}-\frac{1}{R_2}} = kQ \paren{\frac{R_2-R_1}{R_1R_2}}.
  \]
  If $P$ is very very far away from the dipole ($R \gg d$), then
  \[
    R_2 - R_1 = d\cdot \cos\theta.
  \]
  Thus we have
  \[
    V = \frac{kQd\cos\theta}{R_1R_2} = \frac{kQd\cos\theta}{r^2}.
  \]
  Plugging this into the formula we found in the earlier section, we have
  \[
    \vec{E} = \underbrace{\frac{kQd}{r^3}}_{\text{magnitude}}\underbrace{(2\cos\theta\cdot \vec{a}_r+\sin\theta\cdot \vec{a}_\theta)}_{\text{direction}}
  \]
  From this, we can see that $E\sim \frac{1}{r^3}$.
\end{document}
