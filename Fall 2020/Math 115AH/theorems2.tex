\documentclass[class=article, crop=false]{standalone}
% Import packages
\usepackage[margin=1in]{geometry}

\usepackage[many]{tcolorbox}
\usepackage{amssymb, amsthm}
\usepackage{comment}
\usepackage{enumitem}
\usepackage{fancyhdr}
\usepackage{hyperref}
\usepackage{import}
\usepackage{listings}
\usepackage{mathrsfs, mathtools}
\usepackage{pdfpages}
\usepackage{standalone}
\usepackage{transparent}
\usepackage{xcolor}

\usetikzlibrary{decorations.pathreplacing}
\tcbuselibrary{skins}
% Declare math operators
\DeclareMathOperator{\lcm}{lcm}
\DeclareMathOperator{\proj}{proj}
\DeclareMathOperator{\vspan}{span}
\DeclareMathOperator{\im}{im}
\DeclareMathOperator{\range}{range}
\DeclareMathOperator{\Diff}{Diff}
\DeclareMathOperator{\Int}{Int}
\DeclareMathOperator{\fcn}{fcn}
\DeclareMathOperator{\id}{id}
\DeclareMathOperator{\rank}{rank}
\DeclareMathOperator{\tr}{tr}
\DeclareMathOperator{\dive}{div}
\DeclareMathOperator{\row}{row}
\DeclareMathOperator{\col}{col}
% Macros for letters/variables
\renewcommand{\tilde}{\raisebox{0.4ex}{\resizebox{2ex}{!}{\texttildelow}}}
\newcommand{\N}{\ensuremath{\mathbb{N}}}
\newcommand{\Z}{\ensuremath{\mathbb{Z}}}
\newcommand{\Q}{\ensuremath{\mathbb{Q}}}
\newcommand{\R}{\ensuremath{\mathbb{R}}}
\newcommand{\C}{\ensuremath{\mathbb{C}}}
\newcommand{\F}{\ensuremath{\mathbb{F}}}
\newcommand{\M}{\ensuremath{\mathbb{M}}}
\newcommand{\lam}{\ensuremath{\lambda}}
\newcommand{\nab}{\ensuremath{\nabla}}
\newcommand{\eps}{\ensuremath{\varepsilon}}
\newcommand{\es}{\ensuremath{\varnothing}}
% Macros for math symbols
\newcommand{\dx}[1]{\,\mathrm{d}#1}
\newcommand{\inv}{\ensuremath{^{-1}}}
\newcommand{\sm}{\setminus}
\newcommand{\sse}{\subseteq}
\newcommand{\ceq}{\coloneqq}
% Macros for pairs of math symbols
\newcommand{\abs}[1]{\ensuremath{\left\lvert #1 \right\rvert}}
\newcommand{\paren}[1]{\ensuremath{\left( #1 \right)}}
\newcommand{\norm}[1]{\ensuremath{\left\lVert #1\right\rVert}}
\newcommand{\set}[1]{\ensuremath{\left\{#1\right\}}}
\newcommand{\tup}[1]{\ensuremath{\left\langle #1 \right\rangle}}
\newcommand{\floor}[1]{\ensuremath{\left\lfloor #1 \right\rfloor}}
\newcommand{\ceil}[1]{\ensuremath{\left\lceil #1 \right\rceil}}
\newcommand{\eclass}[1]{\ensuremath{\left[ #1 \right]}}

\newcommand{\chapternum}{}
\newcommand{\ex}[1]{\noindent\textbf{Exercise \chapternum.{#1}.}}

\newcommand{\tsub}[1]{\textsubscript{#1}}
\newcommand{\tsup}[1]{\textsuperscript{#1}}

% Include figures
\newcommand{\incfig}[2][1]{%
    \def\svgwidth{#1\columnwidth}
    \import{./figures/}{#2.pdf_tex}
}

\definecolor{problemBackground}{RGB}{212,232,246}

\newenvironment{problem}[1]
  {
    \begin{tcolorbox}[
      boxrule=.5pt,
      titlerule=.5pt,
      sharp corners,
      colback=problemBackground,
      breakable
    ]
    \ifx &#1& \textbf{Problem. }
    \else \textbf{Problem #1.} \fi
  }
  {
    \end{tcolorbox}
  }
\definecolor{exampleBackground}{RGB}{255,249,248}
\definecolor{exampleAccent}{RGB}{158,60,14}
\newenvironment{example}[1]
  {
    \begin{tcolorbox}[
      boxrule=.5pt,
      sharp corners,
      colback=exampleBackground,
      colframe=exampleAccent,
    ]
    \color{exampleAccent}\textbf{Example.} \emph{#1}\color{black}
  }
  {
    \end{tcolorbox}
  }
\definecolor{theoremBackground}{RGB}{234,243,251}
\definecolor{theoremAccent}{RGB}{0,116,183}
\newenvironment{theorem}[1]
  {
    \begin{tcolorbox}[
      boxrule=.5pt,
      titlerule=.5pt,
      sharp corners,
      colback=theoremBackground,
      colframe=theoremAccent,
      breakable
    ]
      \color{theoremAccent}\textbf{Theorem --- }\emph{#1}\\\color{black}
  }
  {
    \end{tcolorbox}
  }
\definecolor{noteBackground}{RGB}{244,249,244}
\definecolor{noteAccent}{RGB}{34,139,34}
\newenvironment{note}[1]
  {
  \begin{tcolorbox}[
    enhanced,
    boxrule=0pt,
    frame hidden,
    sharp corners,
    colback=noteBackground,
    borderline west={3pt}{-1.5pt}{noteAccent},
    breakable
    ]
    \ifx &#1& \color{noteAccent}\textbf{Note. }\color{black}
    \else \color{noteAccent}\textbf{Note (#1). }\color{black} \fi
    }
    {
  \end{tcolorbox}
  }
\definecolor{lemmaBackground}{RGB}{255,247,234}
\definecolor{lemmaAccent}{RGB}{255,153,0}
\newenvironment{lemma}[1]
  {
  \begin{tcolorbox}[
    enhanced,
    boxrule=0pt,
    frame hidden,
    sharp corners,
    colback=lemmaBackground,
    borderline west={3pt}{-1.5pt}{lemmaAccent},
    breakable
    ]
    \ifx &#1& \color{lemmaAccent}\textbf{Lemma. }\color{black}
    \else \color{lemmaAccent}\textbf{Lemma #1. }\color{black} \fi
    }
    {
  \end{tcolorbox}
  }
\definecolor{definitionBackground}{RGB}{246,246,246}
\newenvironment{definition}[1]
  {
    \begin{tcolorbox}[
      enhanced,
      boxrule=0pt,
      frame hidden,
      sharp corners,
      colback=definitionBackground,
      borderline west={3pt}{-1.5pt}{black},
      breakable
    ]
    \textbf{Definition. }\emph{#1}\\
  }
  {
    \end{tcolorbox}
  }

\newenvironment{amatrix}[2]{
    \left[
      \begin{array}{*{#1}{c}|*{#2}c}
  }
  {
      \end{array}
    \right]
  }
\definecolor{codeBackground}{RGB}{253,246,225}
\definecolor{dkgreen}{rgb}{0,0.6,0}
\definecolor{gray}{rgb}{0.5,0.5,0.5}
\definecolor{mauve}{rgb}{0.58,0,0.82}
\lstset{
  language=C++,
  aboveskip=3mm,
  belowskip=3mm,
  backgroundcolor=\color{codeBackground},
  showstringspaces=false,
  columns=flexible,
  basicstyle={\small\ttfamily},
  numbers=none,
  numberstyle=\tiny\color{gray},
  keywordstyle=\color{blue},
  commentstyle=\color{dkgreen},
  stringstyle=\color{mauve},
  breaklines=true,
  breakatwhitespace=true,
  tabsize=2
}

\date{\the\year-\the\month-\the\day}
\author{Kyle Chui}


\fancyhf{}
\lhead{Kyle Chui}
\rhead{Page \thepage}
\pagestyle{fancy}

\begin{document}
  \begin{theorem}{Cauchy-Schwarz Inequality}
    Let $V$ be an inner product space over $F$. Then for all $v_1, v_2\in V$, we have
    \[
      \abs{\tup{v_1, v_2}} \leq \norm{v_1}\norm{v_2}.
    \]
  \end{theorem}
  \begin{proof}
    If either $v_1 = 0$ or $v_2 = 0$, then the result is immediate, so we may assume that $v_1\leq 0$, $v_2\neq 0$. \par
    We define $v$ such that
    \[
      v = v_2 - \frac{\tup{v_2, v_1}}{\norm{v_1}^2}v_1, \tag{Projection onto $v_1$}
    \]
    and claim that $\tup{v, \alpha v_1} = 0$ for all $\alpha\in F$. Observe that
    \begin{align*}
      \tup{v, \alpha v_1} &= \tup{v_2 - \frac{\tup{v_2, v_1}}{\norm{v_1}^2}v_1, \alpha v_1} \\
                          &= \tup{v_2, \alpha v_1} - \tup{\frac{\tup{v_2, v_1}}{\norm{v_1}^2}v_1, \alpha v_1} \\
                          &= \tup{v_2, \alpha v_1} - \frac{\tup{v_2, v_1}}{\norm{v_1}^2}\tup{v_1, \alpha v_1} \\
                          &= \bar{\alpha}\tup{v_2, v_1} - \frac{\tup{v_2, v_1}}{\norm{v_1}^2}\bar{\alpha}\tup{v_1, v_1} \\
                          &= \bar{\alpha}\tup{v_2, v_1} - \bar{\alpha}\tup{v_2, v_1} \\
                          &= 0.
    \end{align*}
    Thus the claim has been shown. We now prove the rest of the theorem. We know that inner products of vectors with themselves are always non-negative, so
    \begin{align*}
      0 &\leq \tup{v, v} \\
        &= \tup{v, v_2 - \frac{\tup{v_2, v_1}}{\norm{v_1}^2}v_1} \\
        &= \tup{v, v_2} + \tup{v, -\frac{\tup{v_2, v_1}}{\norm{v_1}^2}v_1} \\
        &= \tup{v, v_2} \\
        &= \tup{v_2 - \frac{\tup{v_2, v_1}}{\norm{v_1}^2}v_1, v_2} \\
        &= \tup{v_2, v_2} + \tup{-\frac{\tup{v_2, v_1}}{\norm{v_1}^2}v_1, v_2} \\
        &= \tup{v_2, v_2} - \frac{\tup{v_2, v_1}}{\norm{v_1}^2}\tup{v_1, v_2} \\
        &= \norm{v_2}^2 - \frac{\overline{\tup{v_1, v_2}}\tup{v_1, v_2}}{\norm{v_1}^2} \\
        &= \norm{v_2}^2 - \frac{\abs{\tup{v_1, v_2}}^2}{\norm{v_1}^2}\\
      0 &\leq \norm{v_1}^2\norm{v_2}^2 - \abs{\tup{v_1, v_2}}^2 \\
      \abs{\tup{v_1, v_2}}^2 &\leq \norm{v_1}^2\norm{v_2}^2 \\
      \abs{\tup{v_1, v_2}} &\leq \norm{v_1}\norm{v_2}
    \end{align*}
  \end{proof}
  \newpage
  \begin{theorem}{Minkowski Inequality}
    Let $V$ be an inner product space over $F$. Then for all $v_1, v_2\in V$, we have
    \[
      \norm{v_1 + v_2}\leq \norm{v_1} + \norm{v_2}.
    \]
  \end{theorem}
  \begin{proof}
    Observe that
    \begin{align*}
      \norm{v_1 + v_2}^2 &= \tup{v_1 + v_2, v_1 + v_2} \\
                         &= \tup{v_1, v_1} + \tup{v_2, v_1} + \tup{v_1, v_2} + \tup{v_2, v_2} \\ 
                         &= \tup{v_1, v_1} + \tup{v_2, v_1} + \overline{\tup{v_1, v_2}} + \tup{v_2, v_2} 
      \intertext{Suppose $\tup{v_1, v_2} = \alpha + \beta \sqrt{-1}$, so then $\tup{v_1, v_2} + \overline{\tup{v_1, v_2}} = \alpha + \beta \sqrt{-1} + \alpha - \beta \sqrt{-1} = 2\alpha$.}
                         &= \norm{v_1}^2 + 2\alpha + \norm{v_2}^2 \\
                         &\leq \norm{v_1}^2 + 2\sqrt{\alpha^2+\beta^2} + \norm{v_2}^2 \\
                         &= \norm{v_1}^2 + 2\abs{\tup{v_1, v_2}} + \norm{v_2}^2 \\ 
                         &\leq \norm{v_1}^2 + 2\norm{v_1}\norm{v_2} + \norm{v_2}^2 \tag{Cauchy-Schwarz} \\
                         &= \paren{\norm{v_1} + \norm{v_2}}^2.
    \end{align*}
    Thus we have shown that $\norm{v_1+v_2}^2 \leq \paren{\norm{v_1} + \norm{v_2}}^2$, so we have $\norm{v_1+v_2}\leq \norm{v_1}+\norm{v_2}$.
  \end{proof}
  \newpage
  \begin{theorem}{Gram-Schmidt Theorem}
    Let $V$ be an inner product space over $F$ and $\es\neq S_n = \set{v_1,\dotsc,v_n}\sse V$ a linearly independent set. Then there exist $y_1,\dotsc,y_n\in V$ such that
    \begin{enumerate}[label=(\roman*)]
      \item $y_1 = v_1$
      \item $T_n = \set{y_1,\dotsc,y_n}$ is an orthogonal set and linearly independent
      \item $\vspan(T_n) = \vspan(S_n)$
    \end{enumerate}
  \end{theorem}
  \begin{proof}
    We construct $T_n$ from $S_n$ using the Gram-Schmidt Process, and proceed by induction. When $n = 1$, we set $y_1 = v_1$, and the statements above clearly hold. Suppose the statement is true for some $n = k$, so 
    \begin{enumerate}[label=(\roman*)]
      \item $y_1 = v_1$,
      \item $T_k = \set{y_1,\dotsc,y_k}$ is an orthogonal set and linearly independent,
      \item $\vspan(T_k) = \vspan(S_k)$.
    \end{enumerate}
    We define
    \[
      y_{k+1} = v_{k+1} - \sum_{i=1}^{k}\frac{\tup{v_{k+1},y_i}}{\norm{y_i}^2}y_i. \tag{*}
    \]
    We know that $y_{k+1}\neq 0$, because otherwise $v_{k+1}\in \vspan(T_k) = \vspan(S_k)$ and $\set{v_1,\dotsc,v_{k+1}}$ would not be linearly independent, a contradiction. \\[10pt]
    We claim that $\tup{y_{k+1}, y_j} = 0$ for all $j = 1,\dotsc,k$. We can see that
    \begin{align*}
      \tup{y_{k+1}, y_j} &= \tup{v_{k+1} - \sum_{i=1}^{k}\frac{\tup{v_{k+1},y_i}}{\norm{y_i}^2}y_i, y_j} \\
                         &= \tup{v_{k+1}, y_j} - \sum_{i=1}^{k}\frac{\tup{v_{k+1},y_i}}{\norm{y_i}^2}\tup{y_i,y_j} \\
                         &= \tup{v_{k+1}, y_j} - \frac{\tup{v_{k+1},y_j}}{\norm{y_j}^2}\tup{y_j, y_j} \tag{$\tup{y_i, y_j} = \delta_{ij}$}\\
                         &= \tup{v_{k+1}, y_j} - \tup{v_{k+1}, y_j} \\
                         &= 0.
    \end{align*}
    Since $0\notin T_{k+1} = \set{y_1,\dotsc,y_{k+1}}$ and $T_{k+1}$ is an orthogonal set, it must be linearly independent. Furthermore, we know by rearranging (*) that $v_{k+1}\in \vspan(y_1,\dotsc,y_{k+1})$ and has a non-zero component for $y_{k+1}$, and so by the Replacement Theorem
    \[
      \vspan(T_{k+1}) = \vspan(y_1,\dotsc,y_{k+1}) = \vspan(y_1,\dotsc,y_k,v_{k+1}) = \vspan(v_1,\dotsc,v_{k+1}) = S_{k+1}.
    \]
  \end{proof}
  \newpage
  \begin{theorem}{Orthogonal Theorem}
    Let $V$ be a finite-dimensional inner product space over $F$. Then $V$ has an orthogonal basis. If $F = \R$ or $\C$, then $V$ has an orthonormal basis.
  \end{theorem}
  \begin{proof}
    Let $\mathcal{B} = \set{v_1,\dotsc,v_n}$ be a basis for $V$. By the Gram-Schmidt Theorem, we know that there exists a linearly independent and orthogonal set $\mathcal{C} = \set{w_1,\dotsc,w_n}$ such that $\vspan(\mathcal{C}) = \vspan(\mathcal{B}) = V$. Because $\mathcal{C}$ is both linearly independent and spans $V$, it must be a basis for $V$, and so is an orthogonal basis for $V$. \par
    If $F = \R$, then $\set{\frac{w}{\norm{w}}\mid w\in \mathcal{C}}$ is an orthonomal basis for $V$ as $\norm{w}\in \R$ for all $w\in \mathcal{C}$. The same reasoning applies for $F = \C$.
  \end{proof}
  \newpage
  \begin{theorem}{Orthogonal Decomposition Theorem}
    Let $V$ be an inner product space over $F$ (not necessarily finite-dimensional), with $S\sse V$ a finite-dimensional subspace, and $v\in V$. Then there exist unique $s\in S$ and $s^\perp\in S^\perp$ such that $v = s + s^\perp$. In particular, $V = S + S^\perp$ and $S\cap S^\perp = 0$, so $V = S\perp S^\perp$. Moreover, if 
    \[
      v = s + s^\perp, s\in S, s^\perp\in S^\perp,
    \]
    then
    \[
      \norm{v}^2 = \norm{s}^2 + \norm{s^\perp}^2. \tag{Pythagorean Theorem}
    \]
    If, in addition, $V$ is a finite-dimensional inner product space over $F$, then
    \[
      \dim(V) = \dim(S) + \dim(S^\perp).
    \]
  \end{theorem}
  \begin{proof}
    By the Orthogonal Theorem, we know that there exists an orthogonal basis $\mathcal{B} = \set{v_1,\dotsc,v_n}$ for $S$, the finite-dimensional inner product space over $F$. We will first show existence. Let $v\in V$. We define $s\in S = \vspan(\mathcal{B})$ by
    \[
      s = \sum_{i=1}^{n}\frac{\tup{v, v_i}}{\norm{v_i}^2}v_i,
    \]
    and $s^\perp = v - s$. Then $v = s + s^\perp$ and $S\cap S^\perp = 0$, i.e. $V = S\oplus S^\perp$. We first show that $s^\perp\perp v_j$ for $j = 1,\dotsc,n$. Observe that
    \begin{align*}
      \tup{s^\perp, v_j} &= \tup{v-s, v_j} \\
                         &= \tup{v, v_j} - \tup{s, v_j} \\
                         &= \tup{v, v_j} - \tup{\sum_{i=1}^{n}\frac{\tup{v, v_i}}{\norm{v_i}^2}v_i, v_j} \\
                         &= \tup{v, v_j} - \sum_{i=1}^{n}\frac{\tup{v, v_i}}{\norm{v_i}^2}\tup{v_i, v_j} \\
                         &= \tup{v, v_j} - \frac{\tup{v, v_j}}{\norm{v_j}^2}\tup{v_j, v_j} \tag{$\tup{v_i, v_j} = \delta_{ij}$} \\
                         &= \tup{v, v_j} - \tup{v, v_j} \\
                         &= 0.
    \end{align*}
    Since $s = \sum_{i=1}^{n}\alpha_iv_i\in S$, we have
    \begin{align*}
      \tup{s, s^\perp} &= \tup{\sum_{i=1}^{n}\alpha_iv_i, s^\perp} \\
                       &= \sum_{i=1}^{n}\alpha_i \tup{v_i, s^\perp} \\
                       &= 0.
    \end{align*}
    Thus we have shown that $s^\perp\perp s$ for all $s\in S$, so $s^\perp\in S^\perp$, as needed. We must now show uniqueness. Suppose
    \[
      s + s^\perp = v = r + r^\perp.
    \]
    Then $s-r = r^\perp - s^\perp$. However, because $S\cap S^\perp = 0$, we have $s - r = 0 = r^\perp - s^\perp$, so $s = r$ and $s^\perp = r^\perp$. Therefore the decomposition of $v$ is unique. \\[10pt]
    To show the Pythagorean Theorem, we have
    \[
      \norm{v}^2 = \norm{s + s^\perp}^2 = \tup{s + s^\perp, s + s^\perp} = \tup{s, s} + \tup{s, s^\perp} + \tup{s^\perp, s} + \tup{s^\perp, s^\perp} = \norm{s}^2 + \norm{s^\perp}^2.
    \]
    If $V$ is finite-dimensional, by the Counting Theorem, we have that
    \[
      \dim(V) = \dim(S + S^\perp) = \dim(S) + \dim(S^\perp) - \dim(S\cap S^\perp) = \dim(S) + \dim(S^\perp).
    \]
  \end{proof}
  \newpage
  \begin{theorem}{Approximation Theorem}
    Let $V$ be an inner product space over $F$, $S\sse V$ a finite-dimensional subspace, and $v\in V$. Then $v_S$ is \emph{closer} to $v$ than any other vector in $S$, i.e.,
    \[
      d(v, v_S) = \norm{v - v_S} \leq \norm{v - r} = d(v, r)
    \]
    in $\R$ for all $r\in S$. Equivalently, we may write
    \[
      d(v, S) = d(v, v_S).
    \]
    Moreover, if $r\in S$, then
    \[
      \norm{v - v_S} = \norm{v - r}\text{ in $\R$ if and only if }r = v_S.
    \]
    We say $v_S$ gives the \emph{best approximation} to $v$ in $S$.
  \end{theorem}
  \begin{proof}
    Let $v = s + s^\perp$ with $s = v_S$, $s^\perp = v - s = v - v_S$ and $s^\perp\in S^\perp$. Let $r\in S$. Then
    \[
      v - r = (v - v_S) + (v_S - r) = s^\perp + (v_S - r).
    \]
    Because $v_S - r\in S$, we have $\norm{v - r}^2 = \norm{v - v_S}^2 + \norm{v_S - r}^2\geq \norm{v - v_S}^2$ with equality if and only if $\norm{v_S - r}^2 = 0$, or $v_S = r$.
  \end{proof}
  \newpage
  \begin{theorem}{Hermitian Corollary}
    Let $V$ be an inner product space over $F$, $T\colon V\to V$ linear. Suppose that $T$ is hermitian. Then any eigenvalue of $T$ (if any) is real, i.e. lies in $F\cap \R$.
  \end{theorem}
  \begin{proof}
    Because $T$ is hermitian, we have $\tup{Tv, v} = \tup{v, Tv}$ for all $v\in V$. If $\lam$ is an eigenvalue of $T$, then
    \[
      \lam = \frac{\lam \tup{v, v}}{\norm{v}^2} = \frac{\tup{Tv, v}}{\norm{v}^2} = \frac{\tup{v, Tv}}{\norm{v}^2} = \frac{\bar{\lam}\tup{v, v}}{\norm{v}^2} = \bar{\lam},
    \]
    so $\lam = \bar{\lam}$ and $\lam$ must be real.
  \end{proof}
  \newpage
  \begin{theorem}{Key Lemma (for Hermitian Operators)}
    Let $V$ be an inner product space over $F$, $T\colon V\to V$ hermitian, $S\sse V$ a $T$-invariant subspace. Then
    \begin{enumerate}[label=(\roman*)]
      \item $S^\perp$ is $T$-invariant, i.e. $T(S^\perp)\sse S^\perp$
      \item $T|_{S^\perp}\colon S^\perp\to S^\perp$ is hermitian
    \end{enumerate}
  \end{theorem}
  \begin{proof}
    (i) Let $s^\perp\in S^\perp$. Then it suffices to show that $Ts^\perp\in S^\perp$. Observe that for all $s\in S$, we have
    \begin{align*}
      \tup{s, Ts^\perp} &= \tup{Ts, s^\perp} \tag{$T$ is hermitian} \\
                        &= 0. \tag{$S$ is a $T$-invariant subspace}
    \end{align*}
    Therefore $Ts^\perp\in S^\perp$ so $S^\perp$ is $T$-invariant. \\[10pt]
    (ii) By (i) we know that $T|_{S^\perp}$ is linear, and since $T$ is hermitian in $V$, it must be hermitian in $S^\perp$. In other words, because $\tup{Tv, w} = \tup{v, Tw}$ for all $v, w\in V$, it must also hold for all $v, w\in S^\perp$.
  \end{proof}
  \newpage
  \begin{theorem}{Spectral Theorem (for Hermitian Operators) (Refined Version)}
    Let $F = \R$ or $\C$, $V$ a finite-dimensional inner product space over $F$, $T\colon V\to V$ hermitian. Then there exists an ordered orthonormal basis $\mathcal{C}$ of eigenvectors for $V$ of $T$ and every eigenvalue of $T$ is real. Moreover, if $\mathcal{B}$ is any ordered orthonormal basis for $V$, then
    \[
      [T]_{\mathcal{C}} = C[T]_{\mathcal{B}}C^*
    \]
    for some invertible matrix $C\in \M_nF$, i.e. $C = [1_V]_{\mathcal{B},\mathcal{C}}$.
  \end{theorem}
  \begin{theorem}{Spectral Theorem (for Hermitian Operators) (Refined Version)}
    Let $F = \R$ or $\C$, $V$ a finite-dimensional inner product space over $F$, $T\colon V\to V$ hermitian. Then there exists an orthonormal basis $\mathcal{B} = \set{v_1,\dotsc,v_n}$ for $V$ with each $v_i$, $i=1,\dotsc,n$ an eigenvector for some eigenvalue $\lam_i\in \R$, $i=1,\dotsc,n$ (not necessarily distinct). In particular, $T$ is diagonalizable.
  \end{theorem}
  \begin{proof}
    We prove $\mathcal{B}$ exists by induction on $\dim(V) = n$. \par
    Observe that for the $n = 1$ case, we have that $V= \vspan(v)$, for any non-zero $v\in V$. As $Tv\in \vspan(v)$, there exists an $\alpha\in F$ such that $Tv = \alpha v$, so $v\in E_T(\alpha)$. As $T$ is hermitian, $\alpha\in \R$ by the Hermitian Corollary, even if $F = \C$. Thus $\mathcal{B} = \set{\frac{v}{\norm{v}}}$ is a valid orthonormal basis for $V$. \par
    Suppose the statement holds for some $n$, that is if $W$ is a finite-dimensional inner product space over $F$, $\dim(W) = n-1$, $T_0\colon W\to W$ hermitian, then there exists an orthonormal basis for $W$ of eigenvectors of $T_0$ and every eigenvalue of $T_0$ is real. Let $\mathcal{C}$ be an ordered orthonormal basis for $n$-dimensional $V$, which exists as $F = \R$ or $\C$. Let $A = [T]_{\mathcal{C}}\in \M_nF\sse \M_n\C$. Then for all $x, y\in \C^{n\times 1}$, we have that $Ax\cdot y = x\cdot Ay$ (because $T$ is hermitian). In other words, $A\colon \C^{n\times 1}\to \C^{n\times 1}$ is hermitian, where $\C^{n\times 1}$ is an inner product space over $\C$ via the dot product. By the Fundamental Theorem of Algebra, $f_A$ has a root $\lam\in \C$, hence $\lam$ is an eigenvalue of $A\colon \C^{n\times 1}\to \C^{n\times 1}$, which is hermitian. Thus by the Hermitian Corollary, we know that $\lam\in \R$. But
    \[
      f_T = f_{[T]_{\mathcal{C}}} = f_A,
    \]
    so $f_T$ has a real root $\lam\in \R$, if $F = \R$ or $\C$. Thus there exists a non-zero vector $v\in E_T(\lam)\sse V$ that is an eigenvector of $T$. Let $Fv = \vspan(v)\sse E_T(\lam)$. Then $Fv$ is $T$-invariant. By the Orthogonal Decomposition Theorem
    \[
      v = Fv \perp (Fv)^\perp
    \]
    and
    \[
      \dim(V) = \dim(Fv) + \dim(Fv)^\perp = 1 + \dim(Fv)^\perp.
    \]
    Hence $\dim(Fv)^\perp = n - 1$. By the Key Lemma, since $Fv$ is $T$-invariant and $T\colon V\to V$ is hermitian, $(Fv)^\perp$ is $T$-invariant. 
  \end{proof}
  \newpage
  \begin{theorem}{New Key Lemma}
    Let $V$ be a finite-dimensional inner product space over $F$, $T\colon V\to V$ linear. Suppose that $V$ has an orthonormal basis and $W\sse V$ is a $T$-invariant subspace. Then $W^\perp\sse V$ is $T$-invariant. In particular,
    \[
      T^*|_{W^\perp}\colon W^\perp\to W^\perp
    \]
    is linear.
  \end{theorem}
  \begin{proof}
    Let $w^\perp\in W^\perp$ and $x\in W$ be arbitrary. Then
    \[
      \tup{x, T^*w^\perp} = \tup{Tx, w^\perp} = 0,
    \]
    as $Tx\in W$ (because $W$ is $T$-invariant). Thus $T^*w^\perp\in W^\perp$, and so $W^\perp$ is also $T$-invariant.
  \end{proof}
  \newpage
  \begin{theorem}{Schur's Theorem}
    Let $V$ be a finite-dimensional inner product space over $\C$, $T\colon V\to V$ linear. Then $T$ is triangularizable. Moreover, there exists an ordered orthonormal basis $\mathcal{B}$ for $T$ such that $[T]_{\mathcal{B}}$ is upper triangular.
  \end{theorem}
  \begin{proof}
    We induct on $n = \dim(V)$. Let $n = 1$. Then for some non-zero $v\in V$, we have that $\mathcal{B} = \set{\frac{v}{\norm{v}}}$ is a valid orthonormal basis. Let $n > 1$. By the Fundamental Theorem of Algebra, the characteristic polynomial $f_{T^*}$ for $T^*$ has a root $\lam\in \C$, hence $\lam$ is an eigenvalue of $T^*$. Let $0\neq v\in E_{T^*}(\lam)$. By the Orthogonal Decomposition Theorem, 
    \[
      V = (\C v) + \paren{\C v}^\perp
    \]
    and
    \begin{align*}
      n &= \dim(V) \\
        &= \dim(\C v) + \dim(\C v)^\perp \\
        &= 1 + \dim(\C v)^\perp,
    \end{align*}
    so $\dim(\C v)^\perp = n - 1$. We know that $(\C v)$ is $T^*$-invariant as $v\in E_{T^*}(\lam)$, so $(\C v)^\perp$ is $(T^*)^* = T$-invariant by the New Key Lemma. So we may view
    \[
      T|_{(\C v)^\perp}\colon (\C v)^\perp\to (\C v)^\perp \text{ as linear.}
    \]
    By induction, there exists an ordered orthonormal basis $\mathcal{B}_0 = \set{v_1,\dotsc,v_{n-1}}$ for $(\C v)^\perp$ such that $[T|_{(\C v)^\perp}]_{\mathcal{B}_0}$ is upper triangular. Let $\mathcal{B} = \set{v_1,\dotsc,v_{n-1}, \frac{v}{\norm{v}}}$, an ordered orthonormal basis for $V$. Then by (*), we have
    \[
      \begin{bmatrix}
        [T|_{(\C v)^\perp}]_{\mathcal{B}_0} & \cdots & * \\
        \vdots & \ddots & * \\
        0 & \cdots & *
      \end{bmatrix} \in \M_n\C,
    \]
    where the last column (denoted by asterisks) is $\left[T \paren{\dfrac{v}{\norm{v}}}\right]_{\mathcal{B}}$.
  \end{proof}
  \newpage
  \begin{theorem}{Spectral Theorem (for Normal Operators)}
    Let $V$ be a finite-dimensional inner product space over $\C$, $T\colon V\to V$ normal. Then there exists an ordered orthonormal basis $\mathcal{C}$ for $V$ consisting of eigenvectors of $T$. In particular, $T$ is diagonalizable. Moreover, if $\mathcal{B}$ is an ordered orthonormal basis for $V$, then
    \[
      T_{\mathcal{C}} = [1_V]_{\mathcal{B},\mathcal{C}}[T]_{\mathcal{B}}[1_V]_{\mathcal{B},\mathcal{C}}^*.
    \]
  \end{theorem}
  \begin{proof}
    We induct on $n = \dim(V)$. When $n = 1$, the statement is immediate. Let $n > 1$. By the Fundamental Theorem of Algebra, there exists $\bar{\lam}\in \C$ a root of $f_{T^*}\in \C[t]$, and thus is an eigenvalue of $T^*$. Let $0\neq v\in E_{T^*}\paren{\bar{\lam}}$. By the lemma, $v\in E_T(\lam)$. Thus $(\C v)$ is both $T$-invariant and $T^*$-invariant. Hence by the New Key Lemma we know that $(\C v)^\perp$ is also both $T$-invariant and $T^*$-invariant. In particular, for all $x, y\in (\C v)^\perp$, we have
    \[
      \tup{x, T^*y} = \tup{Tx, y}
    \]
    and $\paren{T|_{(\C v)^\perp}}^*$ is the unique linear map 
    \[
      \paren{T|_{(\C v)^\perp}}^*\colon (\C v)^\perp\to (\C v)^\perp
    \]
    such that for all $x, y\in (\C v)^\perp$,
    \begin{align*}
      \tup{x, \paren{T|_{(\C v)^\perp}}^*y}_{(\C v)^\perp} &= \tup{T|_{(\C v)^\perp}x, y}_{(\C v)^\perp} \\
                                                                 &= \tup{Tx, y}_V \\
                                                                 &= \tup{x, T^*y}_V \\
                                                                 &= \tup{x, T^*|_{(\C v)^\perp}y}_{(\C v)^\perp}.
    \end{align*}
    It follows by the uniqueness of the adjoint that $T^*|_{(\C v)^\perp} = \paren{T|_{(\C v)^\perp}}^*$. Hence we have
    \[
      T|_{(\C v)^\perp}\colon (\C v)^\perp\to (\C v)^\perp \text{ is also normal.}
    \]
    Since
    \[
      \dim(V) = \dim(\C v) + \dim((\C v)^\perp) = 1 + \dim((\C v)^\perp),
    \]
    by the Orthogonal Decomposition Theorem, by induction there exists an orthonormal basis $\mathcal{C}_0 = \set{v_2,\dotsc,v_n}$ for $(\C v)^\perp$ of eigenvectors of $T|_{(\C v)^\perp}$ hence of eigenvectors of $T$. It follows that
    \[
      \mathcal{C} = \set{\frac{v}{\norm{v}}, v_2,\dotsc,v_n}
    \]
    is an orthonormal basis for $V$ consisting of eigenvectors of $T$. If $\mathcal{B}$ is an orthonormal basis for $V$, then $[1_V]_{\mathcal{B},\mathcal{C}}* = [1_V]_{\mathcal{B},\mathcal{C}}\inv$, so
    \[
      T_{\mathcal{C}} = [1_V]_{\mathcal{B},\mathcal{C}}[T]_{\mathcal{B}}[1_V]_{\mathcal{B},\mathcal{C}}^*
    \]
    by the Change of Basis Theorem.
  \end{proof}
  \newpage
  \begin{theorem}{Singular Value Theorem}
    Let $F = \R$ or $\C$, $A\in F^{m\times n}$. Then there exists $U\in U_n(F)\ceq \set{B\in \M_nF\mid BB^* = I}$, $X\in U_mF$ such that
    \[
      X^*AU = D\ceq \begin{bmatrix}u_1 & \cdots & 0 \\\vdots & \ddots & \vdots\\ 0 & \cdots & 0\end{bmatrix}\in F^{m\times n}
    \]
    is ``diagonal'' (i.e., $D_{ij} = 0$ for all $i\neq j$) with $D_{ii} = 0$ for all $i > r$, $D_{ii} = \mu_i, i\leq r$ with
    \[
      u_1\geq \dotsb \geq u_r > 0 \quad \text{and}\quad r = \rank(A).
    \]
  \end{theorem}
\end{document}
