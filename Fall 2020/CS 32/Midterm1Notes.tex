\documentclass[class=article, crop=false]{standalone}
% Import packages
\usepackage[margin=1in]{geometry}

\usepackage[many]{tcolorbox}
\usepackage{amssymb, amsthm}
\usepackage{comment}
\usepackage{enumitem}
\usepackage{fancyhdr}
\usepackage{hyperref}
\usepackage{import}
\usepackage{listings}
\usepackage{mathrsfs, mathtools}
\usepackage{pdfpages}
\usepackage{standalone}
\usepackage{transparent}
\usepackage{xcolor}

\usetikzlibrary{decorations.pathreplacing}
\tcbuselibrary{skins}
% Declare math operators
\DeclareMathOperator{\lcm}{lcm}
\DeclareMathOperator{\proj}{proj}
\DeclareMathOperator{\vspan}{span}
\DeclareMathOperator{\im}{im}
\DeclareMathOperator{\range}{range}
\DeclareMathOperator{\Diff}{Diff}
\DeclareMathOperator{\Int}{Int}
\DeclareMathOperator{\fcn}{fcn}
\DeclareMathOperator{\id}{id}
\DeclareMathOperator{\rank}{rank}
\DeclareMathOperator{\tr}{tr}
\DeclareMathOperator{\dive}{div}
\DeclareMathOperator{\row}{row}
\DeclareMathOperator{\col}{col}
% Macros for letters/variables
\renewcommand{\tilde}{\raisebox{0.4ex}{\resizebox{2ex}{!}{\texttildelow}}}
\newcommand{\N}{\ensuremath{\mathbb{N}}}
\newcommand{\Z}{\ensuremath{\mathbb{Z}}}
\newcommand{\Q}{\ensuremath{\mathbb{Q}}}
\newcommand{\R}{\ensuremath{\mathbb{R}}}
\newcommand{\C}{\ensuremath{\mathbb{C}}}
\newcommand{\F}{\ensuremath{\mathbb{F}}}
\newcommand{\M}{\ensuremath{\mathbb{M}}}
\newcommand{\lam}{\ensuremath{\lambda}}
\newcommand{\nab}{\ensuremath{\nabla}}
\newcommand{\eps}{\ensuremath{\varepsilon}}
\newcommand{\es}{\ensuremath{\varnothing}}
% Macros for math symbols
\newcommand{\dx}[1]{\,\mathrm{d}#1}
\newcommand{\inv}{\ensuremath{^{-1}}}
\newcommand{\sm}{\setminus}
\newcommand{\sse}{\subseteq}
\newcommand{\ceq}{\coloneqq}
% Macros for pairs of math symbols
\newcommand{\abs}[1]{\ensuremath{\left\lvert #1 \right\rvert}}
\newcommand{\paren}[1]{\ensuremath{\left( #1 \right)}}
\newcommand{\norm}[1]{\ensuremath{\left\lVert #1\right\rVert}}
\newcommand{\set}[1]{\ensuremath{\left\{#1\right\}}}
\newcommand{\tup}[1]{\ensuremath{\left\langle #1 \right\rangle}}
\newcommand{\floor}[1]{\ensuremath{\left\lfloor #1 \right\rfloor}}
\newcommand{\ceil}[1]{\ensuremath{\left\lceil #1 \right\rceil}}
\newcommand{\eclass}[1]{\ensuremath{\left[ #1 \right]}}

\newcommand{\chapternum}{}
\newcommand{\ex}[1]{\noindent\textbf{Exercise \chapternum.{#1}.}}

\newcommand{\tsub}[1]{\textsubscript{#1}}
\newcommand{\tsup}[1]{\textsuperscript{#1}}

% Include figures
\newcommand{\incfig}[2][1]{%
    \def\svgwidth{#1\columnwidth}
    \import{./figures/}{#2.pdf_tex}
}

\definecolor{problemBackground}{RGB}{212,232,246}

\newenvironment{problem}[1]
  {
    \begin{tcolorbox}[
      boxrule=.5pt,
      titlerule=.5pt,
      sharp corners,
      colback=problemBackground,
      breakable
    ]
    \ifx &#1& \textbf{Problem. }
    \else \textbf{Problem #1.} \fi
  }
  {
    \end{tcolorbox}
  }
\definecolor{exampleBackground}{RGB}{255,249,248}
\definecolor{exampleAccent}{RGB}{158,60,14}
\newenvironment{example}[1]
  {
    \begin{tcolorbox}[
      boxrule=.5pt,
      sharp corners,
      colback=exampleBackground,
      colframe=exampleAccent,
    ]
    \color{exampleAccent}\textbf{Example.} \emph{#1}\color{black}
  }
  {
    \end{tcolorbox}
  }
\definecolor{theoremBackground}{RGB}{234,243,251}
\definecolor{theoremAccent}{RGB}{0,116,183}
\newenvironment{theorem}[1]
  {
    \begin{tcolorbox}[
      boxrule=.5pt,
      titlerule=.5pt,
      sharp corners,
      colback=theoremBackground,
      colframe=theoremAccent,
      breakable
    ]
      \color{theoremAccent}\textbf{Theorem --- }\emph{#1}\\\color{black}
  }
  {
    \end{tcolorbox}
  }
\definecolor{noteBackground}{RGB}{244,249,244}
\definecolor{noteAccent}{RGB}{34,139,34}
\newenvironment{note}[1]
  {
  \begin{tcolorbox}[
    enhanced,
    boxrule=0pt,
    frame hidden,
    sharp corners,
    colback=noteBackground,
    borderline west={3pt}{-1.5pt}{noteAccent},
    breakable
    ]
    \ifx &#1& \color{noteAccent}\textbf{Note. }\color{black}
    \else \color{noteAccent}\textbf{Note (#1). }\color{black} \fi
    }
    {
  \end{tcolorbox}
  }
\definecolor{lemmaBackground}{RGB}{255,247,234}
\definecolor{lemmaAccent}{RGB}{255,153,0}
\newenvironment{lemma}[1]
  {
  \begin{tcolorbox}[
    enhanced,
    boxrule=0pt,
    frame hidden,
    sharp corners,
    colback=lemmaBackground,
    borderline west={3pt}{-1.5pt}{lemmaAccent},
    breakable
    ]
    \ifx &#1& \color{lemmaAccent}\textbf{Lemma. }\color{black}
    \else \color{lemmaAccent}\textbf{Lemma #1. }\color{black} \fi
    }
    {
  \end{tcolorbox}
  }
\definecolor{definitionBackground}{RGB}{246,246,246}
\newenvironment{definition}[1]
  {
    \begin{tcolorbox}[
      enhanced,
      boxrule=0pt,
      frame hidden,
      sharp corners,
      colback=definitionBackground,
      borderline west={3pt}{-1.5pt}{black},
      breakable
    ]
    \textbf{Definition. }\emph{#1}\\
  }
  {
    \end{tcolorbox}
  }

\newenvironment{amatrix}[2]{
    \left[
      \begin{array}{*{#1}{c}|*{#2}c}
  }
  {
      \end{array}
    \right]
  }
\definecolor{codeBackground}{RGB}{253,246,225}
\definecolor{dkgreen}{rgb}{0,0.6,0}
\definecolor{gray}{rgb}{0.5,0.5,0.5}
\definecolor{mauve}{rgb}{0.58,0,0.82}
\lstset{
  language=C++,
  aboveskip=3mm,
  belowskip=3mm,
  backgroundcolor=\color{codeBackground},
  showstringspaces=false,
  columns=flexible,
  basicstyle={\small\ttfamily},
  numbers=none,
  numberstyle=\tiny\color{gray},
  keywordstyle=\color{blue},
  commentstyle=\color{dkgreen},
  stringstyle=\color{mauve},
  breaklines=true,
  breakatwhitespace=true,
  tabsize=2
}

\date{\the\year-\the\month-\the\day}
\author{Kyle Chui}


\fancyhf{}
\lhead{CS32 Midterm $1$ Notes}
\rhead{Page \thepage}
\pagestyle{fancy}

\begin{document}
  \tableofcontents
  \newpage
  \section{Algorithms and Data Structures}
  \begin{definition}{Algorithms}
    An \emph{algorithm} is a set of instructions/steps that solves a particular problem.
  \end{definition}
  Each algorithm operates on input data, and produces an output result. Algorithms can be classified by how long they take to run on particular input, and by the quality of their results. 
  \begin{definition}{Data Structures}
    A \emph{data structure} is the set of variable(s) that an algorithm uses to solve a problem.
  \end{definition}
  To solve a problem, you have $o$ design both the algorithms and the data structures together. You then need to provide a set of simple ``interface'' functions to let any programmer use them easily. We call this an ADT, or abstract data type.
  \subsection{Abstract Data Types}
  \begin{definition}{ADT}
    An \emph{ADT} or \emph{abstract data type} is a coordinated group of data structures, algorithms, and interface functions that is used to solve a particular problem.
  \end{definition}
  In an ADT, the data structures and algorithms are kept secret, while the interface is public to enable the rest of the program to use the ADT. Typically, a program is built from a collection of ADTs, each of which solves a different sub-problem.
  \subsubsection{ADTs in C++}
  In C++, we use classes to define ADTs in our programs. Each C++ class can hold algorithms, data, and interface functions.
  \begin{itemize}
    \item Once the class has been defined, the rest of our program can treat the functions as a ``black box'', i.e. not knowing how exactly how the functions work.
    \item All our program needs to do is call the functions in our class' public interface.
    \item All of the underlying data structures and algorithms are hidden from the user.
    \item The rest of the program can ignore the details of how things work and just use its features.
  \end{itemize}
  \begin{note}{}
    The benefit of this is that you can change the implementation of the class as much as you want, and as long as the interface functions do what they're supposed to, nothing will break outside of the class. This reduces complexity by breaking down large, complex problems into smaller, self-contained chunks.
  \end{note}
  \newpage
  \subsection{Defining a Class in C++}
  We first write the outer shell of the class and give it a name, then give the class public and private variables/functions. An example of a \texttt{Nerd} class header file is given below:
  \begin{lstlisting}
    // Nerd.h
    class Nerd
    {
    public:
      Nerd(int stink, int IQ) {
        myStinkiness = stink;
        myIQ = IQ;
      }
      void study (int hours) {
        myStinkiness += 3*hours;
        myIQ *= 1.01;
      }
      int getStinkyLevel() {
        int total_stink = myIQ * 10 + myStinkiness;
        return total_stink;
      }
    private:
      int myStinkiness, myIQ;
    };
  \end{lstlisting}
  \begin{note}{}
    Don't forget the semicolon at the end of the class!
  \end{note}
  \subsection{Using a New Class}
  Once we define a new class, we can use it to define variables like any traditional data type.
  \begin{itemize}
    \item The class defines a new data type like int, float, or string.
    \item You typically define each class in its own \texttt{.h} file (``header file'') and put the file in the same folder as your \texttt{.cpp} files. For example, in a file titled \texttt{ucla.cpp}, we could have
    \begin{lstlisting}
      // ucla.cpp
      #include "Nerd.h"
      int main()
      {
        int num_nerds = 1;
        Nerd david(30, 150);

        david.study(10);
      }
    \end{lstlisting}
    \item A header file is similar to a \texttt{.cpp} file except you typically only put class declarations and constants in it (you typically put the actual function bodies in your \texttt{.cpp} file).
    \item To import the contents of a header file, just put \texttt{\#include "filename.h"} at the beginning of the file.
    \item Once an instance of the class is defined, you can call its member functions, i.e. \texttt{study}.
    \begin{note}{}
      A class' primitive types start out with random values and not zero, so remember to initialise them with the constructor.
    \end{note}
    \item You typically only use member variables to store permanent attributes of a class, i.e. \texttt{myStinkiness} and \texttt{myIQ}.
    \item All functions in the \texttt{public} section of your class can be seen/called by all parts of your program.
    \item All functions and data defined in the \texttt{private} section of your class are hidden from the rest of your program.
    \begin{definition}{Encapsulation}
      Hiding the internal implementation details of a class (from the rest of your program) is called \emph{encapsulation}.
    \end{definition}
  \end{itemize}
  \section{Constructors and Destructors}
  \textbf{Main Idea.} A constructor is used to reset an object's member variables when the object is first created; otherwise they'd be random. An object will often reserve memory slots from the operating system while it runs, so we need a destructor function to free that memory when we delete the object.
  \subsection{Constructor Basics}
  We create a constructor just like any other function, but it doesn't have a return type (not even \texttt{void}).
  \begin{lstlisting}
    // Gassy
    class Gassy
    {
    public:
      // The constructor
      Gassy(int age, bool ateBeans)
      {
        m_age = age;
        m_ateBeans = ateBeans;
      }

      int getFartsPerHr()
      {
        if (m_ateBeans == true)
          return(100);
        return(3 * m_age);
      }
    private:
      // Member variables
      int m_age;
      bool m_ateBeans;
    };
  \end{lstlisting}
  \lstset{backgroundcolor=\color{noteBackground}}
  \begin{lstlisting}
    int main()
    {
      Gassy betty(18, true); // Requires two arguments!

      Gassy alan; // error!
    }
  \end{lstlisting}
  A class can have many different constructors, as long as they have different parameters and/or types. 
  \begin{definition}{Constructor Overloading}
    When we have multiple constructors in a class, we call this \emph{constructor overloading}.
  \end{definition}
  \begin{note}{}
    If you have a constructor that doesn't take any parameters, you may omit the parentheses when creating an instance of the class.
  \end{note}
  \subsection{When Constructors are Called}
  \begin{itemize}
    \item When you create a new variable of a class.
    \item A constructor is called $N$ times when you create an array of size $N$ (that holds the object type).
    \item A constructor is called when you use \texttt{new} to dynamically allocate a new variable.
    \item If a variable is declared in a loop, it is newly constructed during \emph{every} iteration.
    \item A constructor is \emph{not called} when you just define a pointer variable.
  \end{itemize}
  \subsection{Destructors}
  Just as every class has a constructor, every class also has a destructor function (it may have \emph{exactly one} destructor). It does not have any parameters. To define a constructor function, place a tilde $n$ front of the name of the class, i.e.
  \lstset{backgroundcolor=\color{codeBackground}}
  \begin{lstlisting}
    class Gassy
    {
    public:
      ~Gassy()
      {
        // Destructor code goes here
      }
    private:
      ...
    }
  \end{lstlisting}
  \subsubsection{Why Do We Need Destructors?}
  If we did not have a destructor, everytime we create a new instance of the class we would take up more and more memory, eventually running out of memory on our system. We need to use destructors to free up memory taken up by deleted objects.
  \begin{definition}{Memory leak}
    A \emph{memory leak} occurs when you delete an object without first clearing all of the reserved space for that object, effectively ``leaking'' memory.
  \end{definition}
  \subsection{When Must You Have a Destructor?}
  \begin{center}
    \begin{tabular}{cc}
      Any time a class allocates & Your class must have \\
      a system resource\ldots & a destructor that\ldots \\[10pt]
      Reserves memory using  & Frees the allocated memory \\
      the \texttt{new} command & with the \texttt{delete} command \\[10pt]
      Opens a disk file & Closes the disk file \\[10pt]
      Connects to another computer  & Disconnects from the \\
      over the network & other computer 
    \end{tabular}
  \end{center}
  \newpage
  \section{Class Composition}
  \textbf{Main Idea.} Classes inside classes. Need I say more?
  \subsection{Construction Order}
  \begin{itemize}
    \item C++ always constructs member variables first, in the order that they're defined in the class.
    \item Then, C++ constructs the outer class after the member variables have been initialised/constructed.
  \end{itemize}
  \begin{note}{}
    This makes sense. Since the outer class constructor might need the member variables for construction, they must be constructed before the outer variable.
  \end{note}
  \subsection{Destruction Order}
  \begin{itemize}
    \item C++ always runs the main/outer object's destructor first. Its member variables are still valid when the outer destructor runs, so the outer destructor can run properly.
    \item After the outer class's destructor runs, then C++ runs the destructors of the member variables in the \emph{reverse order} they were constructed.
  \end{itemize}
  \begin{note}{Auto-generated Constructors/Destructors}
    If we leave out a constructor and/or destructor, C++ auto-generates an empty constructor and/or destructor for your class. Be careful, because the default constructor will not initialise your primitive member variables. If your class has any primitives you should define your own constructor and initialise them.
  \end{note}
  \subsection{Initialiser Lists}
  Any time you have a member variable that requires \emph{one or more} parameters for construction, you \emph{must} add an initaliser list to \emph{all} of your outer class's constructor. The syntax is as follows:
  \begin{lstlisting}
    class HungryNerd
    {
    public:
      HungryNerd() :
        myBelly(10), myBrain(150), myAge(19) // Runs the corresponding constructors
                                                 // with the given values
      {
        myBelly.eat();
      }
    private:
      Belly myBelly;
      Brain myBrain;
      int myAge;
    };
  \end{lstlisting}
  \newpage
  \section{Miscellaneous Topics}
  \subsection{Include Etiquette}
  \begin{enumerate}[label=(\alph*)]
    \item \emph{Never} include a \texttt{.cpp} file in another \texttt{.cpp} or \texttt{.h} file.
    \item \emph{Never} put a \texttt{using namespace} command in a header file. This is called ``namespace pollution'' and is bad because it forces \texttt{.cpp} files that include the header file to use its namespace. Instead, put all of the \texttt{using} commands into your C++ files.
    \item \emph{Never} assume that a header file will include some other header file for you. If whoever manages the header file decides to change its implementation, then that will break your code.
  \end{enumerate}
  \subsection{Preprocessor Directives (Include Guards)}
  \textbf{Problem.} If our header file that we are including already includes another dependency that we are already including in our \texttt{.cpp} file, then we will have included that dependency twice! This causes issues like defining methods multiple times, etc.
  \begin{enumerate}[label=(\alph*)]
    \item First, we use \texttt{\#define} to define new constants, i.e. \texttt{\#define FOOBAR\_H}
    \item We then use the commands \texttt{\#ifdef} and/or \texttt{\#ifndef} to check if a variable has been defined (or not). To close the if statement, we use the \texttt{\#endif} command.
  \end{enumerate}
  Putting both of the above steps, we have something like this:
  \begin{lstlisting}
    #ifndef FOOBAR_H // Only run the following if FOOBAR_H isn't defined
    #define FOOBAR_H // Define FOOBAR_H
    // Notice that this only runs if it has not been run before
    class FooBar
    {
    public:
      some stuff...
    private:
      some other stuff...
    };
    #endif // End the if statement
  \end{lstlisting}
  We call this an ``include guard'', because it ``guards'' the class and prevents it from being included multiple times.
  \subsection{When to Include .h Files}
  You must include the header file (containing the full definition of the class) when:
  \begin{itemize}
    \item You define a regular variable of that class' type.
    \item You use the variable in any way (call a method on it, return it, etc).
  \end{itemize}
  You must do this because C++ needs to know the class's details in order to actually do anything with instances of that class. \\[10pt]
  You must give a class declaration (just \texttt{class ClassName;}) when:
  \begin{itemize}
    \item You use the class to define a parameter to a function.
    \item You use the class as the return type for a function.
    \item You use the class to define a pointer or reference variable.
  \end{itemize}
  Since you aren't actually using any of the methods of the class nor creating any instances of the class, you don't need to include the full class definition.
  \begin{note}{}
    The reason why we don't always include the full class definition is because the header file might be large (i.e. thousands of lines) or two classes refer to each other (causing a loop).
  \end{note}
  \subsection{Default Arguments}
  You can specify a default number of arguments for a function by defining it in the function header, i.e.
  \begin{lstlisting}
    void doSomething(int varOne = 0, string varTwo = "some value")
    {
      do something with the parameters...
    }
  \end{lstlisting}
  You can leave out the default parameter if you want and C++ will automatically pass in the default value, i.e. \texttt{doSomething(5);} or \texttt{doSomething();}. However, you can only pass the first $k$ elements as defaults, and may not skip defaults, i.e. \texttt{doSomething(,"some string")} is invalid. \\[10pt]
  One final quirk about default arguments is that if the $j$th parameter has a default value, then all of the subsequent parameters ($j+1$ to $n$) \emph{must} also have default values.
  \section{Addresses and Pointers}
  \textbf{Main Idea.} We use pointers to access/modify variables defined in other parts of our program. Just like every house has a street address, every variable has a memory address. A pointer is just a variable that holds another variable's address.
  \subsection{Every Variable Has an Address}
  Every time you define a variable in your program, the compiler finds an unused address in memory and reserves one or more bytes there to store your variable.
  \begin{note}{}
    The address of a variable is defined to be the \emph{lowest} address in memory where the variable is stored. For instance, if a variable takes up bytes \texttt{1006} to \texttt{1009}, then its address is \texttt{1006}.
  \end{note}
  \subsection{Getting the Address of a Variable}
  We can get the address of a variable by using C++'s \texttt{\&} operator. If you place an \texttt{\&} before a variable name in a program statement, it means ``give me the numerical address of the variable''. 
  \subsection{Pointers vs. Regular Variables}
  Instead of holding a regular value, a pointer variable holds \emph{another variable's address}.
  \begin{itemize}
    \item The way you define a pointer variable is by putting an asterisk \texttt{*} in front of the variable ame when you define it, i.e. \texttt{int* p} or \texttt{string* s}.
    \begin{note}{}
      You may also see the asterisk right next to the variable like \texttt{int *p}---there's no difference. I prefer to put it next to the type because I like to pretend that \texttt{int*} is its own type, i.e. a pointer to an integer.
    \end{note}
    \item The way you understand pointers is by reading from right to left, i.e. \texttt{string* s} reads ``\texttt{s} is a pointer to a string variable''.
    \item If a pointer \texttt{p} points to the address of some variable \texttt{myVar}, we may just say ``\texttt{p} points to \texttt{myVar}'' or ``\texttt{p} holds a value of [whatever \texttt{myVar}'s address is]''.
  \end{itemize}
  \subsection{Reading and Writing with Pointers}
  We can use the pointer and the star operator (also an asterisk) to read/write the other variable. If you have a pointer \texttt{p}, then \texttt{*p} (notice how the operator is on the left) is the variable that \texttt{p} points to. For example, consider the following code:
  \begin{lstlisting}
    int main()
    {
      int m_var = 42;

      int *p;
      p = &m_var; // #1

      cout << *p; // #2
      *p = 5; // #3
    }
  \end{lstlisting}
  In line \#1, we make \texttt{p} point to \texttt{m\_var}. That way, when we print \texttt{*p} on line \#2, we are really just printing out \texttt{m\_var}'s value, $42$. Finally on line \#3, we change the value of the variable that \texttt{p} points to (\texttt{m\_var}) to $5$. \\[10pt]
  One use case of this is using pointers to modify variables inside other functions. Without pointers, the variables that are inside a function get destroyed when the function ends, and won't affect anything outside the function.
  \begin{note}{}
    Passing by reference is actually just the same thing as passing a pointer, but with simpler notation. Internally, C++ actually uses a pointer to handle this.
  \end{note}
  Always initialise your pointers to \texttt{nullptr} to avoid issues with dereferencing. If you dereference a pointer without first giving it a value, C++ will just go to a random location in memory and do your operation there, instead of where you wanted (and won't throw an error). To avoid this, always initialise pointers to be null so that if you do accidentally dereference without giving it a value, you'll immediately get an error.
  \subsection{Arrays and Pointers}
  \begin{itemize}
    \item Just like any other variable, every array has an address in memory.
    \item However, you don't need to use the \texttt{\&} operator to get an array's address.
    \item Instead, you may just write the array's name (without the square brackets) to get its address.
  \end{itemize}
  In C++, a pointer to an array can be used just as if it were an array itself. For example, if \texttt{ptr} is a pointer to an array, then \texttt{ptr[j]} and \texttt{*(ptr + j)} are identical, meaning ``get the $j$th element from the array''. The reason that the latter works is because incrementing a pointer moves it $n$ bytes forwards, not just one (where $n$ is the size of the type that the pointer is pointing to). For example, incrementing a pointer to an array of \texttt{int}s would move the pointer forwards by $4$ bytes, whereas incrementing a pointer to an array of \texttt{double}s would move the pointer forwards by $8$ bytes.
  \subsection{Pointers with Classes/Structures}
  We can also use pointers to access classes/structures. We can first use \texttt{*} to get the object, and then the dot operator (\texttt{.}) to access member variables/functions. Alternatively, you can use C++'s \texttt{->} operator to access fields. For example, \texttt{(*ptr).function()} and \texttt{ptr->function()} are the same. \\[10pt]
  When you create a class, C++ implicitly creates a \texttt{this} pointer that points to the object that you are working in. \\[10pt]
  Just like you can have pointers to variables, you can also have pointers to functions, using this funky syntax:
  \[
    \underbrace{\texttt{void}}_{\mathclap{\text{return type}}}\texttt{(*f)}\overbrace{\texttt{(int)}}^{\mathclap{\text{parameter type}}}\texttt{;}
  \]
  You can then use the pointer the same way that you would use the function. When initialising the pointer, you may choose to include or omit the \texttt{\&} operator, i.e. \texttt{f = \&squared;} or \texttt{f = cubed;}.
  \section{Dynamic Memory Allocation}
  \textbf{Main Idea.} Often a program won't know how much memory it needs to solve a problem until it's actually running. In these cases, C++ allows you to reserve a chunk of memory on-demand, and gives you a pointer to it. You can use the pointer to access the memory, and then unreserve it afterwards (\texttt{delete} it).
  \subsection{The New and Delete Commands (For Arrays)}
  The \texttt{new} command can be used to allocate an arbitrary amount of memory for an array.
  \begin{enumerate}
    \item First, define a new pointer variable.
    \item Then determine the size of the array you need.
    \item Then use the \texttt{new} command to reserve the memory. Your pointer gets the address of the memory.
    \item Free the memory when you're done.
  \end{enumerate}
  You can see the steps above in the following code:
  \begin{lstlisting}
    int main()
    {
      int* arr; // Step 1
      int size;

      cin >> size; // Step 2

      arr = new int[size]; // Step 3

      arr[0] = 10;
      arr[2] = 75;

      delete [] arr; // Step 4
    }
  \end{lstlisting}
  \begin{note}{}
    Don't forget the square brackets when you're deleting an array!
  \end{note}
  The \texttt{new} command requires two pieces of information:
  \begin{enumerate}
    \item What type of array you want to allocate.
    \item How many slots you want in your array.
  \end{enumerate}
  \begin{note}{}
    Make sure that the pointer's type is the same as the type of array that you are creating!
  \end{note}
  You can also dynamically allocate memory for non-array objects, i.e. \texttt{int* ptr = new int;}. When you delete such a pointer, you don't need the square brackets, and can just do \texttt{delete ptr;}.
  \section{Copy Construction}
  \textbf{Main Idea.} Copy construction is when we create (or construct) a new object by copying the value of an existing object.
  \subsection{Implementing a Copy Constructor}
  Just like we can define a constructor that initialises our object with certain values, we can also define a constructor that takes in a parameter of \emph{an existing object}. For example, the copy constructor of a \texttt{Circ} class might look like:
  \begin{lstlisting}
    Circ(const Circ& old)
    {
      // Copy over the values from the other Circ
      m_x = old.m_x;
      m_y = old.m_y;
      m_rad = old.m_rad;
    }
  \end{lstlisting}
  We can then use this constructor by writing \texttt{Circ c2(c1);}.
  \begin{note}{}
    There's no issue with accessing another \texttt{Circ}'s private member variables, because both objects are of the type \texttt{Circ}.
  \end{note}
  \begin{note}{}
    The parameter to your copy constructor should \emph{always} be \texttt{const}! This forces the coder to not modify the original object. Furthermore, the parameter to your copy constructor \emph{must} be a reference to the class type.
  \end{note}
  C++ allows you to use an alternative syntax to \texttt{Circ b(a);}---you may also write \texttt{Circ b = a;} to mean the same thing. \\[10pt]
  If you don't define your own copy constructor, C++ will provide one for you that just copies all of the member variables from the old instance to the new instance. However, this is just a byte-for-byte copy of the data, otherwise known as a ``shallow copy''.
  \begin{note}{}
    If your object contains any member variables that require dynamically allocated memory, you must write your own copy constructor, otherwise your two objects will both have pointers that point to the same member variable. If you decide to change the variable's value in one object, it will change for the other object.
  \end{note}
  \newpage
  \section{Assignment Operators}
  \textbf{Main Idea.} Just as we need a copy constructor to create a new class variable froman existing one, we often need to create a special assignment funcion to correctly change an existing variable's value to another existing variable. It will automatically be called when we use \texttt{=}.
  \subsection{Implementing an Assignment Operator}
  Just like the copy constructor, we need a special syntax for our assignment operator, as follows:
  \begin{lstlisting}
    Circ &operator=(const Circ& src)
    {
      if (this == &src) // #1
        return *this;
      // Copy over the member variables
      m_x = src.m_x;
      m_y = src.m_y;
      m_rad = src.m_rad;
      return *this; // Don't forget this line!
    }
  \end{lstlisting}
  In line \#1, we need to do an \emph{alias check}---we are essentially checking to see if we are setting a variable equal to itself. If we don't do this, then things will break when we have dynamically allocated member variables (in particular, when deleting the old contents of the object).
  \section{Linked Lists}
  \textbf{Main Idea.} A linked list reserves a new memory block for each item as it's added, and links blocks together with pointers. This allows a linked list to hold a variable number of items, unlike an array.
  \subsection{Why Arrays Suck}
  \begin{itemize}
    \item You can't increase their size.
    \item Inserting/removing an item from the middle of an array is horrible.
  \end{itemize}
  \subsection{Implementing a Linked List}
  We implement linked lists like a scavenger hunt---each item in the list contains a value and a pointer to the next item in the list (contained in a structure, maybe a \texttt{Node}).
  \begin{note}{}
    The \texttt{delete} keyword will delete what a pointer \emph{points to}, not the pointer itself.
  \end{note}
  \subsection{Downsides of Linked Lists}
  \begin{itemize}
    \item Linked lists are \emph{much more} complex than arrays.
    \item To access the $k$th item, you must first traverse through the first $k-1$ items (no instant access).
    \item To add an item at the end of the list, you must first traverse through all $n$ existing nodes.
  \end{itemize}
  \subsection{Doubly Linked Lists}
  To speed up operations performed near the end of a linked list, we can add a pointer to each node that points to the \emph{previous} item, in addition to the pointer that points to the next item. We can then add a tail pointer to allow us to quickly access elements near the end of a linked list. However, this not only makes the code more complicated, but also takes up more space.
  \subsubsection{Linked Lists with a Dummy Node}
  \begin{itemize}
    \item Get rid of the head pointer.
    \item Add a node member variable to your class, called \texttt{dummy}.
    \item Update the member functions to use the dummy node instead of the head pointer (this will simplify your code by removing edge cases).
  \end{itemize}
\end{document}
