\documentclass[class=article, crop=false]{standalone}
% Import packages
\usepackage[margin=1in]{geometry}

\usepackage[many]{tcolorbox}
\usepackage{amssymb, amsthm}
\usepackage{comment}
\usepackage{enumitem}
\usepackage{fancyhdr}
\usepackage{hyperref}
\usepackage{import}
\usepackage{listings}
\usepackage{mathrsfs, mathtools}
\usepackage{pdfpages}
\usepackage{standalone}
\usepackage{transparent}
\usepackage{xcolor}

\usetikzlibrary{decorations.pathreplacing}
\tcbuselibrary{skins}
% Declare math operators
\DeclareMathOperator{\lcm}{lcm}
\DeclareMathOperator{\proj}{proj}
\DeclareMathOperator{\vspan}{span}
\DeclareMathOperator{\im}{im}
\DeclareMathOperator{\range}{range}
\DeclareMathOperator{\Diff}{Diff}
\DeclareMathOperator{\Int}{Int}
\DeclareMathOperator{\fcn}{fcn}
\DeclareMathOperator{\id}{id}
\DeclareMathOperator{\rank}{rank}
\DeclareMathOperator{\tr}{tr}
\DeclareMathOperator{\dive}{div}
\DeclareMathOperator{\row}{row}
\DeclareMathOperator{\col}{col}
% Macros for letters/variables
\renewcommand{\tilde}{\raisebox{0.4ex}{\resizebox{2ex}{!}{\texttildelow}}}
\newcommand{\N}{\ensuremath{\mathbb{N}}}
\newcommand{\Z}{\ensuremath{\mathbb{Z}}}
\newcommand{\Q}{\ensuremath{\mathbb{Q}}}
\newcommand{\R}{\ensuremath{\mathbb{R}}}
\newcommand{\C}{\ensuremath{\mathbb{C}}}
\newcommand{\F}{\ensuremath{\mathbb{F}}}
\newcommand{\M}{\ensuremath{\mathbb{M}}}
\newcommand{\lam}{\ensuremath{\lambda}}
\newcommand{\nab}{\ensuremath{\nabla}}
\newcommand{\eps}{\ensuremath{\varepsilon}}
\newcommand{\es}{\ensuremath{\varnothing}}
% Macros for math symbols
\newcommand{\dx}[1]{\,\mathrm{d}#1}
\newcommand{\inv}{\ensuremath{^{-1}}}
\newcommand{\sm}{\setminus}
\newcommand{\sse}{\subseteq}
\newcommand{\ceq}{\coloneqq}
% Macros for pairs of math symbols
\newcommand{\abs}[1]{\ensuremath{\left\lvert #1 \right\rvert}}
\newcommand{\paren}[1]{\ensuremath{\left( #1 \right)}}
\newcommand{\norm}[1]{\ensuremath{\left\lVert #1\right\rVert}}
\newcommand{\set}[1]{\ensuremath{\left\{#1\right\}}}
\newcommand{\tup}[1]{\ensuremath{\left\langle #1 \right\rangle}}
\newcommand{\floor}[1]{\ensuremath{\left\lfloor #1 \right\rfloor}}
\newcommand{\ceil}[1]{\ensuremath{\left\lceil #1 \right\rceil}}
\newcommand{\eclass}[1]{\ensuremath{\left[ #1 \right]}}

\newcommand{\chapternum}{}
\newcommand{\ex}[1]{\noindent\textbf{Exercise \chapternum.{#1}.}}

\newcommand{\tsub}[1]{\textsubscript{#1}}
\newcommand{\tsup}[1]{\textsuperscript{#1}}

% Include figures
\newcommand{\incfig}[2][1]{%
    \def\svgwidth{#1\columnwidth}
    \import{./figures/}{#2.pdf_tex}
}

\definecolor{problemBackground}{RGB}{212,232,246}

\newenvironment{problem}[1]
  {
    \begin{tcolorbox}[
      boxrule=.5pt,
      titlerule=.5pt,
      sharp corners,
      colback=problemBackground,
      breakable
    ]
    \ifx &#1& \textbf{Problem. }
    \else \textbf{Problem #1.} \fi
  }
  {
    \end{tcolorbox}
  }
\definecolor{exampleBackground}{RGB}{255,249,248}
\definecolor{exampleAccent}{RGB}{158,60,14}
\newenvironment{example}[1]
  {
    \begin{tcolorbox}[
      boxrule=.5pt,
      sharp corners,
      colback=exampleBackground,
      colframe=exampleAccent,
    ]
    \color{exampleAccent}\textbf{Example.} \emph{#1}\color{black}
  }
  {
    \end{tcolorbox}
  }
\definecolor{theoremBackground}{RGB}{234,243,251}
\definecolor{theoremAccent}{RGB}{0,116,183}
\newenvironment{theorem}[1]
  {
    \begin{tcolorbox}[
      boxrule=.5pt,
      titlerule=.5pt,
      sharp corners,
      colback=theoremBackground,
      colframe=theoremAccent,
      breakable
    ]
      \color{theoremAccent}\textbf{Theorem --- }\emph{#1}\\\color{black}
  }
  {
    \end{tcolorbox}
  }
\definecolor{noteBackground}{RGB}{244,249,244}
\definecolor{noteAccent}{RGB}{34,139,34}
\newenvironment{note}[1]
  {
  \begin{tcolorbox}[
    enhanced,
    boxrule=0pt,
    frame hidden,
    sharp corners,
    colback=noteBackground,
    borderline west={3pt}{-1.5pt}{noteAccent},
    breakable
    ]
    \ifx &#1& \color{noteAccent}\textbf{Note. }\color{black}
    \else \color{noteAccent}\textbf{Note (#1). }\color{black} \fi
    }
    {
  \end{tcolorbox}
  }
\definecolor{lemmaBackground}{RGB}{255,247,234}
\definecolor{lemmaAccent}{RGB}{255,153,0}
\newenvironment{lemma}[1]
  {
  \begin{tcolorbox}[
    enhanced,
    boxrule=0pt,
    frame hidden,
    sharp corners,
    colback=lemmaBackground,
    borderline west={3pt}{-1.5pt}{lemmaAccent},
    breakable
    ]
    \ifx &#1& \color{lemmaAccent}\textbf{Lemma. }\color{black}
    \else \color{lemmaAccent}\textbf{Lemma #1. }\color{black} \fi
    }
    {
  \end{tcolorbox}
  }
\definecolor{definitionBackground}{RGB}{246,246,246}
\newenvironment{definition}[1]
  {
    \begin{tcolorbox}[
      enhanced,
      boxrule=0pt,
      frame hidden,
      sharp corners,
      colback=definitionBackground,
      borderline west={3pt}{-1.5pt}{black},
      breakable
    ]
    \textbf{Definition. }\emph{#1}\\
  }
  {
    \end{tcolorbox}
  }

\newenvironment{amatrix}[2]{
    \left[
      \begin{array}{*{#1}{c}|*{#2}c}
  }
  {
      \end{array}
    \right]
  }
\definecolor{codeBackground}{RGB}{253,246,225}
\definecolor{dkgreen}{rgb}{0,0.6,0}
\definecolor{gray}{rgb}{0.5,0.5,0.5}
\definecolor{mauve}{rgb}{0.58,0,0.82}
\lstset{
  language=C++,
  aboveskip=3mm,
  belowskip=3mm,
  backgroundcolor=\color{codeBackground},
  showstringspaces=false,
  columns=flexible,
  basicstyle={\small\ttfamily},
  numbers=none,
  numberstyle=\tiny\color{gray},
  keywordstyle=\color{blue},
  commentstyle=\color{dkgreen},
  stringstyle=\color{mauve},
  breaklines=true,
  breakatwhitespace=true,
  tabsize=2
}

\date{\the\year-\the\month-\the\day}
\author{Kyle Chui}


\fancyhf{}
\lhead{Kyle Chui}
\rhead{Page \thepage}
\pagestyle{fancy}
\pagenumbering{gobble}

\title{Chapter 1 Exercises}

\begin{document}
  \maketitle
  \newpage
  \noindent
  \pagenumbering{arabic}
  \renewcommand{\chapternum}{1.B}
  \indent\ex{1} Prove that $-(-v) = v$ for every $v \in V$.
  \begin{proof}
    \begin{align*}
      -(-v) &= -(-v) + 0 \tag{Additive Identity} \\
      &= -(-v) + (-v + v) \tag{Additive Inverse} \\
      &= (-(-v) + -v) + v \tag{Associativity of Addition} \\
      &= 0 + v \tag{Additive Inverse} \\
      &= v \tag{Additive Identity}
    \end{align*}
  \end{proof}
  \ex{2} Suppose $a \in \mathbf F, v \in V$, and $av = 0$. Prove that $a = 0$ or $v = 0$.
  \begin{proof}
    We begin by first proving that the statement is true for cases where $a = 0$, then by proving it true when $a \neq 0$. Suppose that $a = 0$. Then $a = 0$ and the statement is true. Suppose $a \neq 0$. Then
    \begin{align*}
      av &= 0 \\
      av &= 1 \cdot 0 \tag{Multiplicative Identity} \\
      av &= \left(a \cdot a\inv\right) \cdot 0 \tag{Multiplicative Inverse} \\
      av &= a \cdot \left(a\inv \cdot 0\right) \tag{Associativity of Multiplication} \\
      v &= a\inv \cdot 0 \\
      v &= 0
    \end{align*}
  \end{proof}
  \ex{3} Suppose $v, w \in V$. Explain why there exists a unique $x \in V$ such that $v + 3x = w$.
  \begin{proof}
    We will first show that there exists an $x \in V$ such that $v + 3x = w$. Consider the vector $x = \frac13w - \frac13v$. Then
    \begin{align*}
      v + 3x &= v + 3\left(\frac13w - \frac13v\right) \\
      &= v + 3\cdot\frac13w - 3\cdot\frac13v \\
      &= v + w - v \\
      &= v + w + (-v) \\
      &= v + (-v) + w \\
      &= 0 + w \\
      &= w
    \end{align*}
    Now we will show that such an $x$ is unique. Suppose there exists some $x'$ such that $v + 3x' = w$. Then
    \begin{align*}
      v + 3x' &= w \\
      v + 3x' &= 0 + w \\
      v + 3x' &= (v + (-v)) + w \\
      v + 3x' &= v + (-v + w) \\
      3x' &= -v + w \\
      3x' &= 1\cdot(w + (-v)) \\
      3x' &= 3\cdot\frac13\cdot(w + (-v)) \\
      x' &= \frac13\cdot(w + (-v)) \\
      x' &= \frac13w - \frac13v \\
      x' &= x
    \end{align*}
    Thus $x' = x$ which shows that $x$ is unique.
  \end{proof}
  \ex{4} The empty set is not a vector space. The empty set fails to satisfy only one of the requirements listed in 1.19. Which one? \\[5pt]
  The empty set does not have an additive identity 0, so it is not a vector space. \par\vspace{5pt}
  \ex{5} Show that in the definition of a vector space (1.19), the additive inverse condition can be replaced with the condition that
  \[
    0v = 0 \text{ for all } v \in V.
  \]
  \begin{proof}
    ($\Rightarrow$) We will show that $0v = 0$ implies there exists some $w \in V$ such that $v + w = 0$.
    \begin{align*}
      0v &= 0 \\
      (1 + (-1))v &= 0 \\
      v + (-1\cdot v) &= 0
    \end{align*}
    Thus if $0v = 0$ and $w = -1\cdot v$, $v + w = 0$. \\
    ($\Leftarrow$) We will now show that for all $v \in V$, if there exists a $w \in V$ such that $v + w = 0$, then $0v = 0$. Suppose there exists some $w$ such that $0v + w = 0$. Observe that
    \begin{align*}
      0v + w &= (0 + 0)v + w \\
      &= 0v + 0v + w \\
      &= 0v + 0 \tag{$0v + w = 0$} \\
      &= 0v
    \end{align*}
    Thus $0v + w = 0v = 0$.
  \end{proof}
  \ex{6} Let $\infty$ and $-\infty$ denote two distinct objects, neither of which is in $\mathbf R$. Define an addition and scalar multiplication on $\mathbf R \cup \set\infty \cup \set{-\infty}$ as you could guess from the notation. Specifically the sum and product of two real numbers is as usual, and for $t \in \mathbf R$ define
  \[\begin{matrix}
    t\infty = \begin{cases}
      -\infty & \text{if } t < 0, \\
      0 & \text{if } t = 0, \\
      \infty & \text{if } t > 0,
    \end{cases}
    &\hspace{1cm}
    t(-\infty) = \begin{cases}
      \infty & \text{if } t < 0, \\
      0 & \text{if } t = 0, \\
      -\infty & \text{if } t > 0,
    \end{cases}
  \end{matrix}\]
  \[
    t + \infty = \infty + t = \infty,\hspace{1cm} t + (-\infty) = (-\infty) + t = -\infty,
  \]
  \[
    \infty + \infty = \infty, \hspace{1cm} (-\infty) + (-\infty) = -\infty, \hspace{1cm} \infty + (-\infty) = 0.
  \]
  Is $\mathbf R \cup \set\infty \cup \set{-\infty}$ a vector space over $\mathbf R$? Explain.
  \begin{proof}
    It is not a vector space over $\mathbf R$, because addition is not associative for all vectors in $\mathbf R \cup \set\infty \cup \set{-\infty}$. Consider the following:
    \[
      (\infty + \infty) + (-\infty) = \infty + (-\infty) = 0 \neq \infty = \infty + 0 = \infty + (\infty + (-\infty)).\qedhere
    \]
  \end{proof}
  \newpage
  \renewcommand{\chapternum}{1.C}
  \ex{1} For each of the following subsets of $\mathbf F^3$, determine whether it is a subspace of $\mathbf F^3$:
  \begin{enumerate}[label=(\alph*)]
    \item $\set{(x_1, x_2, x_3) \in \mathbf F^3 \mid x_1 + 2x_2 + 3x_3 = 0}$; \\[10pt]
      Yes, it is a subspace. Suppose $u = (u_1, u_2, u_3)$ and $v = (v_1, v_2, v_3)$ are vectors in the set. Then we have the identity when $u_1 = u_2 = u_3 = 0$. Additionally, $u_1 + 2u_2 + 3u_3 = 0$ and $v_1 + 2v_2 + 3v_3 = 0$. Thus $(u_1 + v_1) + 2(u_2 + v_2) + 3(u_3 + v_3) = 0$ and we have closure under addition. Finally we have that $k(u_1) + k(2u_2) + k(3u_3) = 0$, so there is closure under scalar multiplication. Therefore the set is a subspace of $\mathbf F^3$.
    \item $\set{(x_1, x_2, x_3) \in \mathbf F^3 \mid x_1 + 2x_2 + 3x_3 = 4}$; \\[10pt]
      No, this is not a subspace because there is no identity in the subset of $\mathbf F^3$.
    \item $\set{(x_1, x_2, x_3) \in \mathbf F^3 \mid x_1x_2x_3 = 0}$; \\[10pt]
      No, this is not a subspace because there is no closure under addition. Suppose $u = (0, 1, 1)$ and $v = (1, 1, 0)$. The vector $u + v = (1, 2, 1)$ is not in the subset.
    \item $\set{(x_1, x_2, x_3) \in \mathbf F^3 \mid x_1 = 5x_3}$; \\[10pt]
      Yes, this is a subspace. Suppose $u = (5u_3, u_2, u_3)$ and $v = (5v_3, v_2, v_3)$ are vectors in the set. Then we have the identity when $u_2 = u_3 = 0$. Additionally, $u + v = (5(u_3 + v_3), u_2 + v_2, u_3 + v_3)$, which is also in the set so we have closure under addition. Finally, $kv = (5(kv_3), kv_2, kv_3)$ is also in the set so we have closure under scalar multiplication.
  \end{enumerate}
  \ex{2} Verify all of the following:
  \begin{enumerate}[label=(\alph*)]
    \item If $b \in \mathbb F$, then
    \[
      \set{(x_1, x_2, x_3, x_4) \in \mathbb F^4 \mid x_3 = 5x_4 + b}
    \]
    is a subspace of $\mathbf F^4$ if and only if $b = 0$.
    \begin{proof}
      ($\Rightarrow$) Suppose $U = \set{(x_1, x_2, x_3, x_4) \in \mathbb F^4 \mid x_3 = 5x_4 + b}$ is a subspace of $\mathbf F^4$. Then we have closure under addition and scalar multiplication. Thus if $u = (u_1, u_2, u_3, u_4)$ and $v = (v_1, v_2, v_3, v_4)$ we have $u_3 = 5u_4 + b$ and $v_3 = 5v_4 + b$. Let $w = u + v = (u_1 + v_1, u_2 + v_2, u_3 + v_3, u_4 + v_4)$. Observe that
      \begin{align*}
        w_3 &= u_3 + v_3 \\
        &= (5u_4 + b) + (5v_4 + b) \tag{$x_3 = 5x_4 + b$}\\
        &= 5(u_4 + v_4) + 2b.
      \end{align*}
      Because we have closure under addition, $w \in U$ so $w_3 = 5w_4 + b = 5(u_4 + v_4) + b$. Therefore $2b = b$ and $b = 0$. \\
      ($\Leftarrow$) Suppose $b = 0$. Then let $U = \set{(x_1, x_2, x_3, x_4) \in \mathbf F^4 \mid x_3 = 5x_4}$. Suppose $u = (u_1, u_2, u_3, u_4)$, $v = (v_1, v_2, v_3, v_4)$, and $w = u + v = (u_1 + v_1, u_2 + v_2, u_3 + v_3, u_4 + v_4)$. Observe that the identity is in the set when $u_1 = u_2 = u_4 = 0$. Then
      \begin{align*}
        w_3 &= u_3 + v_3 \\
        &= 5u_4 + 5v_4 \tag{$x_3 = 5x_4$}\\
        &= 5(u_4 + v_4) \\
        &= 5w_4.
      \end{align*}
      Thus $w$ is in the set and we have closure under addition. We will now show that the set has closure under scalar multiplication.
      \begin{align*}
        ku &= (ku_1, ku_2, ku_3, ku_4) \\
        &= (ku_1, ku_2, k(5u_4), u_4) \tag{$x_3 = 5x_4$}\\
        &= (ku_1, ku_2, 5(ku_4), ku_4)
      \end{align*}
      Because $(ku)_3 = 5(ku)_4$, we have closure under scalar multiplication and thus the set is a subspace.
    \end{proof}
    \item The set of continuous real-valued functions on the interval $[0, 1]$ is a subspace of $\mathbf R^{[0, 1]}$.
    \begin{proof}
      Let $U$ be the set of functions $f\colon [0, 1] \to \R$. If $g, h \in U$, then $kg \in U$ and $g + h \in U$, for all $k \in \R$. Thus $U$ is closed under addition and scalar multiplication. The identity function $f(x) = 0$ is also in $U$, so $U$ is a subspace of $\mathbf R^{[0, 1]}$.
    \end{proof}
    \item The set of differentiable real-valued functions on $\mathbf R$ is a subspace of $\mathbf R^{\mathbf R}$.
    \begin{proof}
      Let $U$ be the set of differentiable functions $f\colon \R \to \R$. Then let $g, h \in U$. If $g = 0$, then it is the identity for $U$ because $0 + f = f + 0 = f$ for all $f \in U$. Because the sum of differentiable functions is differentiable, $f + g \in U$ and the set is closed under addition. Because the product of a scalar and a differentiable function is still differentiable, for all $k \in \R$, we have $kg \in U$ and thus $U$ is closed under scalar multiplication. Therefore $U$ is a subspace of $\R^\R$.
    \end{proof}
    \item The set of differentiable real-valued functions $f$ on the interval (0, 3) such that $f'(2) = b$ is a subspace of $\mathbf R^{(0, 3)}$ if and only if $b = 0$.
    \begin{proof}
      ($\Rightarrow$) Suppose $U$ is the set of differentiable real-valued functions $f$ on the interval (0, 3) such that $f'(2) = b$, and that $U$ is a subspace of $\mathbf R^{(0, 3)}$. Then for all $k \in \R$, if $g \in U$ then $kg \in U$. Thus $(kg)' = b = kb = kg'$ so $b = 0$. \\
      ($\Leftarrow$) Suppose $U$ is the set of differentiable real-valued functions $f$ on the interval (0, 3) such that $f'(2) = 0$. Let $g, h \in U$ so that $g'(2) = 0$ and $h'(2) = 0$. If $g = 0$, then it is the identity because $0 + f = f + 0 = f$ for all $f \in U$. Because differentiation is a linear operator, for all $k \in \R$, we have $(kg + h)'(2) = kg'(2) + h'(2) = 0$ so $kg + h \in U$. Therefore $U$ is a subspace of $\R^{(0, 3)}$.
    \end{proof}
    \item The set of all sequences of complex numbers with limit 0 is a subspace of $\mathbf C^\infty$.
    \begin{proof}
      Let $U = \set{c \mid c\text{ is a complex sequence with limit 0}}$. Suppose $c, d \in U$ so
      \[
        \lim_{n\to\infty}c_n = \lim_{n\to\infty}d_n = 0.
      \]
      Consider the sequence $e = (0, 0, \ldots)$. Then $e + c = c + e = c$ for all sequences $c \in U$. Thus $e$ is the identity for $U$. Then for all $k \in \C$,
      \begin{align*}
        \lim_{n\to\infty} (kc_n + d_n) &= \lim_{n\to\infty} kc_n + \lim_{n\to\infty} d_n \\
        &= k\lim_{n\to\infty} c_n + \lim_{n\to\infty} d_n \\
        &= k\cdot 0 + 0 \\
        &= 0
      \end{align*}
      Thus $kc + d \in U$ and $U$ is a subspace of $\C^\infty$.
    \end{proof}
  \end{enumerate}
  \ex{3} Show that the set of differentiable real-valued functions $f$ on the interval $(-4, 4)$ such that $f'(-1) = 3f(2)$ is a subspace of $\mathbf R^{(-4, 4)}$.
  \begin{proof}
    Let $U$ be the set of all differentiable real-valued functions $f$ on the interval $(-4, 4)$ such that $f'(-1) = 3f(2)$. Suppose $g, h \in U$ so $g'(-1) = 3g(2)$ and $h'(-1) = 3h(2)$. Consider the function $e = 0$, which satisfies $e'(-1) = 0 = 3e(2)$. Then $e + g = g + e = g$ for all $g \in U$, so $e$ is the identity for $U$. Then for all $k \in \R$,
    \begin{align*}
      (kg + h)'(-1) &= kg'(-1) + h'(-1) \\
      &= k(3g(2)) + 3h(2) \\
      &= 3(kg(2) + h(2)) \\
      &= 3(kg + h)(2)
    \end{align*}
    Thus $kg + h \in U$ and $U$ is a subspace of $\mathbf R^{(-4, 4)}$.
  \end{proof}
  \newpage
  \ex{4} Suppose $b \in \mathbf R$. Show that the set of continuous real-valued functions $f$ on the interval $[0, 1]$ such that $\int_0^1 f = b$ is a subspace of $\mathbf R^{[0, 1]}$ if and only if $b = 0$.
  \begin{proof}
    ($\Rightarrow$) Let $U$ be the set of continuous real-valued functions $f$ on the interval $[0, 1]$ such that $\int_0^1 f = b$ and $U$ is a subspace of $\mathbf R^{[0, 1]}$. Because $U$ is a subspace, for all $k \in \R$ and all $g \in U$, we have $kg \in U$. Thus $\int_0^1 kg = \int_0^1 g = b$. However, $\int_0^1 kg = k\int_0^1 g = kb$ so we have $kb = b$ and thus $b = 0$. \\
    ($\Leftarrow$) Let $U$ be the set of continuous real-valued functions $f$ on the interval $[0, 1]$ such that $\int_0^1 f = b$. Let $b = 0$, so $\int_0^1 f = 0$ for all functions $f \in U$. Observe that the function $f = 0$ is in $U$ and serves as the identity for $U$. Let $g, h \in U$ so $\int_0^1 g = \int_0^1 h = 0$. Then for all $k \in \R$, we have $\int_0^1 (kg + h) = k\int_0^1 g + \int_0^h = 0$. Therefore $kg + h$ is in $U$ and $U$ is a subspace.
  \end{proof}
  \ex{7} Give an example of a nonempty subset $U$ of $\mathbf R^2$ such that $U$ is closed under addition and under taking additive inverses (meaning $-u \in U$ whenever $u \in U$), but $U$ is not a subspace of $\mathbf R^2$. \\[10pt]
  $U = \set{(x, 0) \in \R^2\mid x \in \Z}$. \\[10pt]
  \ex{8} Give an example of a nonempty subset $U$ of $\mathbf R^2$ such that $U$ is closed under scalar multiplication, but $U$ is not a subspace of $\mathbf R^2$. \\[10pt]
  $U = \set{(x, 0) \in \R^2\mid x \in \R} \cup \set{(0, y) \in \R^2\mid y \in \R}$. \\[10pt]
  \ex{9} A function $f\colon \mathbf R \to \mathbf R$ is called \textbf{\emph{periodic}} if there exists a positive number $p$ such that $f(x) = f(x + p)$ for all $x \in \mathbf R$. Is the set of periodic functions from $\mathbf R$ to $\mathbf R$ a subspace of $\mathbf R^{\mathbf R}$? Explain.
  \begin{proof}
    No, it is not a subspace. The sum of two periodic functions is not periodic if either of the periods is irrational. Thus the set of all periodic functions is not closed under addition, and is not a subspace.
  \end{proof}
  \ex{12} Prove that the union of two subspaces of $V$ is a subspace of $V$ if and only if one of the subspaces is contained in the other.
  \begin{proof}
    ($\Rightarrow$) Suppose $V_1, V_2$ are subspaces of $V$ and $V_1 \cup V_2$ is also a subspace of $V$. Let $v_1 \in V_1$ and $v_2 \in V_2$. Because both $v_1$ and $v_2$ are in $V_1 \cup V_2$, their sum must be in $V_1 \cup V_2$. Without loss of generality, suppose $v_1 + v_2 \in V_1$. Then $v_1 + v_2 + (-v_1) = v_2 \in V_1$ by closure under addition. Thus every element $v_2 \in V_2$ is also in $V_1$, so $V_2$ is contained in $V_1$. \\
    ($\Leftarrow$) Suppose $V_1, V_2$ are subspaces of $V$ such that one is contained within the other. Without loss of generality, suppose $V_1 \subseteq V_2$. Then $V_1 \cup V_2 = V_2$ so the union is also a subspace.
  \end{proof}
  \ex{16} Is the operation of addition on the subspaces of $V$ commutative? In other words, if $U$ and $W$ are subspaces of $V$, is $U + W = W + U$?
  \begin{proof}
    \begin{align*}
      U + W &= \set{u + w\mid u \in U, w \in W} \\
      &= \set{w + u\mid u \in U, w \in W} \\
      &= W + U \qedhere
    \end{align*}
  \end{proof}
  \ex{23} Prove or give a counterexample: if $U_1, U_2, W$ are subspaces of $V$ such that
  \[
    V = U_1\oplus W \quad\text{and}\quad V = U_2\oplus W,
  \]
  then $U_1 = U_2$.
  \begin{proof}
    For every element $v \in V$, there is only one way to express it as a sum of elements from $U_1$ and $W$. Let $v = u_1 + w$, where $u_1 \in U_1$ and $w \in W$. However, we also have $v = u_2 + w$, where $u_2 \in U_2$. Because $u_1, u_2$, and $w$ are unique for all $v$, $u_2 + w = u_1 + w$. Therefore $u_1 = u_2$ for all $v \in V$ and $U_1 = U_2$.
  \end{proof}
\end{document}
