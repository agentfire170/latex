% arara: pdflatex
\documentclass[class=article, crop=false]{standalone}
% Import packages
\usepackage[margin=1in]{geometry}

\usepackage[many]{tcolorbox}
\usepackage{amssymb, amsthm}
\usepackage{comment}
\usepackage{enumitem}
\usepackage{fancyhdr}
\usepackage{hyperref}
\usepackage{import}
\usepackage{listings}
\usepackage{mathrsfs, mathtools}
\usepackage{pdfpages}
\usepackage{standalone}
\usepackage{transparent}
\usepackage{xcolor}

\usetikzlibrary{decorations.pathreplacing}
\tcbuselibrary{skins}
% Declare math operators
\DeclareMathOperator{\lcm}{lcm}
\DeclareMathOperator{\proj}{proj}
\DeclareMathOperator{\vspan}{span}
\DeclareMathOperator{\im}{im}
\DeclareMathOperator{\range}{range}
\DeclareMathOperator{\Diff}{Diff}
\DeclareMathOperator{\Int}{Int}
\DeclareMathOperator{\fcn}{fcn}
\DeclareMathOperator{\id}{id}
\DeclareMathOperator{\rank}{rank}
\DeclareMathOperator{\tr}{tr}
\DeclareMathOperator{\dive}{div}
\DeclareMathOperator{\row}{row}
\DeclareMathOperator{\col}{col}
% Macros for letters/variables
\renewcommand{\tilde}{\raisebox{0.4ex}{\resizebox{2ex}{!}{\texttildelow}}}
\newcommand{\N}{\ensuremath{\mathbb{N}}}
\newcommand{\Z}{\ensuremath{\mathbb{Z}}}
\newcommand{\Q}{\ensuremath{\mathbb{Q}}}
\newcommand{\R}{\ensuremath{\mathbb{R}}}
\newcommand{\C}{\ensuremath{\mathbb{C}}}
\newcommand{\F}{\ensuremath{\mathbb{F}}}
\newcommand{\M}{\ensuremath{\mathbb{M}}}
\newcommand{\lam}{\ensuremath{\lambda}}
\newcommand{\nab}{\ensuremath{\nabla}}
\newcommand{\eps}{\ensuremath{\varepsilon}}
\newcommand{\es}{\ensuremath{\varnothing}}
% Macros for math symbols
\newcommand{\dx}[1]{\,\mathrm{d}#1}
\newcommand{\inv}{\ensuremath{^{-1}}}
\newcommand{\sm}{\setminus}
\newcommand{\sse}{\subseteq}
\newcommand{\ceq}{\coloneqq}
% Macros for pairs of math symbols
\newcommand{\abs}[1]{\ensuremath{\left\lvert #1 \right\rvert}}
\newcommand{\paren}[1]{\ensuremath{\left( #1 \right)}}
\newcommand{\norm}[1]{\ensuremath{\left\lVert #1\right\rVert}}
\newcommand{\set}[1]{\ensuremath{\left\{#1\right\}}}
\newcommand{\tup}[1]{\ensuremath{\left\langle #1 \right\rangle}}
\newcommand{\floor}[1]{\ensuremath{\left\lfloor #1 \right\rfloor}}
\newcommand{\ceil}[1]{\ensuremath{\left\lceil #1 \right\rceil}}
\newcommand{\eclass}[1]{\ensuremath{\left[ #1 \right]}}

\newcommand{\chapternum}{}
\newcommand{\ex}[1]{\noindent\textbf{Exercise \chapternum.{#1}.}}

\newcommand{\tsub}[1]{\textsubscript{#1}}
\newcommand{\tsup}[1]{\textsuperscript{#1}}

% Include figures
\newcommand{\incfig}[2][1]{%
    \def\svgwidth{#1\columnwidth}
    \import{./figures/}{#2.pdf_tex}
}

\definecolor{problemBackground}{RGB}{212,232,246}

\newenvironment{problem}[1]
  {
    \begin{tcolorbox}[
      boxrule=.5pt,
      titlerule=.5pt,
      sharp corners,
      colback=problemBackground,
      breakable
    ]
    \ifx &#1& \textbf{Problem. }
    \else \textbf{Problem #1.} \fi
  }
  {
    \end{tcolorbox}
  }
\definecolor{exampleBackground}{RGB}{255,249,248}
\definecolor{exampleAccent}{RGB}{158,60,14}
\newenvironment{example}[1]
  {
    \begin{tcolorbox}[
      boxrule=.5pt,
      sharp corners,
      colback=exampleBackground,
      colframe=exampleAccent,
    ]
    \color{exampleAccent}\textbf{Example.} \emph{#1}\color{black}
  }
  {
    \end{tcolorbox}
  }
\definecolor{theoremBackground}{RGB}{234,243,251}
\definecolor{theoremAccent}{RGB}{0,116,183}
\newenvironment{theorem}[1]
  {
    \begin{tcolorbox}[
      boxrule=.5pt,
      titlerule=.5pt,
      sharp corners,
      colback=theoremBackground,
      colframe=theoremAccent,
      breakable
    ]
      \color{theoremAccent}\textbf{Theorem --- }\emph{#1}\\\color{black}
  }
  {
    \end{tcolorbox}
  }
\definecolor{noteBackground}{RGB}{244,249,244}
\definecolor{noteAccent}{RGB}{34,139,34}
\newenvironment{note}[1]
  {
  \begin{tcolorbox}[
    enhanced,
    boxrule=0pt,
    frame hidden,
    sharp corners,
    colback=noteBackground,
    borderline west={3pt}{-1.5pt}{noteAccent},
    breakable
    ]
    \ifx &#1& \color{noteAccent}\textbf{Note. }\color{black}
    \else \color{noteAccent}\textbf{Note (#1). }\color{black} \fi
    }
    {
  \end{tcolorbox}
  }
\definecolor{lemmaBackground}{RGB}{255,247,234}
\definecolor{lemmaAccent}{RGB}{255,153,0}
\newenvironment{lemma}[1]
  {
  \begin{tcolorbox}[
    enhanced,
    boxrule=0pt,
    frame hidden,
    sharp corners,
    colback=lemmaBackground,
    borderline west={3pt}{-1.5pt}{lemmaAccent},
    breakable
    ]
    \ifx &#1& \color{lemmaAccent}\textbf{Lemma. }\color{black}
    \else \color{lemmaAccent}\textbf{Lemma #1. }\color{black} \fi
    }
    {
  \end{tcolorbox}
  }
\definecolor{definitionBackground}{RGB}{246,246,246}
\newenvironment{definition}[1]
  {
    \begin{tcolorbox}[
      enhanced,
      boxrule=0pt,
      frame hidden,
      sharp corners,
      colback=definitionBackground,
      borderline west={3pt}{-1.5pt}{black},
      breakable
    ]
    \textbf{Definition. }\emph{#1}\\
  }
  {
    \end{tcolorbox}
  }

\newenvironment{amatrix}[2]{
    \left[
      \begin{array}{*{#1}{c}|*{#2}c}
  }
  {
      \end{array}
    \right]
  }
\definecolor{codeBackground}{RGB}{253,246,225}
\definecolor{dkgreen}{rgb}{0,0.6,0}
\definecolor{gray}{rgb}{0.5,0.5,0.5}
\definecolor{mauve}{rgb}{0.58,0,0.82}
\lstset{
  language=C++,
  aboveskip=3mm,
  belowskip=3mm,
  backgroundcolor=\color{codeBackground},
  showstringspaces=false,
  columns=flexible,
  basicstyle={\small\ttfamily},
  numbers=none,
  numberstyle=\tiny\color{gray},
  keywordstyle=\color{blue},
  commentstyle=\color{dkgreen},
  stringstyle=\color{mauve},
  breaklines=true,
  breakatwhitespace=true,
  tabsize=2
}

\date{\the\year-\the\month-\the\day}
\author{Kyle Chui}


\fancyhf{}
\lhead{Kyle Chui}
\rhead{Page \thepage}
\pagestyle{fancy}

\begin{document}
  \subsection{Equivalence Relations}
  \begin{definition}{Equivalence relation}
    A relation $R$ on  aset $X$ is an \emph{equivalence relation} if it is reflexive, symmetric, and transitive.
  \end{definition}
  \begin{note}{}
    An equivalence relation gives us a notion of two different elements in a set being ``the same".
    \begin{itemize}
      \item Reflexive: Everything is ``the same" as itself
      \item Symmetric: If $x$ is ``the same" as $y$, then $y$ is ``the same" as $x$
      \item Transitive: If $x$ is ``the same" as $y$, and $y$ is ``the same" as $z$, then $x$ is ``the same" as $z$
    \end{itemize}
  \end{note}
  \begin{example}{Equivalence Relations}
    \begin{enumerate}[label=(\alph*)]
      \item The relation $E$ on the integers where $xEy$ if $x-y$ is even.
      \begin{itemize}
        \item Reflexive: For all $x\in\Z$, $x - x = 0$, which is even, so $xEx$
        \item Symmetric: For all $x, y\in\Z$, if $x - y$ is even, so is $-(x-y)=y-x$. Thus if $xEy$, then $yEx$
        \item Transitive: For all $x,y,z\in\Z$, if $x-y$ is even and $y-z$ is even, then their sum, $x-z$, is also even. Thus if $xEy$ and $yEz$, then $xEz$.
      \end{itemize}
      Observe that this relation relates two integers if they have the same parity.
      \item Let $Y$ be any finite set, and $a,b\in Y^*$ (the set of all strings constructed using $Y$). Consider the relation $L$ over $Y^*$ such that $aLb$ if $a$ and $b$ have the same length.
      \item Let $X$ be the set of all animals, with animals $x,y\in X$. Consider the relation $S$ over $X$ such that $xSy$ if $x$ and $y$ are of the same species.
      \item Let $x,y\in\R$. Consider the relation $C$ over $\R$ such that $xCy$ if $x-y$ is an integer.
    \end{enumerate}
  \end{example}
  \begin{definition}{Equivalence Classes}
    If $R$ is an equivalence relation on a set $X$, then for $x\in X$, the \emph{equivalence class} of $X$ is the set (with respect to $R$), denoted by $[x] = [x]_R = \set{y\in X\mid xRy}$.
  \end{definition}
  \begin{example}{Equivalence Classes}
    \begin{enumerate}[label=(\alph*)]
      \item Let $E$ be a relation on $\Z$, where $xEy$ if $x-y$ is even. The equivalence classes for $E$ are $[0]$ (the evens) and $[1]$ (the odds). So, the set of equivalence classes $= \set{[0], [1]}$.
      \item Let $x, y\in\R$, with the relation $C$ over $\R$ defined by $xCy$ if $x - y$ is an integer. The set of equivalence classes $= \set{[x]\mid x\in[0, 1)}$.
    \end{enumerate}
  \end{example}
  If $R$ is an equivalence relation on a set $X$, then:
  \begin{itemize}
    \item For all $x\in X$, if $x\in [y]$ and $x\in [z]$, then $[y] = [z]$.
    \begin{proof}
      Suppose $x\in [y]$ and $x\in [z]$. Let $w\in [y]$. Because $w\in [y]$, we know that $yRw$. We also know that $yRx$ because $x\in[y]$. By symmetry of $R$, we have $wRy$, and by transitivity, we have $wRx$. But $x\in [z]$, so $zRx$, and by symmetry we have $xRw$. By transitivity, $zRw$ so $w\in [z]$. Thus $[y]\sse[z]$. \par
      By a similar argument, we have that $[z]\sse[y]$, so $[y]=[z]$.
    \end{proof}
    \item For any $x\in X$, $x$ is in some equivalence class, $x\in [x]$ by reflexivity. \\
    So, for every $x\in X$, $x$ is in exactly one equivalence class. If $x$ is in another equivalence class $[y]$, then by the above $[x] = [y]$.
  \end{itemize}
  \begin{definition}{Partition}
    For $X$ a set, a \emph{partition} $\mathcal{S}$ of $X$ is a set of nonempty subsets of $X$ such that every element of $X$ is an element of exactly one of the subsets. In other words, for all $A, B\in \mathcal{S}$
    \begin{itemize}
      \item $A, B\sse X$
      \item $A, B\neq\es$
      \item If $A\cap B \neq\es$ then $A = B$
      \item For all $x\in X$, there exists exactly one $A\in \mathcal{S}$ such that $x\in A$
    \end{itemize}
  \end{definition}
  \begin{note}{}
    We showed that if $R$ is an equivalence relation on $X$ then $\set{[x]_R\mid x\in X}$ is a partition of $X$.
  \end{note}
  \begin{theorem}{Equivalence Relations and Partitions}
    For $X$ a set, there is a bijection $F\colon \text{Set of equivalence relations on $X$}\to \text{Set of partitions of $X$}$, defined by
    \[
      F(E) = \set{[x]_E\mid x\in X},
    \]
    the inverse function $F\inv$ sends a partition $\mathcal{S}$ to the equivalence relation $F\inv(\mathcal{S})$ defined by $xF\inv(\mathcal{S})y$ if and only if $x$ and $y$ are in the same element of $\mathcal{S}$ (in the same equivalence class of $E$).
  \end{theorem}
\end{document}
