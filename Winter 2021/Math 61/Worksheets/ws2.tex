\documentclass[class=article, crop=false]{standalone}
% Import packages
\usepackage[margin=1in]{geometry}

\usepackage[many]{tcolorbox}
\usepackage{amssymb, amsthm}
\usepackage{comment}
\usepackage{enumitem}
\usepackage{fancyhdr}
\usepackage{hyperref}
\usepackage{import}
\usepackage{listings}
\usepackage{mathrsfs, mathtools}
\usepackage{pdfpages}
\usepackage{standalone}
\usepackage{transparent}
\usepackage{xcolor}

\usetikzlibrary{decorations.pathreplacing}
\tcbuselibrary{skins}
% Declare math operators
\DeclareMathOperator{\lcm}{lcm}
\DeclareMathOperator{\proj}{proj}
\DeclareMathOperator{\vspan}{span}
\DeclareMathOperator{\im}{im}
\DeclareMathOperator{\range}{range}
\DeclareMathOperator{\Diff}{Diff}
\DeclareMathOperator{\Int}{Int}
\DeclareMathOperator{\fcn}{fcn}
\DeclareMathOperator{\id}{id}
\DeclareMathOperator{\rank}{rank}
\DeclareMathOperator{\tr}{tr}
\DeclareMathOperator{\dive}{div}
\DeclareMathOperator{\row}{row}
\DeclareMathOperator{\col}{col}
% Macros for letters/variables
\renewcommand{\tilde}{\raisebox{0.4ex}{\resizebox{2ex}{!}{\texttildelow}}}
\newcommand{\N}{\ensuremath{\mathbb{N}}}
\newcommand{\Z}{\ensuremath{\mathbb{Z}}}
\newcommand{\Q}{\ensuremath{\mathbb{Q}}}
\newcommand{\R}{\ensuremath{\mathbb{R}}}
\newcommand{\C}{\ensuremath{\mathbb{C}}}
\newcommand{\F}{\ensuremath{\mathbb{F}}}
\newcommand{\M}{\ensuremath{\mathbb{M}}}
\newcommand{\lam}{\ensuremath{\lambda}}
\newcommand{\nab}{\ensuremath{\nabla}}
\newcommand{\eps}{\ensuremath{\varepsilon}}
\newcommand{\es}{\ensuremath{\varnothing}}
% Macros for math symbols
\newcommand{\dx}[1]{\,\mathrm{d}#1}
\newcommand{\inv}{\ensuremath{^{-1}}}
\newcommand{\sm}{\setminus}
\newcommand{\sse}{\subseteq}
\newcommand{\ceq}{\coloneqq}
% Macros for pairs of math symbols
\newcommand{\abs}[1]{\ensuremath{\left\lvert #1 \right\rvert}}
\newcommand{\paren}[1]{\ensuremath{\left( #1 \right)}}
\newcommand{\norm}[1]{\ensuremath{\left\lVert #1\right\rVert}}
\newcommand{\set}[1]{\ensuremath{\left\{#1\right\}}}
\newcommand{\tup}[1]{\ensuremath{\left\langle #1 \right\rangle}}
\newcommand{\floor}[1]{\ensuremath{\left\lfloor #1 \right\rfloor}}
\newcommand{\ceil}[1]{\ensuremath{\left\lceil #1 \right\rceil}}
\newcommand{\eclass}[1]{\ensuremath{\left[ #1 \right]}}

\newcommand{\chapternum}{}
\newcommand{\ex}[1]{\noindent\textbf{Exercise \chapternum.{#1}.}}

\newcommand{\tsub}[1]{\textsubscript{#1}}
\newcommand{\tsup}[1]{\textsuperscript{#1}}

% Include figures
\newcommand{\incfig}[2][1]{%
    \def\svgwidth{#1\columnwidth}
    \import{./figures/}{#2.pdf_tex}
}

\definecolor{problemBackground}{RGB}{212,232,246}

\newenvironment{problem}[1]
  {
    \begin{tcolorbox}[
      boxrule=.5pt,
      titlerule=.5pt,
      sharp corners,
      colback=problemBackground,
      breakable
    ]
    \ifx &#1& \textbf{Problem. }
    \else \textbf{Problem #1.} \fi
  }
  {
    \end{tcolorbox}
  }
\definecolor{exampleBackground}{RGB}{255,249,248}
\definecolor{exampleAccent}{RGB}{158,60,14}
\newenvironment{example}[1]
  {
    \begin{tcolorbox}[
      boxrule=.5pt,
      sharp corners,
      colback=exampleBackground,
      colframe=exampleAccent,
    ]
    \color{exampleAccent}\textbf{Example.} \emph{#1}\color{black}
  }
  {
    \end{tcolorbox}
  }
\definecolor{theoremBackground}{RGB}{234,243,251}
\definecolor{theoremAccent}{RGB}{0,116,183}
\newenvironment{theorem}[1]
  {
    \begin{tcolorbox}[
      boxrule=.5pt,
      titlerule=.5pt,
      sharp corners,
      colback=theoremBackground,
      colframe=theoremAccent,
      breakable
    ]
      \color{theoremAccent}\textbf{Theorem --- }\emph{#1}\\\color{black}
  }
  {
    \end{tcolorbox}
  }
\definecolor{noteBackground}{RGB}{244,249,244}
\definecolor{noteAccent}{RGB}{34,139,34}
\newenvironment{note}[1]
  {
  \begin{tcolorbox}[
    enhanced,
    boxrule=0pt,
    frame hidden,
    sharp corners,
    colback=noteBackground,
    borderline west={3pt}{-1.5pt}{noteAccent},
    breakable
    ]
    \ifx &#1& \color{noteAccent}\textbf{Note. }\color{black}
    \else \color{noteAccent}\textbf{Note (#1). }\color{black} \fi
    }
    {
  \end{tcolorbox}
  }
\definecolor{lemmaBackground}{RGB}{255,247,234}
\definecolor{lemmaAccent}{RGB}{255,153,0}
\newenvironment{lemma}[1]
  {
  \begin{tcolorbox}[
    enhanced,
    boxrule=0pt,
    frame hidden,
    sharp corners,
    colback=lemmaBackground,
    borderline west={3pt}{-1.5pt}{lemmaAccent},
    breakable
    ]
    \ifx &#1& \color{lemmaAccent}\textbf{Lemma. }\color{black}
    \else \color{lemmaAccent}\textbf{Lemma #1. }\color{black} \fi
    }
    {
  \end{tcolorbox}
  }
\definecolor{definitionBackground}{RGB}{246,246,246}
\newenvironment{definition}[1]
  {
    \begin{tcolorbox}[
      enhanced,
      boxrule=0pt,
      frame hidden,
      sharp corners,
      colback=definitionBackground,
      borderline west={3pt}{-1.5pt}{black},
      breakable
    ]
    \textbf{Definition. }\emph{#1}\\
  }
  {
    \end{tcolorbox}
  }

\newenvironment{amatrix}[2]{
    \left[
      \begin{array}{*{#1}{c}|*{#2}c}
  }
  {
      \end{array}
    \right]
  }
\definecolor{codeBackground}{RGB}{253,246,225}
\definecolor{dkgreen}{rgb}{0,0.6,0}
\definecolor{gray}{rgb}{0.5,0.5,0.5}
\definecolor{mauve}{rgb}{0.58,0,0.82}
\lstset{
  language=C++,
  aboveskip=3mm,
  belowskip=3mm,
  backgroundcolor=\color{codeBackground},
  showstringspaces=false,
  columns=flexible,
  basicstyle={\small\ttfamily},
  numbers=none,
  numberstyle=\tiny\color{gray},
  keywordstyle=\color{blue},
  commentstyle=\color{dkgreen},
  stringstyle=\color{mauve},
  breaklines=true,
  breakatwhitespace=true,
  tabsize=2
}

\date{\the\year-\the\month-\the\day}
\author{Kyle Chui}


\fancyhf{}
\lhead{Math 61 \\Worksheet 2}
\rhead{Kyle Chui \\Page \thepage}
\pagestyle{fancy}
\pagenumbering{gobble}

\title{Worksheet 2}

\begin{document}
  \maketitle
  \newpage
  \pagenumbering{arabic}
  \begin{problem}{1}
    Let $\set{0, 1}^X$ denote the set of functions $X\to2$ for some set $X$.\par
    Define a function $F$ from $\mathcal{P}(X)$ to $\set{0, 1}^X$ by for $A \in \mathcal{P}(X)$, $F(A)$ is the function $X\to \set{0, 1}^X$ defined by $F(A)(x) = \begin{cases}0 & x\in A \\ 1 & x\notin A\end{cases}$.
    \begin{enumerate}[label=(\alph*)]
      \item List the elements of the set $\set{0, 1}^{\set{a, b}}$.
      \item Let $X = \set{a, b, c}$. Compute $F(\set{b, c})$, $F(\set{a})$, and $F(\set{a, b, c})$. (Remember all the outputs are functions $X\to \set{0, 1}$).
      \item Again let $X = \set{a, b, c}$. Let $g\colon X\to \set{0, 1}$ be the function defined by $g(a) = 1$, $g(b) = 1$, $g(c) = 1$. Find a subset $A$ of $X$ so that $F(A) = g$.
      \item Show that for any set $X$ the function $F\colon \mathcal{P}(X)\to \set{0, 1}^X$ is a bijection (is injective and surjective).
      \item Use this bijection to give another proof that if $X$ is a finite set then $\abs{\mathcal{P}(X)}=2^{\abs{X}}$.
    \end{enumerate}
  \end{problem}
  \begin{enumerate}[label=(\alph*)]
    \item The elements of the set $\set{0, 1}^{\set{a, b}}$ are:
    \[
      \set{(a,0),(b,0)}, \set{(a,1),(b,0)}, \set{(a,0),(b,1)}, \set{(a,1),(b,1)}.
    \]
    \item
    \begin{align*}
      F(\set{b, c}) &= \set{(a, 1), (b, 0), (c, 0)} \\
      F(\set{a}) &= \set{(a, 0), (b, 1), (c, 1)} \\
      F(\set{a, b, c}) &= \set{(a, 0), (b, 0), (c, 0)}
    \end{align*}
    \item Because all elements of $X$ map to 1, we know that none of the elements of $X$ are in $A$. Thus $A = \es$.
    \item
    \begin{proof}
      We will first show that $F$ is injective. Let $A \neq B$. Without loss of generality, there exists some $x\in A$ such that $x \notin B$. Because $F$ is a function, every element in the domain must get mapped to something, so either $(x, 0)$ or $(x, 1)$ is in $F(A)$, but not in $F(B)$. Therefore $F(A) \neq F(B)$ and $F$ is injective. \par
      We will now show that $F$ is surjective. Observe that for every function $f\in \set{0, 1}^{X}$, the set
      \[
        S = \set{x\in X\mid (x, 0)\in f}
      \]
      maps to $f$. Thus $F$ is a bijection.
    \end{proof}
    \item
    \begin{proof}
      Observe that $\abs{\set{0, 1}^X} = 2^X$, because every element in $X$ can map to one of two elements, 0 or 1. We will show that a bijection between finite sets implies that they are of the same cardinality.
    \end{proof}
  \end{enumerate}
  \newpage
  \begin{problem}{2}
    Let $\N_{>1}$ be the set of natural numbers that are bigger than one. For $i \geq 2$, set $X_i = \set{ik\mid k \in \N_{>1}}$. Describe $\N_{>1} \sm \bigpar{\bigcup_{i=2}^\infty X_i}$.
  \end{problem}
  Observe that $X_2$ is the set of all even numbers, $X_3$ the set of all multiples of three, et cetera. Thus the union of all such sets is $\set{x\mid x \text{ is an integer larger than one}}$. Therefore the indicated set is just the empty set.
  \begin{problem}{3}
    Let $f\colon X\to Y$ be a function. Show that $f$ is onto if and only if for every onto function $g\colon Y\to Z$ the function $g\circ f$ is onto.
  \end{problem}
  \begin{proof}
    (\Rightarrow) Suppose $f, g$ are surjective functions. Then for all $z\in Z$, there exists some $y\in Y$ such that $g(y) = z$. Furthermore, for all $y\in Y$ there exists some $x\in X$ such that $f(x) = y$. Thus, for all $z\in Z$, we have $g(f(x)) = (g\circ f)(x) = z$, so $g\circ f$ is surjective. \par
    (\Leftarrow) Suppose $g, g\circ f$ are surjective functions. Then for all $z\in Z$, there exists some $x\in X$ such that $(g\circ f)(x) = z$.
  \end{proof}
\end{document}
