\documentclass{article}
% Import packages
\usepackage[margin=1in]{geometry}

\usepackage[many]{tcolorbox}
\usepackage{amssymb, amsthm}
\usepackage{comment}
\usepackage{enumitem}
\usepackage{fancyhdr}
\usepackage{hyperref}
\usepackage{import}
\usepackage{listings}
\usepackage{mathrsfs, mathtools}
\usepackage{pdfpages}
\usepackage{standalone}
\usepackage{transparent}
\usepackage{xcolor}

\usetikzlibrary{decorations.pathreplacing}
\tcbuselibrary{skins}
% Declare math operators
\DeclareMathOperator{\lcm}{lcm}
\DeclareMathOperator{\proj}{proj}
\DeclareMathOperator{\vspan}{span}
\DeclareMathOperator{\im}{im}
\DeclareMathOperator{\range}{range}
\DeclareMathOperator{\Diff}{Diff}
\DeclareMathOperator{\Int}{Int}
\DeclareMathOperator{\fcn}{fcn}
\DeclareMathOperator{\id}{id}
\DeclareMathOperator{\rank}{rank}
\DeclareMathOperator{\tr}{tr}
\DeclareMathOperator{\dive}{div}
\DeclareMathOperator{\row}{row}
\DeclareMathOperator{\col}{col}
% Macros for letters/variables
\renewcommand{\tilde}{\raisebox{0.4ex}{\resizebox{2ex}{!}{\texttildelow}}}
\newcommand{\N}{\ensuremath{\mathbb{N}}}
\newcommand{\Z}{\ensuremath{\mathbb{Z}}}
\newcommand{\Q}{\ensuremath{\mathbb{Q}}}
\newcommand{\R}{\ensuremath{\mathbb{R}}}
\newcommand{\C}{\ensuremath{\mathbb{C}}}
\newcommand{\F}{\ensuremath{\mathbb{F}}}
\newcommand{\M}{\ensuremath{\mathbb{M}}}
\newcommand{\lam}{\ensuremath{\lambda}}
\newcommand{\nab}{\ensuremath{\nabla}}
\newcommand{\eps}{\ensuremath{\varepsilon}}
\newcommand{\es}{\ensuremath{\varnothing}}
% Macros for math symbols
\newcommand{\dx}[1]{\,\mathrm{d}#1}
\newcommand{\inv}{\ensuremath{^{-1}}}
\newcommand{\sm}{\setminus}
\newcommand{\sse}{\subseteq}
\newcommand{\ceq}{\coloneqq}
% Macros for pairs of math symbols
\newcommand{\abs}[1]{\ensuremath{\left\lvert #1 \right\rvert}}
\newcommand{\paren}[1]{\ensuremath{\left( #1 \right)}}
\newcommand{\norm}[1]{\ensuremath{\left\lVert #1\right\rVert}}
\newcommand{\set}[1]{\ensuremath{\left\{#1\right\}}}
\newcommand{\tup}[1]{\ensuremath{\left\langle #1 \right\rangle}}
\newcommand{\floor}[1]{\ensuremath{\left\lfloor #1 \right\rfloor}}
\newcommand{\ceil}[1]{\ensuremath{\left\lceil #1 \right\rceil}}
\newcommand{\eclass}[1]{\ensuremath{\left[ #1 \right]}}

\newcommand{\chapternum}{}
\newcommand{\ex}[1]{\noindent\textbf{Exercise \chapternum.{#1}.}}

\newcommand{\tsub}[1]{\textsubscript{#1}}
\newcommand{\tsup}[1]{\textsuperscript{#1}}

% Include figures
\newcommand{\incfig}[2][1]{%
    \def\svgwidth{#1\columnwidth}
    \import{./figures/}{#2.pdf_tex}
}

\definecolor{problemBackground}{RGB}{212,232,246}

\newenvironment{problem}[1]
  {
    \begin{tcolorbox}[
      boxrule=.5pt,
      titlerule=.5pt,
      sharp corners,
      colback=problemBackground,
      breakable
    ]
    \ifx &#1& \textbf{Problem. }
    \else \textbf{Problem #1.} \fi
  }
  {
    \end{tcolorbox}
  }
\definecolor{exampleBackground}{RGB}{255,249,248}
\definecolor{exampleAccent}{RGB}{158,60,14}
\newenvironment{example}[1]
  {
    \begin{tcolorbox}[
      boxrule=.5pt,
      sharp corners,
      colback=exampleBackground,
      colframe=exampleAccent,
    ]
    \color{exampleAccent}\textbf{Example.} \emph{#1}\color{black}
  }
  {
    \end{tcolorbox}
  }
\definecolor{theoremBackground}{RGB}{234,243,251}
\definecolor{theoremAccent}{RGB}{0,116,183}
\newenvironment{theorem}[1]
  {
    \begin{tcolorbox}[
      boxrule=.5pt,
      titlerule=.5pt,
      sharp corners,
      colback=theoremBackground,
      colframe=theoremAccent,
      breakable
    ]
      \color{theoremAccent}\textbf{Theorem --- }\emph{#1}\\\color{black}
  }
  {
    \end{tcolorbox}
  }
\definecolor{noteBackground}{RGB}{244,249,244}
\definecolor{noteAccent}{RGB}{34,139,34}
\newenvironment{note}[1]
  {
  \begin{tcolorbox}[
    enhanced,
    boxrule=0pt,
    frame hidden,
    sharp corners,
    colback=noteBackground,
    borderline west={3pt}{-1.5pt}{noteAccent},
    breakable
    ]
    \ifx &#1& \color{noteAccent}\textbf{Note. }\color{black}
    \else \color{noteAccent}\textbf{Note (#1). }\color{black} \fi
    }
    {
  \end{tcolorbox}
  }
\definecolor{lemmaBackground}{RGB}{255,247,234}
\definecolor{lemmaAccent}{RGB}{255,153,0}
\newenvironment{lemma}[1]
  {
  \begin{tcolorbox}[
    enhanced,
    boxrule=0pt,
    frame hidden,
    sharp corners,
    colback=lemmaBackground,
    borderline west={3pt}{-1.5pt}{lemmaAccent},
    breakable
    ]
    \ifx &#1& \color{lemmaAccent}\textbf{Lemma. }\color{black}
    \else \color{lemmaAccent}\textbf{Lemma #1. }\color{black} \fi
    }
    {
  \end{tcolorbox}
  }
\definecolor{definitionBackground}{RGB}{246,246,246}
\newenvironment{definition}[1]
  {
    \begin{tcolorbox}[
      enhanced,
      boxrule=0pt,
      frame hidden,
      sharp corners,
      colback=definitionBackground,
      borderline west={3pt}{-1.5pt}{black},
      breakable
    ]
    \textbf{Definition. }\emph{#1}\\
  }
  {
    \end{tcolorbox}
  }

\newenvironment{amatrix}[2]{
    \left[
      \begin{array}{*{#1}{c}|*{#2}c}
  }
  {
      \end{array}
    \right]
  }
\definecolor{codeBackground}{RGB}{253,246,225}
\definecolor{dkgreen}{rgb}{0,0.6,0}
\definecolor{gray}{rgb}{0.5,0.5,0.5}
\definecolor{mauve}{rgb}{0.58,0,0.82}
\lstset{
  language=C++,
  aboveskip=3mm,
  belowskip=3mm,
  backgroundcolor=\color{codeBackground},
  showstringspaces=false,
  columns=flexible,
  basicstyle={\small\ttfamily},
  numbers=none,
  numberstyle=\tiny\color{gray},
  keywordstyle=\color{blue},
  commentstyle=\color{dkgreen},
  stringstyle=\color{mauve},
  breaklines=true,
  breakatwhitespace=true,
  tabsize=2
}

\date{\the\year-\the\month-\the\day}
\author{Kyle Chui}


\fancyhf{}
\lhead{Math 61 \\Worksheet 3}
\rhead{Kyle Chui \\Page \thepage}
\pagestyle{fancy}
\pagenumbering{gobble}

\title{Worksheet 3}

\begin{document}
  \maketitle
  \newpage
  \pagenumbering{arabic}
  \begin{problem}{1}
    \begin{enumerate}[label=(\alph*)]
      \item Put a relation $E$ on the integers by $xEy$ if $x+y$ is divisible by 2. Show that $E$ is reflexive, symmetric, and transitive. What are all the elements related to 0? What about all the elements related to 1?
      \item Now consider the relation $T$ on the integers defined by $xTy$ if $x + y$ is divisible by 3. Is this relation reflexive? Symmetric? Transitive?
    \end{enumerate}
  \end{problem}
  \begin{enumerate}[label=(\alph*)]
    \item
    \begin{proof}
      Observe that for any integer $x$, $x + x = 2x$ is divisible by 2. Thus $xEx$ and $E$ is reflexive. Suppose $xEy$ for some integers $x$ and $y$. Then for some integer $k$, we have $x+y = 2k = y+x$, so $yEx$ and $E$ is symmetric. Finally, suppose $xEy$ and $yEz$. Then for some integers $m$ and $n$, we have $x + y = 2m$ and $y + z = 2n$. Thus
      \begin{align*}
        (x+y) + (y+z) &= 2m + 2n \\
        x + 2y + z &= 2m + 2n \\
        x + z &= 2m + 2n - 2y \\
        x + z &= 2(m + n - y).
      \end{align*}
      Therefore $xEz$ and $E$ is transitive. All of the elements related to $0$ are the even integers, and all of the elements related to $1$ are the odd integers.
    \end{proof}
    \item
    \begin{proof}
      The relation is not reflexive. Observe that $1 + 1 = 2$ is not divisible by $3$. The relation is symmetric. Suppose $xTy$, so that there exists some integer $k$ such that $x + y = 3k$. Then $y + x = 3k$, so $yEx$ and the relation is symmetric. The relation is not transitive. Observe that $1T2$ because $1 + 2 = 3$ is divisible by $3$, and $2T1$ because $2 + 1 = 3$ is divisible by $3$, but $1 + 1 = 2$ is not divisible by $3$. In other words, $1T2$ and $2T1$ does not imply $1T1$.
    \end{proof}
  \end{enumerate}
  \newpage
  \begin{problem}{2}
    Suppose that $X$ is a set and that we have a function $f\colon X\to \R$. Consider the relation on $X$ defined by $aRb$ if $f(a) \geq f(b)$.
    \begin{enumerate}[label=(\alph*)]
      \item Show that this relation is reflexive and transitive.
      \item What condition on $f$ is necessary for this relation to be antisymmetric?
      \item Describe how by choosing $X$ and $f$ appropriately this relation can give relations on people such as ``$a$ is related to $b$ if $a$ is wealthier than $b$", or ``$c$ is related to $d$ if $c$ is older than $d$".
    \end{enumerate}
  \end{problem}
  \begin{enumerate}[label=(\alph*)]
    \item
    \begin{proof}
      Let $x, y, z\in X$. Observe that for all $x$ that $f(x) \geq f(x)$, so $xRx$ and $R$ is reflexive. Suppose $xRy$ and $yRz$. Then $f(x)\geq f(y)$ and $f(y)\geq f(z)$. Thus $f(x)\geq f(z)$, so $xRz$ and $R$ is transitive.
    \end{proof}
    \item We need $f$ to be injective.
    \begin{proof}
      Suppose $f$ is injective and let $x, y\in X$ such that $xRy$ and $yRx$. Then $f(x)\geq f(y)$ and $f(y)\geq f(x)$, so $f(x) = f(y)$. Because $f$ is injective, we have $x = y$ and thus $R$ is antisymmetric.\par
      Suppose towards a contraposition that $f$ is not injective. Then there exists some $x, y\in X$ such that $f(x) = f(y)$ but $x\neq y$. Thus $xRy$ and $yRx$ and $x\neq y$, so $R$ is not antisymmetric.
    \end{proof}
    \item We may choose $X$ to be the set of people and $f$ to either give a person's wealth or their age to yield the two relations listed.
  \end{enumerate}
  \newpage
  \begin{problem}{3}
    Put a relation $Q$ on $\Z \times (\Z\sm \set{0})$ by $(a, b)Q(c, d)$ if $ad = bc$. Show that $Q$ is reflexive, symmetric, and transitive. What mathematical structure does this remind you of?
  \end{problem}
  \begin{proof}
    Let $x, y, z\in \Z \times (\Z\sm \set{0})$ such that $x = (x_1, x_2)$, $y = (y_1, y_2)$, and $z = (z_1, z_2)$. Then $xQx$ because $x_1x_2 = x_2x_1$ implies $x_1x_2 = x_2x_1$, so $Q$ is reflexive. Now suppose $xQy$, so $x_1 y_2 = x_2 y_1$. Then $x_2 y_1 = x_1 y_2$, so $Q$ is symmetric. Finally, suppose $xQy$ and $yQz$, so $x_1 y_2 = x_2 y_1$ and $y_1 z_2 = y_2 z_1$. Multiplying the two equations, we get
    \[
      x_1 z_2 (y_1 y_2) = x_2 z_1 (y_1 y_2).
    \]
    If $y_1 \neq 0$, then we may divide both sides of the above equation to get $x_1 z_2 = x_2 z_1$. If $y_1 = 0$, then $x_1 y_2 = 0 = y_2 z_1$, so $x_1 = 0 = z_1$. Therefore $x_1 z_2 = 0 = x_2 z_1$. In either case, $Q$ is transitive.
  \end{proof}
  This equivalence relation is the same as the structure of the rational numbers.
  \newpage
  \begin{problem}{4}
    The Fibonacci numbers are the sequence $\set{F_n}^\infty_{n=0}$ defined by $F_0 = 0$, $F_1 = 1$, and for $n > 1$ $F_n = F_{n-1} + F_{n-2}$. Compute the first 10 Fibonacci numbers. Show that $\sum_{i=0}^{k}F_i = F_{k+2}-1$.\par
    Are the Fibonacci numbers increasing, decreasing, nonincreasing, nondecreasing, or none of these? What about the sequence defined by for $k\in\N$ by $s_k = \sum_{i=0}^{k}F_i$?
  \end{problem}
  \[\begin{array}{c|cccccccccc}
    n & 0 & 1 & 2 & 3 & 4 & 5 & 6 & 7 & 8 & 9 \\
    \hline
    F_n & 0 & 1 & 1 & 2 & 3 & 5 & 8 & 13 & 21 & 34
  \end{array}\]
  \begin{proof}
    Observe that for $n = 0$, we have $\sum_{i=0}^{0}F_i = F_0 = 0 = F_2 - 1$. Furthermore, when $n = 1$, we have $\sum_{i=0}^{1}F_i = 0 + 1 = 2 = F_3 - 1$. Suppose that the statement holds for all non-negative integers $m$ between $0$ and $k$, inclusive. Then
    \begin{align*}
      \sum_{i=0}^{k+1}F_i &= \sum_{i=0}^{k}F_i + F_{k+1} \\
      &= F_{k+2} - 1 + F_{k+1} \\
      &= F_{k+3} - 1.
    \end{align*}
    Thus the statement holds for $n = k+1$, so the statement is true for all non-negative integers.
  \end{proof}
  Observe that $F_2 = F_1 > F_0$, and $F_n - F_{n-1} = F_{n-2}\geq 0$ for all $n \geq 2$. Thus $F_n$ is nondecreasing. As for $s_n$, observe that $s_2 > s_1 > s_0$, and $s_n - s_{n-1} = F_n > 0$ for all $n\geq 2$. Thus $s_n$ is increasing.
\end{document}
