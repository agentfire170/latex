\documentclass[class=article, crop=false]{standalone}
% Import packages
\usepackage[margin=1in]{geometry}

\usepackage[many]{tcolorbox}
\usepackage{amssymb, amsthm}
\usepackage{comment}
\usepackage{enumitem}
\usepackage{fancyhdr}
\usepackage{hyperref}
\usepackage{import}
\usepackage{listings}
\usepackage{mathrsfs, mathtools}
\usepackage{pdfpages}
\usepackage{standalone}
\usepackage{transparent}
\usepackage{xcolor}

\usetikzlibrary{decorations.pathreplacing}
\tcbuselibrary{skins}
% Declare math operators
\DeclareMathOperator{\lcm}{lcm}
\DeclareMathOperator{\proj}{proj}
\DeclareMathOperator{\vspan}{span}
\DeclareMathOperator{\im}{im}
\DeclareMathOperator{\range}{range}
\DeclareMathOperator{\Diff}{Diff}
\DeclareMathOperator{\Int}{Int}
\DeclareMathOperator{\fcn}{fcn}
\DeclareMathOperator{\id}{id}
\DeclareMathOperator{\rank}{rank}
\DeclareMathOperator{\tr}{tr}
\DeclareMathOperator{\dive}{div}
\DeclareMathOperator{\row}{row}
\DeclareMathOperator{\col}{col}
% Macros for letters/variables
\renewcommand{\tilde}{\raisebox{0.4ex}{\resizebox{2ex}{!}{\texttildelow}}}
\newcommand{\N}{\ensuremath{\mathbb{N}}}
\newcommand{\Z}{\ensuremath{\mathbb{Z}}}
\newcommand{\Q}{\ensuremath{\mathbb{Q}}}
\newcommand{\R}{\ensuremath{\mathbb{R}}}
\newcommand{\C}{\ensuremath{\mathbb{C}}}
\newcommand{\F}{\ensuremath{\mathbb{F}}}
\newcommand{\M}{\ensuremath{\mathbb{M}}}
\newcommand{\lam}{\ensuremath{\lambda}}
\newcommand{\nab}{\ensuremath{\nabla}}
\newcommand{\eps}{\ensuremath{\varepsilon}}
\newcommand{\es}{\ensuremath{\varnothing}}
% Macros for math symbols
\newcommand{\dx}[1]{\,\mathrm{d}#1}
\newcommand{\inv}{\ensuremath{^{-1}}}
\newcommand{\sm}{\setminus}
\newcommand{\sse}{\subseteq}
\newcommand{\ceq}{\coloneqq}
% Macros for pairs of math symbols
\newcommand{\abs}[1]{\ensuremath{\left\lvert #1 \right\rvert}}
\newcommand{\paren}[1]{\ensuremath{\left( #1 \right)}}
\newcommand{\norm}[1]{\ensuremath{\left\lVert #1\right\rVert}}
\newcommand{\set}[1]{\ensuremath{\left\{#1\right\}}}
\newcommand{\tup}[1]{\ensuremath{\left\langle #1 \right\rangle}}
\newcommand{\floor}[1]{\ensuremath{\left\lfloor #1 \right\rfloor}}
\newcommand{\ceil}[1]{\ensuremath{\left\lceil #1 \right\rceil}}
\newcommand{\eclass}[1]{\ensuremath{\left[ #1 \right]}}

\newcommand{\chapternum}{}
\newcommand{\ex}[1]{\noindent\textbf{Exercise \chapternum.{#1}.}}

\newcommand{\tsub}[1]{\textsubscript{#1}}
\newcommand{\tsup}[1]{\textsuperscript{#1}}

% Include figures
\newcommand{\incfig}[2][1]{%
    \def\svgwidth{#1\columnwidth}
    \import{./figures/}{#2.pdf_tex}
}

\definecolor{problemBackground}{RGB}{212,232,246}

\newenvironment{problem}[1]
  {
    \begin{tcolorbox}[
      boxrule=.5pt,
      titlerule=.5pt,
      sharp corners,
      colback=problemBackground,
      breakable
    ]
    \ifx &#1& \textbf{Problem. }
    \else \textbf{Problem #1.} \fi
  }
  {
    \end{tcolorbox}
  }
\definecolor{exampleBackground}{RGB}{255,249,248}
\definecolor{exampleAccent}{RGB}{158,60,14}
\newenvironment{example}[1]
  {
    \begin{tcolorbox}[
      boxrule=.5pt,
      sharp corners,
      colback=exampleBackground,
      colframe=exampleAccent,
    ]
    \color{exampleAccent}\textbf{Example.} \emph{#1}\color{black}
  }
  {
    \end{tcolorbox}
  }
\definecolor{theoremBackground}{RGB}{234,243,251}
\definecolor{theoremAccent}{RGB}{0,116,183}
\newenvironment{theorem}[1]
  {
    \begin{tcolorbox}[
      boxrule=.5pt,
      titlerule=.5pt,
      sharp corners,
      colback=theoremBackground,
      colframe=theoremAccent,
      breakable
    ]
      \color{theoremAccent}\textbf{Theorem --- }\emph{#1}\\\color{black}
  }
  {
    \end{tcolorbox}
  }
\definecolor{noteBackground}{RGB}{244,249,244}
\definecolor{noteAccent}{RGB}{34,139,34}
\newenvironment{note}[1]
  {
  \begin{tcolorbox}[
    enhanced,
    boxrule=0pt,
    frame hidden,
    sharp corners,
    colback=noteBackground,
    borderline west={3pt}{-1.5pt}{noteAccent},
    breakable
    ]
    \ifx &#1& \color{noteAccent}\textbf{Note. }\color{black}
    \else \color{noteAccent}\textbf{Note (#1). }\color{black} \fi
    }
    {
  \end{tcolorbox}
  }
\definecolor{lemmaBackground}{RGB}{255,247,234}
\definecolor{lemmaAccent}{RGB}{255,153,0}
\newenvironment{lemma}[1]
  {
  \begin{tcolorbox}[
    enhanced,
    boxrule=0pt,
    frame hidden,
    sharp corners,
    colback=lemmaBackground,
    borderline west={3pt}{-1.5pt}{lemmaAccent},
    breakable
    ]
    \ifx &#1& \color{lemmaAccent}\textbf{Lemma. }\color{black}
    \else \color{lemmaAccent}\textbf{Lemma #1. }\color{black} \fi
    }
    {
  \end{tcolorbox}
  }
\definecolor{definitionBackground}{RGB}{246,246,246}
\newenvironment{definition}[1]
  {
    \begin{tcolorbox}[
      enhanced,
      boxrule=0pt,
      frame hidden,
      sharp corners,
      colback=definitionBackground,
      borderline west={3pt}{-1.5pt}{black},
      breakable
    ]
    \textbf{Definition. }\emph{#1}\\
  }
  {
    \end{tcolorbox}
  }

\newenvironment{amatrix}[2]{
    \left[
      \begin{array}{*{#1}{c}|*{#2}c}
  }
  {
      \end{array}
    \right]
  }
\definecolor{codeBackground}{RGB}{253,246,225}
\definecolor{dkgreen}{rgb}{0,0.6,0}
\definecolor{gray}{rgb}{0.5,0.5,0.5}
\definecolor{mauve}{rgb}{0.58,0,0.82}
\lstset{
  language=C++,
  aboveskip=3mm,
  belowskip=3mm,
  backgroundcolor=\color{codeBackground},
  showstringspaces=false,
  columns=flexible,
  basicstyle={\small\ttfamily},
  numbers=none,
  numberstyle=\tiny\color{gray},
  keywordstyle=\color{blue},
  commentstyle=\color{dkgreen},
  stringstyle=\color{mauve},
  breaklines=true,
  breakatwhitespace=true,
  tabsize=2
}

\date{\the\year-\the\month-\the\day}
\author{Kyle Chui}


\fancyhf{}
\lhead{Kyle Chui}
\rhead{Page \thepage}
\pagestyle{fancy}
\pagenumbering{gobble}

\title{Lecture 3 Notes}

\begin{document}
  \maketitle
  \newpage
  \pagenumbering{arabic}
  \section{Sets and Functions}
  \subsection{What are sets?}
  \boxed{\textbf{Definition.}} A \emph{set} is a collection of objects. We usually denote sets using capital letters. Here's an example of a set:
  \[
    X = \set{a, b, c, d} \quad a\in X.
  \]
  We say $a \in X$ means that $a$ is an element of $X$. The order of elements in a set does not matter, and neither do repetitions, so we have:
  \[
    \set{a, b, c, d} = \set{a, c, d, b} = \set{a, a, b, b, c, c, d, d}.
  \]
  The elements of a set don't matter either, so we may have a set like
  \[
    \set{\sqrt{2}, \text{James Cameron (the director)}, \text{James Cameron (the professor)}}
  \]
  Sets can have other sets as elements, for example
  \[
    \set{\set{0, 1}, 0, 1}.
  \]
  The three elements of the above set are: $\set{0, 1}, \set{0}$, and $\set{1}$.
  \subsection{Set-builder notation}
  We use set builder notation to make sets. For example,
  \[
    \text{UCLA} = \set{x : x \text{ is a person who is a student at UCLA}}.
  \]
  Common sets that we use in this class:
  \begin{itemize}
    \item $\N$ is the set of natural numbers. $\set{0, 1, 2, \dotsc}$
    \item $\Z$ is the set of integers. $\set{\dotsc, -2, -1, 0, 1, 2, \dotsc}$
    \item $\Q$ is the set of rationals (fractions)
    \item $\R$ is the set of reals
    \item $\C$ is the set of complex numbers
  \end{itemize}
  Consider the set $E = \set{x\in \Z\mid x=  2k\text{ for some }k\in \Z}$, which is the even integers. The empty set has no items in it, and is denoted $\varnothing = \set{}$. \\
  \boxed{\textbf{Definition.}} Suppose $X, Y$ are sets. We say that $X$ is a subset of $Y$ (we write $X \subseteq Y$) if $a \in X$ implies $a \in Y$. In other words, every element of $X$ is also an element of $Y$. \\
  \boxed{\textbf{Definition.}} For sets $X, Y$ we have that $X = Y$ if they have the same statements. \\
  \boxed{\textbf{Proposition.}} For sets $X, Y$ we have $X = Y$ if and only if $X \subseteq Y$ and $Y \subseteq X$.
  \begin{proof}
    Suppose that $X = Y$. Let $a \in X$. Since $X = Y$, we have $a \in Y$ as well, so $X \subseteq Y$. Similarly, if $b \in Y$ then $b \in X$ because $X = Y$, so $Y \subseteq X$. \par
    We will now show the other direction of the statement. Suppose $X \subseteq Y$ and $Y \subseteq X$. Thus if $a \in X$ then $a \in Y$ and if $b \in Y$ then $b \in X$, so $X$ and $Y$ have the same elements and $X = Y$.
  \end{proof}
  \noindent\boxed{\textbf{Note.}} If $X$ is any set, then $\varnothing \subseteq X$. \\
  The empty set is not an element of every set, although it \emph{could} be an element of a set. For example, $\varnothing \notin \Z$ but $\varnothing \subseteq \Z$.
  \[
    \varnothing \in \set{\varnothing} = \set{\set{}} \tag{Note: $\varnothing \neq \set{\varnothing}$}
  \]
  \newpage
  \begin{center}\[\begin{array}{c|c|c}
    \text{It is raining} & \text{I got wet} & \text{If it is raining, then I got wet} \\
    \hline
    T & T & T \\
    T & F & F \\
    F & T & T \\
    F & F & T
  \end{array}\]\end{center}
  The last two statements are vacuously true, because the first part is false. Think about it in terms of ``lying". If it is not raining and you did not get wet, that does not mean that the person who said ``If it is raining, then I got wet" is lying. Their statement was just not applicable in the situation.
  \subsection{Constructing sets from other sets}
  \begin{enumerate}[label=(\roman*)]
    \item $X\cup Y = \set{a\mid a \in X \text{ or } a \in Y}$
    \item $X\cap Y = \set{a\mid a\in X \text{ and } a \in Y}$
    \item $X\times Y = \set{(a, b)\mid a\in X \text{ and } b\in Y}$
    \item $Y \sm X = \set{y\in Y\mid y\notin X}$
  \end{enumerate}
  \boxed{\textbf{Proposition.}} If $X, Y$ are sets:
  \begin{itemize}
    \item $X\subseteq X\cup Y$ and $Y \subseteq X\cup Y$
    \item $X \cap Y \subseteq X$, $X\cap Y\subseteq Y$
    \item $Y\sm X \subseteq Y$
    \item $X\cap (Y\sm X) = \varnothing$
  \end{itemize}
  \boxed{\textbf{Definition.}} If $X$ is a set, $\mathscr{P}(X)$ (the ``power set" of $X$) is the set of subsets of $X$. \\
  \boxed{\textbf{Proposition.}} If $X$ has $n$ elements, then $\mathscr{P}(X)$ has $2^n$ elements.
\end{document}
