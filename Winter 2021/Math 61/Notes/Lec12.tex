\documentclass[class=article, crop=false]{standalone}
% Import packages
\usepackage[margin=1in]{geometry}

\usepackage[many]{tcolorbox}
\usepackage{amssymb, amsthm}
\usepackage{comment}
\usepackage{enumitem}
\usepackage{fancyhdr}
\usepackage{hyperref}
\usepackage{import}
\usepackage{listings}
\usepackage{mathrsfs, mathtools}
\usepackage{pdfpages}
\usepackage{standalone}
\usepackage{transparent}
\usepackage{xcolor}

\usetikzlibrary{decorations.pathreplacing}
\tcbuselibrary{skins}
% Declare math operators
\DeclareMathOperator{\lcm}{lcm}
\DeclareMathOperator{\proj}{proj}
\DeclareMathOperator{\vspan}{span}
\DeclareMathOperator{\im}{im}
\DeclareMathOperator{\range}{range}
\DeclareMathOperator{\Diff}{Diff}
\DeclareMathOperator{\Int}{Int}
\DeclareMathOperator{\fcn}{fcn}
\DeclareMathOperator{\id}{id}
\DeclareMathOperator{\rank}{rank}
\DeclareMathOperator{\tr}{tr}
\DeclareMathOperator{\dive}{div}
\DeclareMathOperator{\row}{row}
\DeclareMathOperator{\col}{col}
% Macros for letters/variables
\renewcommand{\tilde}{\raisebox{0.4ex}{\resizebox{2ex}{!}{\texttildelow}}}
\newcommand{\N}{\ensuremath{\mathbb{N}}}
\newcommand{\Z}{\ensuremath{\mathbb{Z}}}
\newcommand{\Q}{\ensuremath{\mathbb{Q}}}
\newcommand{\R}{\ensuremath{\mathbb{R}}}
\newcommand{\C}{\ensuremath{\mathbb{C}}}
\newcommand{\F}{\ensuremath{\mathbb{F}}}
\newcommand{\M}{\ensuremath{\mathbb{M}}}
\newcommand{\lam}{\ensuremath{\lambda}}
\newcommand{\nab}{\ensuremath{\nabla}}
\newcommand{\eps}{\ensuremath{\varepsilon}}
\newcommand{\es}{\ensuremath{\varnothing}}
% Macros for math symbols
\newcommand{\dx}[1]{\,\mathrm{d}#1}
\newcommand{\inv}{\ensuremath{^{-1}}}
\newcommand{\sm}{\setminus}
\newcommand{\sse}{\subseteq}
\newcommand{\ceq}{\coloneqq}
% Macros for pairs of math symbols
\newcommand{\abs}[1]{\ensuremath{\left\lvert #1 \right\rvert}}
\newcommand{\paren}[1]{\ensuremath{\left( #1 \right)}}
\newcommand{\norm}[1]{\ensuremath{\left\lVert #1\right\rVert}}
\newcommand{\set}[1]{\ensuremath{\left\{#1\right\}}}
\newcommand{\tup}[1]{\ensuremath{\left\langle #1 \right\rangle}}
\newcommand{\floor}[1]{\ensuremath{\left\lfloor #1 \right\rfloor}}
\newcommand{\ceil}[1]{\ensuremath{\left\lceil #1 \right\rceil}}
\newcommand{\eclass}[1]{\ensuremath{\left[ #1 \right]}}

\newcommand{\chapternum}{}
\newcommand{\ex}[1]{\noindent\textbf{Exercise \chapternum.{#1}.}}

\newcommand{\tsub}[1]{\textsubscript{#1}}
\newcommand{\tsup}[1]{\textsuperscript{#1}}

% Include figures
\newcommand{\incfig}[2][1]{%
    \def\svgwidth{#1\columnwidth}
    \import{./figures/}{#2.pdf_tex}
}

\definecolor{problemBackground}{RGB}{212,232,246}

\newenvironment{problem}[1]
  {
    \begin{tcolorbox}[
      boxrule=.5pt,
      titlerule=.5pt,
      sharp corners,
      colback=problemBackground,
      breakable
    ]
    \ifx &#1& \textbf{Problem. }
    \else \textbf{Problem #1.} \fi
  }
  {
    \end{tcolorbox}
  }
\definecolor{exampleBackground}{RGB}{255,249,248}
\definecolor{exampleAccent}{RGB}{158,60,14}
\newenvironment{example}[1]
  {
    \begin{tcolorbox}[
      boxrule=.5pt,
      sharp corners,
      colback=exampleBackground,
      colframe=exampleAccent,
    ]
    \color{exampleAccent}\textbf{Example.} \emph{#1}\color{black}
  }
  {
    \end{tcolorbox}
  }
\definecolor{theoremBackground}{RGB}{234,243,251}
\definecolor{theoremAccent}{RGB}{0,116,183}
\newenvironment{theorem}[1]
  {
    \begin{tcolorbox}[
      boxrule=.5pt,
      titlerule=.5pt,
      sharp corners,
      colback=theoremBackground,
      colframe=theoremAccent,
      breakable
    ]
      \color{theoremAccent}\textbf{Theorem --- }\emph{#1}\\\color{black}
  }
  {
    \end{tcolorbox}
  }
\definecolor{noteBackground}{RGB}{244,249,244}
\definecolor{noteAccent}{RGB}{34,139,34}
\newenvironment{note}[1]
  {
  \begin{tcolorbox}[
    enhanced,
    boxrule=0pt,
    frame hidden,
    sharp corners,
    colback=noteBackground,
    borderline west={3pt}{-1.5pt}{noteAccent},
    breakable
    ]
    \ifx &#1& \color{noteAccent}\textbf{Note. }\color{black}
    \else \color{noteAccent}\textbf{Note (#1). }\color{black} \fi
    }
    {
  \end{tcolorbox}
  }
\definecolor{lemmaBackground}{RGB}{255,247,234}
\definecolor{lemmaAccent}{RGB}{255,153,0}
\newenvironment{lemma}[1]
  {
  \begin{tcolorbox}[
    enhanced,
    boxrule=0pt,
    frame hidden,
    sharp corners,
    colback=lemmaBackground,
    borderline west={3pt}{-1.5pt}{lemmaAccent},
    breakable
    ]
    \ifx &#1& \color{lemmaAccent}\textbf{Lemma. }\color{black}
    \else \color{lemmaAccent}\textbf{Lemma #1. }\color{black} \fi
    }
    {
  \end{tcolorbox}
  }
\definecolor{definitionBackground}{RGB}{246,246,246}
\newenvironment{definition}[1]
  {
    \begin{tcolorbox}[
      enhanced,
      boxrule=0pt,
      frame hidden,
      sharp corners,
      colback=definitionBackground,
      borderline west={3pt}{-1.5pt}{black},
      breakable
    ]
    \textbf{Definition. }\emph{#1}\\
  }
  {
    \end{tcolorbox}
  }

\newenvironment{amatrix}[2]{
    \left[
      \begin{array}{*{#1}{c}|*{#2}c}
  }
  {
      \end{array}
    \right]
  }
\definecolor{codeBackground}{RGB}{253,246,225}
\definecolor{dkgreen}{rgb}{0,0.6,0}
\definecolor{gray}{rgb}{0.5,0.5,0.5}
\definecolor{mauve}{rgb}{0.58,0,0.82}
\lstset{
  language=C++,
  aboveskip=3mm,
  belowskip=3mm,
  backgroundcolor=\color{codeBackground},
  showstringspaces=false,
  columns=flexible,
  basicstyle={\small\ttfamily},
  numbers=none,
  numberstyle=\tiny\color{gray},
  keywordstyle=\color{blue},
  commentstyle=\color{dkgreen},
  stringstyle=\color{mauve},
  breaklines=true,
  breakatwhitespace=true,
  tabsize=2
}

\date{\the\year-\the\month-\the\day}
\author{Kyle Chui}


\fancyhf{}
\lhead{Kyle Chui}
\rhead{Page \thepage}
\pagestyle{fancy}

\begin{document}
  \textbf{Claim.} $\sum_{i=k}^{n}\binom{i}{k}=\binom{n+1}{k+1}$.
  \begin{proof}
    We know that $\binom{n+1}{k+1}$ is the number of ways to choose $k+1$ distinct objects from $n+1$ objects. We partition these ways into sets $S_k, S_{k+1}, S_n$, where $\abs{S_i} = \binom{i}{k}$. Consider fixing the maximum of the set, and then choosing the remaining $k$ items from the items less than the ``max". Then for some maximum $i+1$, there are $\binom{i}{k}$ ways to choose the remaining $k$ items from the first $i$ items. Iterating from $i=k$ to $i=n$, we see that this counts the ways to choose $k+1$ items from $n+1$ total items, so the statement holds.
  \end{proof}
  \begin{example}{}
    How many ways are there to choose $4$ items from the numbers $1$-$7$ where the largest item is $6$? \\[10pt]
    You first pick $6$, then you need to pick $3$ items from the first $5$ items, so we have $\binom{6-1}{4-1} = \binom{5}{3}$ ways to do this. This is the same logic that we are using in the above, except we are also steadily increasing the maximum to form a partition.
  \end{example}
  \textbf{Alternative Proof of the Above Claim.} We will use the fact that $\binom{n}{i}+\binom{n}{i-1}=\binom{n+1}{i}$.
  \begin{proof}
    \begin{align*}
      \binom{k}{k}+\binom{k+1}{k}+ \dotsb+\binom{n}{k} &= \binom{k}{k}+\paren{\binom{k+2}{k+1}-\binom{k+1}{k+1}}+ \dotsb+\paren{\binom{n+1}{k+1}-\binom{n}{k+1}} \\
                                                       &= \binom{k}{k} - \binom{k+1}{k+1} + \binom{n+1}{k+1} \\
                                                       &= \binom{n+1}{k+1}.
    \end{align*}
  \end{proof}
  \subsection{Pigeonhole Principle}
  If $n$ pigeons fly into $k$ holes, and $n>k$, at least one hole will contain more than one pigeon. \\[10pt]
  \textbf{Claim.} There are at least two people in this Zoom room that were born on the same day of the month (not necessarily the same month). \\[10pt]
  There are $31$ possible days of the month to be born on (the holes), and there are $45$ people in the Zoom room (the pigeons), so there is at least one day which has more than one person born on it.
  \begin{definition}{Pigeonhole Principle}
    If $X$ and $Y$ are finite sets and $\abs{X} > \abs{Y}$, then every function $f\colon X\to Y$ is \emph{not} injective.
  \end{definition}
  \begin{example}{}
    There is a party with $10$ people---at least $2$ people know the same number of people there (assume knowing is symmetric). \\[10pt]
    Let the ten people be denoted $p_1, p_2, \dotsc, p_{10}$. There are $10$ options for how many other people they can know (0-9). However, because knowing is symmetric, we cannot simultaneously have someone who knows $0$ people and someone who knows $9$ people (everyone else). So, there are really $9$ options (either 0-8 or 1-9). Thus, by pigeonhole principle, there exists two people that know the same number of people.
  \end{example}
  \begin{note}{}
    The pigeonhole principle is an \emph{existence proof}. It doesn't tell us who the people are, how many of them there are---just that they \emph{exist}. It is also called \emph{non-constructive}, because we don't explicitly say who the two people are (construct the group).
  \end{note}
  \begin{example}{}
    In this class there are $170$ people. What's the largest number of people (in the class) that we know for sure are born in the same month? \\[10pt]
    Because $170 > 12\cdot 14$, we know that it can't be the case that in each month $\leq 14$ people were born in that month. So, there must exist some month where at least $15$ of us were born in that month.
  \end{example}
  \begin{definition}{Generalized Pigeonhole Principle}
    If there are more than $m\cdot k$ pigeons that go into $k$ holes, then some hole has more than $m$ pigeons. In other words, if $f\colon X\to Y$, $\abs{X}=n$, $\abs{Y}=m$, $k=\ceil{\frac{n}{m}}$, then there are $k$ distinct values $a_1, \dotsc, a_k\in X$ with $f(a_1) = f(a_2) = \dotsb = f(a_k)$.
  \end{definition}
\end{document}
