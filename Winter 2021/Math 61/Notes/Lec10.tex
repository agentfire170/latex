\documentclass[class=article, crop=false]{standalone}
% Import packages
\usepackage[margin=1in]{geometry}

\usepackage[many]{tcolorbox}
\usepackage{amssymb, amsthm}
\usepackage{comment}
\usepackage{enumitem}
\usepackage{fancyhdr}
\usepackage{hyperref}
\usepackage{import}
\usepackage{listings}
\usepackage{mathrsfs, mathtools}
\usepackage{pdfpages}
\usepackage{standalone}
\usepackage{transparent}
\usepackage{xcolor}

\usetikzlibrary{decorations.pathreplacing}
\tcbuselibrary{skins}
% Declare math operators
\DeclareMathOperator{\lcm}{lcm}
\DeclareMathOperator{\proj}{proj}
\DeclareMathOperator{\vspan}{span}
\DeclareMathOperator{\im}{im}
\DeclareMathOperator{\range}{range}
\DeclareMathOperator{\Diff}{Diff}
\DeclareMathOperator{\Int}{Int}
\DeclareMathOperator{\fcn}{fcn}
\DeclareMathOperator{\id}{id}
\DeclareMathOperator{\rank}{rank}
\DeclareMathOperator{\tr}{tr}
\DeclareMathOperator{\dive}{div}
\DeclareMathOperator{\row}{row}
\DeclareMathOperator{\col}{col}
% Macros for letters/variables
\renewcommand{\tilde}{\raisebox{0.4ex}{\resizebox{2ex}{!}{\texttildelow}}}
\newcommand{\N}{\ensuremath{\mathbb{N}}}
\newcommand{\Z}{\ensuremath{\mathbb{Z}}}
\newcommand{\Q}{\ensuremath{\mathbb{Q}}}
\newcommand{\R}{\ensuremath{\mathbb{R}}}
\newcommand{\C}{\ensuremath{\mathbb{C}}}
\newcommand{\F}{\ensuremath{\mathbb{F}}}
\newcommand{\M}{\ensuremath{\mathbb{M}}}
\newcommand{\lam}{\ensuremath{\lambda}}
\newcommand{\nab}{\ensuremath{\nabla}}
\newcommand{\eps}{\ensuremath{\varepsilon}}
\newcommand{\es}{\ensuremath{\varnothing}}
% Macros for math symbols
\newcommand{\dx}[1]{\,\mathrm{d}#1}
\newcommand{\inv}{\ensuremath{^{-1}}}
\newcommand{\sm}{\setminus}
\newcommand{\sse}{\subseteq}
\newcommand{\ceq}{\coloneqq}
% Macros for pairs of math symbols
\newcommand{\abs}[1]{\ensuremath{\left\lvert #1 \right\rvert}}
\newcommand{\paren}[1]{\ensuremath{\left( #1 \right)}}
\newcommand{\norm}[1]{\ensuremath{\left\lVert #1\right\rVert}}
\newcommand{\set}[1]{\ensuremath{\left\{#1\right\}}}
\newcommand{\tup}[1]{\ensuremath{\left\langle #1 \right\rangle}}
\newcommand{\floor}[1]{\ensuremath{\left\lfloor #1 \right\rfloor}}
\newcommand{\ceil}[1]{\ensuremath{\left\lceil #1 \right\rceil}}
\newcommand{\eclass}[1]{\ensuremath{\left[ #1 \right]}}

\newcommand{\chapternum}{}
\newcommand{\ex}[1]{\noindent\textbf{Exercise \chapternum.{#1}.}}

\newcommand{\tsub}[1]{\textsubscript{#1}}
\newcommand{\tsup}[1]{\textsuperscript{#1}}

% Include figures
\newcommand{\incfig}[2][1]{%
    \def\svgwidth{#1\columnwidth}
    \import{./figures/}{#2.pdf_tex}
}

\definecolor{problemBackground}{RGB}{212,232,246}

\newenvironment{problem}[1]
  {
    \begin{tcolorbox}[
      boxrule=.5pt,
      titlerule=.5pt,
      sharp corners,
      colback=problemBackground,
      breakable
    ]
    \ifx &#1& \textbf{Problem. }
    \else \textbf{Problem #1.} \fi
  }
  {
    \end{tcolorbox}
  }
\definecolor{exampleBackground}{RGB}{255,249,248}
\definecolor{exampleAccent}{RGB}{158,60,14}
\newenvironment{example}[1]
  {
    \begin{tcolorbox}[
      boxrule=.5pt,
      sharp corners,
      colback=exampleBackground,
      colframe=exampleAccent,
    ]
    \color{exampleAccent}\textbf{Example.} \emph{#1}\color{black}
  }
  {
    \end{tcolorbox}
  }
\definecolor{theoremBackground}{RGB}{234,243,251}
\definecolor{theoremAccent}{RGB}{0,116,183}
\newenvironment{theorem}[1]
  {
    \begin{tcolorbox}[
      boxrule=.5pt,
      titlerule=.5pt,
      sharp corners,
      colback=theoremBackground,
      colframe=theoremAccent,
      breakable
    ]
      \color{theoremAccent}\textbf{Theorem --- }\emph{#1}\\\color{black}
  }
  {
    \end{tcolorbox}
  }
\definecolor{noteBackground}{RGB}{244,249,244}
\definecolor{noteAccent}{RGB}{34,139,34}
\newenvironment{note}[1]
  {
  \begin{tcolorbox}[
    enhanced,
    boxrule=0pt,
    frame hidden,
    sharp corners,
    colback=noteBackground,
    borderline west={3pt}{-1.5pt}{noteAccent},
    breakable
    ]
    \ifx &#1& \color{noteAccent}\textbf{Note. }\color{black}
    \else \color{noteAccent}\textbf{Note (#1). }\color{black} \fi
    }
    {
  \end{tcolorbox}
  }
\definecolor{lemmaBackground}{RGB}{255,247,234}
\definecolor{lemmaAccent}{RGB}{255,153,0}
\newenvironment{lemma}[1]
  {
  \begin{tcolorbox}[
    enhanced,
    boxrule=0pt,
    frame hidden,
    sharp corners,
    colback=lemmaBackground,
    borderline west={3pt}{-1.5pt}{lemmaAccent},
    breakable
    ]
    \ifx &#1& \color{lemmaAccent}\textbf{Lemma. }\color{black}
    \else \color{lemmaAccent}\textbf{Lemma #1. }\color{black} \fi
    }
    {
  \end{tcolorbox}
  }
\definecolor{definitionBackground}{RGB}{246,246,246}
\newenvironment{definition}[1]
  {
    \begin{tcolorbox}[
      enhanced,
      boxrule=0pt,
      frame hidden,
      sharp corners,
      colback=definitionBackground,
      borderline west={3pt}{-1.5pt}{black},
      breakable
    ]
    \textbf{Definition. }\emph{#1}\\
  }
  {
    \end{tcolorbox}
  }

\newenvironment{amatrix}[2]{
    \left[
      \begin{array}{*{#1}{c}|*{#2}c}
  }
  {
      \end{array}
    \right]
  }
\definecolor{codeBackground}{RGB}{253,246,225}
\definecolor{dkgreen}{rgb}{0,0.6,0}
\definecolor{gray}{rgb}{0.5,0.5,0.5}
\definecolor{mauve}{rgb}{0.58,0,0.82}
\lstset{
  language=C++,
  aboveskip=3mm,
  belowskip=3mm,
  backgroundcolor=\color{codeBackground},
  showstringspaces=false,
  columns=flexible,
  basicstyle={\small\ttfamily},
  numbers=none,
  numberstyle=\tiny\color{gray},
  keywordstyle=\color{blue},
  commentstyle=\color{dkgreen},
  stringstyle=\color{mauve},
  breaklines=true,
  breakatwhitespace=true,
  tabsize=2
}

\date{\the\year-\the\month-\the\day}
\author{Kyle Chui}


\fancyhf{}
\lhead{Kyle Chui}
\rhead{Page \thepage}
\pagestyle{fancy}

\begin{document}
  \subsection{More Counting}
  \begin{example}{Poker Hands}\\
    A deck of cards has 52 cards in four suits. There are thirteen different denominations:
    \begin{itemize}
      \item The numbers 2-10
      \item Jack
      \item Queen
      \item King
      \item Ace
    \end{itemize}
    A hand in poker is just five cards.
    \begin{enumerate}[label=(\alph*)]
      \item How many distinct hands are there? \\[10pt]
      There are $\binom{52}{5} \sim 2.5$ million ways to choose 5 cards out of 52.
      \item How many flushes are there? (When all the cards are the same suit)\\[10pt]
      There are $\binom{13}{5} \cdot 4$ ways, because there are $\binom{13}{5}$ ways to get a flush for a given suit, and you multiply by 4 because there are 4 suits.
      \item How many hands have three cards of one denomination and two of another? \\[10pt]
      There are $13 \cdot \binom{4}{3} \cdot 12 \cdot \binom{4}{2}$ such hands. There are 13 choices for the first denomination, and you choose 3 out of the 4 cards to form your hand, then 12 remaining denominations, and you choose 2 out of those 4 cards to form your hand.
    \end{enumerate}
  \end{example}
  \begin{example}{}
    How many words/strings can be formed using all of the letters of COMBINATORICS? (i.e. how many ways are there to rearrange the letters of COMBINATORICS) \\[10pt]
    Although the word has 13 letters, it is not $13!$ because you have repeated letters that are indistinguishable from each other, so you will overcount. For example, C\textsubscript{1}OMBINATORIC\textsubscript{2}S and C\textsubscript{2}OMBINATORIC\textsubscript{1}S are the same word, although $13!$ would count them differently. Thus we divide by the number of ways to rearrange the indistinct letters, namely the 2 C's, 2 O's, and 2 I's. We get $\frac{13!}{2!2!2!}$.\par
    Alternatively, we could also imagine filling thirteen slots with the letters in the word combinatorics. There are $\binom{13}{2}$ ways to put down the two C's, $\binom{11}{2}$ ways to put the two O's in the remaining 11 slots, and $\binom{9}{2}$ ways to put down the two I's in the 9 remaining slots. Finally, the last 7 letters may be put in any order. Thus we have $\binom{13}{2}\binom{11}{2}\binom{9}{2}7!$ ways.
  \end{example}
  \begin{note}{Generalized permutations}
    If a collection of $n$ items has: $n_1$ of one type (all identical), $n_2$ of another type (also all identical), $\dotsc,$ $n_t$ of type $t$ (where $n_1 + n_2 + \dotsb + n_t = n$), then the number of ways to order the $n$ items is:
    \[
      \binom{n}{n_1}\binom{n-n_1}{n_2}\dotsm \binom{n-(n_1+\dotsb n_{t-1})}{n_t},
    \]
    or we may also write
    \[
      \frac{n!}{n_1!n_2!\dotsm n_t!}.
    \]
  \end{note}
  \begin{example}{}
    Harry, Ron, and Hermione are sharing $10$ (distinct) every flavor beans. Harry gets $5$, Hermione $3$, and Ron gets $2$. In how many ways can they distribute the beans? \\[10pt]
    There are $10!$ total ways to arrange $10$ distinct beans. We then divide by the number of ways to rearrange each person's beans, so we have $\frac{10!}{5!3!2!}$ total ways to distribute the beans.\par
    Alternatively, we may choose 5 beans from 10 to give to Harry, 3 from the remaining 5 for Hermione, and 2 from the last 2 for Ron. Thus we have $\binom{10}{5}\binom{5}{3}\binom{2}{2}$ ways to distribute the beans.
  \end{example}
  \begin{example}{}
    Harry, Ron, and Hermione have $10$ identical beans. In how many ways can they divide up the beans? \\[10pt]
    We will denote the beans as stars, and use bars to divvy up the beans between the three people. Thus we have $10$ stars and $2$ bars. An example distribution looks like this:
    \[
      \underbrace{**}_{\text{Harry's beans}}|\underbrace{*****}_{\text{Hermione's beans}}|\underbrace{***}_{\text{Ron's beans}}
    \]
    Observe that any such distribution is akin to having $12$ items total and just choosing where the $2$ bars go. Thus we have $\binom{10+3-1}{3-1}$ ways to distribute the beans (The $-1$ is because three people only need $2$ bars, not $3$).
  \end{example}
  \begin{note}{Stars and Bars}
    If a set has $t$ elements, the number of unordered selections from the set with $k$-elements (with repetitions allowed) is $\binom{k+t+1}{t-1}$.
  \end{note}
  \begin{example}{}
    How many solutions are there to the equation:
    \[
      x_1 + x_2 + x_3 + x_4 + x_5 = 25, \text{where all }x_i \geq0 \text{ and } x_i\in\Z
    \]
    We have $25$ stars (the total sum) and $4$ bars (dividing the $25$ stars into 5 categories). The number of stars in each area represents the value of each variable. Thus we have $\binom{25+5-1}{5-1} = \binom{29}{4}$ ways to satisfy the equation with the given restrictions. \\[10pt]
    \textbf{Problem.} What if all $x_i\geq 1$? \\[10pt]
    This is the same as if the number of solutions to $x_1 + x_2 + x_3 + x_4 + x_5 = 20$.
  \end{example}
\end{document}
