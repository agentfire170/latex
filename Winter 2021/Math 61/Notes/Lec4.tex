\documentclass[class=article, crop=false]{standalone}
% Import packages
\usepackage[margin=1in]{geometry}

\usepackage[many]{tcolorbox}
\usepackage{amssymb, amsthm}
\usepackage{comment}
\usepackage{enumitem}
\usepackage{fancyhdr}
\usepackage{hyperref}
\usepackage{import}
\usepackage{listings}
\usepackage{mathrsfs, mathtools}
\usepackage{pdfpages}
\usepackage{standalone}
\usepackage{transparent}
\usepackage{xcolor}

\usetikzlibrary{decorations.pathreplacing}
\tcbuselibrary{skins}
% Declare math operators
\DeclareMathOperator{\lcm}{lcm}
\DeclareMathOperator{\proj}{proj}
\DeclareMathOperator{\vspan}{span}
\DeclareMathOperator{\im}{im}
\DeclareMathOperator{\range}{range}
\DeclareMathOperator{\Diff}{Diff}
\DeclareMathOperator{\Int}{Int}
\DeclareMathOperator{\fcn}{fcn}
\DeclareMathOperator{\id}{id}
\DeclareMathOperator{\rank}{rank}
\DeclareMathOperator{\tr}{tr}
\DeclareMathOperator{\dive}{div}
\DeclareMathOperator{\row}{row}
\DeclareMathOperator{\col}{col}
% Macros for letters/variables
\renewcommand{\tilde}{\raisebox{0.4ex}{\resizebox{2ex}{!}{\texttildelow}}}
\newcommand{\N}{\ensuremath{\mathbb{N}}}
\newcommand{\Z}{\ensuremath{\mathbb{Z}}}
\newcommand{\Q}{\ensuremath{\mathbb{Q}}}
\newcommand{\R}{\ensuremath{\mathbb{R}}}
\newcommand{\C}{\ensuremath{\mathbb{C}}}
\newcommand{\F}{\ensuremath{\mathbb{F}}}
\newcommand{\M}{\ensuremath{\mathbb{M}}}
\newcommand{\lam}{\ensuremath{\lambda}}
\newcommand{\nab}{\ensuremath{\nabla}}
\newcommand{\eps}{\ensuremath{\varepsilon}}
\newcommand{\es}{\ensuremath{\varnothing}}
% Macros for math symbols
\newcommand{\dx}[1]{\,\mathrm{d}#1}
\newcommand{\inv}{\ensuremath{^{-1}}}
\newcommand{\sm}{\setminus}
\newcommand{\sse}{\subseteq}
\newcommand{\ceq}{\coloneqq}
% Macros for pairs of math symbols
\newcommand{\abs}[1]{\ensuremath{\left\lvert #1 \right\rvert}}
\newcommand{\paren}[1]{\ensuremath{\left( #1 \right)}}
\newcommand{\norm}[1]{\ensuremath{\left\lVert #1\right\rVert}}
\newcommand{\set}[1]{\ensuremath{\left\{#1\right\}}}
\newcommand{\tup}[1]{\ensuremath{\left\langle #1 \right\rangle}}
\newcommand{\floor}[1]{\ensuremath{\left\lfloor #1 \right\rfloor}}
\newcommand{\ceil}[1]{\ensuremath{\left\lceil #1 \right\rceil}}
\newcommand{\eclass}[1]{\ensuremath{\left[ #1 \right]}}

\newcommand{\chapternum}{}
\newcommand{\ex}[1]{\noindent\textbf{Exercise \chapternum.{#1}.}}

\newcommand{\tsub}[1]{\textsubscript{#1}}
\newcommand{\tsup}[1]{\textsuperscript{#1}}

% Include figures
\newcommand{\incfig}[2][1]{%
    \def\svgwidth{#1\columnwidth}
    \import{./figures/}{#2.pdf_tex}
}

\definecolor{problemBackground}{RGB}{212,232,246}

\newenvironment{problem}[1]
  {
    \begin{tcolorbox}[
      boxrule=.5pt,
      titlerule=.5pt,
      sharp corners,
      colback=problemBackground,
      breakable
    ]
    \ifx &#1& \textbf{Problem. }
    \else \textbf{Problem #1.} \fi
  }
  {
    \end{tcolorbox}
  }
\definecolor{exampleBackground}{RGB}{255,249,248}
\definecolor{exampleAccent}{RGB}{158,60,14}
\newenvironment{example}[1]
  {
    \begin{tcolorbox}[
      boxrule=.5pt,
      sharp corners,
      colback=exampleBackground,
      colframe=exampleAccent,
    ]
    \color{exampleAccent}\textbf{Example.} \emph{#1}\color{black}
  }
  {
    \end{tcolorbox}
  }
\definecolor{theoremBackground}{RGB}{234,243,251}
\definecolor{theoremAccent}{RGB}{0,116,183}
\newenvironment{theorem}[1]
  {
    \begin{tcolorbox}[
      boxrule=.5pt,
      titlerule=.5pt,
      sharp corners,
      colback=theoremBackground,
      colframe=theoremAccent,
      breakable
    ]
      \color{theoremAccent}\textbf{Theorem --- }\emph{#1}\\\color{black}
  }
  {
    \end{tcolorbox}
  }
\definecolor{noteBackground}{RGB}{244,249,244}
\definecolor{noteAccent}{RGB}{34,139,34}
\newenvironment{note}[1]
  {
  \begin{tcolorbox}[
    enhanced,
    boxrule=0pt,
    frame hidden,
    sharp corners,
    colback=noteBackground,
    borderline west={3pt}{-1.5pt}{noteAccent},
    breakable
    ]
    \ifx &#1& \color{noteAccent}\textbf{Note. }\color{black}
    \else \color{noteAccent}\textbf{Note (#1). }\color{black} \fi
    }
    {
  \end{tcolorbox}
  }
\definecolor{lemmaBackground}{RGB}{255,247,234}
\definecolor{lemmaAccent}{RGB}{255,153,0}
\newenvironment{lemma}[1]
  {
  \begin{tcolorbox}[
    enhanced,
    boxrule=0pt,
    frame hidden,
    sharp corners,
    colback=lemmaBackground,
    borderline west={3pt}{-1.5pt}{lemmaAccent},
    breakable
    ]
    \ifx &#1& \color{lemmaAccent}\textbf{Lemma. }\color{black}
    \else \color{lemmaAccent}\textbf{Lemma #1. }\color{black} \fi
    }
    {
  \end{tcolorbox}
  }
\definecolor{definitionBackground}{RGB}{246,246,246}
\newenvironment{definition}[1]
  {
    \begin{tcolorbox}[
      enhanced,
      boxrule=0pt,
      frame hidden,
      sharp corners,
      colback=definitionBackground,
      borderline west={3pt}{-1.5pt}{black},
      breakable
    ]
    \textbf{Definition. }\emph{#1}\\
  }
  {
    \end{tcolorbox}
  }

\newenvironment{amatrix}[2]{
    \left[
      \begin{array}{*{#1}{c}|*{#2}c}
  }
  {
      \end{array}
    \right]
  }
\definecolor{codeBackground}{RGB}{253,246,225}
\definecolor{dkgreen}{rgb}{0,0.6,0}
\definecolor{gray}{rgb}{0.5,0.5,0.5}
\definecolor{mauve}{rgb}{0.58,0,0.82}
\lstset{
  language=C++,
  aboveskip=3mm,
  belowskip=3mm,
  backgroundcolor=\color{codeBackground},
  showstringspaces=false,
  columns=flexible,
  basicstyle={\small\ttfamily},
  numbers=none,
  numberstyle=\tiny\color{gray},
  keywordstyle=\color{blue},
  commentstyle=\color{dkgreen},
  stringstyle=\color{mauve},
  breaklines=true,
  breakatwhitespace=true,
  tabsize=2
}

\date{\the\year-\the\month-\the\day}
\author{Kyle Chui}


\fancyhf{}
\lhead{Kyle Chui}
\rhead{Page \thepage}
\pagestyle{fancy}

\begin{document}
  \section{Sets and Functions}
  \subsection{Power Sets}
  \begin{definition}{Power Set}
    If $X$ is a set, the \emph{power set} of $X$, denoted $\mathscr{P}(X)$, is the set of subsets of $X$.
  \end{definition}
  \begin{example}{Power Sets}
    \begin{itemize}
      \item $\mathscr{P}(\es) = \set{\es}$
      \item $\mathscr{P}(\set{a, b}) = \set{\es, \set{a}, \set{b}, \set{a, b}}$
      \item $\mathscr{P}(\set{a, b, c}) = \set{\es, \set{a}, \set{b}, \set{c}, \set{a, b}, \set{b, c}, \set{a, c}, \set{a, b, c}}$
    \end{itemize}
  \end{example}
  \begin{definition}{Cardinality of Finite Sets}
    If $X$ has finitely many elements, then $|X|$ denotes the number of elements of $X$.
  \end{definition}
  \begin{theorem}{Cardinality of Power Sets}
    If $X$ is finite, then $\abs{\mathscr{P}(X)} = 2^{\abs{X}}$.
  \end{theorem}
  \begin{proof}
    Let us induct on the cardinality of the set $X$. Suppose $\abs{X} = 0$, so that $X = \es$. Then $\mathscr{P}(X) = \set{\es}$, so $\abs{\mathscr{P}(X)} = 1 = 2^0$. Thus the statement is true when $\abs{X} = 0$. \par
    Suppose that the statement holds for some non-negative integer $k$. Let $Y$ be a set such that $\abs{Y} = k + 1$, and $y\in Y$. Observe that we may split $\mathscr{P}(Y)$ into two groups: the subsets containing $y$, and the subsets that do not contain $y$. A subset of $Y$ that does not contain $y$ is exactly $Y \sm \set{y}$, which has $k$ elements. By the inductive hypothesis, there exist $2^k$ such subsets. A subset of $Y$ that does contain $y$ is obtained by adding $y$ to a subset of $Y$ which does not contain $y$. Again, there are $2^k$ such subsets. Any subset of $Y$ either does or does not contain $y$ (but not both), so there are $2^k + 2^k = 2^{k + 1}$ subsets of $Y$. Therefore $\mathscr{P}(X) = 2^{\abs{X}}$ for all finite sets $\abs{X}$.
  \end{proof}
  \subsection{Functions}
  \begin{definition}{Function}
    If $X, Y$ are sets, a function $f$ from $X$ to $Y$, written $f\colon X\to Y$ is a subset of $X\times Y$ satisfying two properties:
    \begin{itemize}
      \item For all $a\in X$, there exists $b\in Y$ such that $(a, b)\in f$
      \begin{itemize}
        \item Everything in the domain must get mapped to something in the codomain
      \end{itemize}
      \item For all $a\in X$ and $b, b'\in Y$, if $(a, b), (a, b')\in f$, then $b = b'$
      \begin{itemize}
        \item Every element in the domain can map to at most one element in the codomain
      \end{itemize}
    \end{itemize}
  \end{definition}
  \begin{note}{Function Notation}
    If $(a, b)\in f$, we write $f(a) = b$.
  \end{note}
  \begin{example}{Functions}
    \begin{itemize}
      \item $f\colon \Z\to \N$ such that $f(x) = x^2$
      \item $g\colon \N\to \N$ such that $g(x) = x^2$
    \end{itemize}
    Note that $f$ and $g$ are different functions.
  \end{example}
  \begin{definition}{Domain and Codomain of a Function}
    If $f\colon X\to Y$, $X$ is the domain of $f$ and $Y$ is the codomain of $f$.
  \end{definition}
  \begin{definition}{Range of a Function}
    For $f\colon X\to Y$, the range of $f$ is:
    \[
      \range f = \set{y\in Y\mid y = f(x) \text{ for some }x\in X}
    \]
  \end{definition}
  \begin{definition}{Surjectivity}
    A function $f\colon X\to Y$ is \emph{onto} or \emph{surjective} if $\range f = Y$. In other words, a function is surjective if its range is equal to its codomain.
  \end{definition}
  \begin{example}{Surjective Functions}
    \begin{itemize}
      \item $f\colon \set{a, b, c}\to \set{d, e, f}$ defined by $f = \set{(a, d), (b, e), (c, f)}$
      \item $f\colon \Z\to \N$ defined by $f(x) = |x|$
    \end{itemize}
  \end{example}
  \begin{definition}{Injectivity}
    A function $f\colon X\to Y$ is \emph{one-to-one} or \emph{injective} if, for all $x, y\in X$, $f(x) = f(y)$ implies that $x = y$. In other words, different elements in the domain map to different elements in the codomain.
  \end{definition}
  \begin{example}{Injective Functions}
    \begin{itemize}
      \item $g\colon \N\to \N$ defined by $g(x) = x^2$
    \end{itemize}
  \end{example}
  \begin{note}{Properties of Functions}
    Observe that the both the domain and codomain of a function matter when it comes to determining whether the function satisfies certain properties. For instance, $f\colon \Z\to \N$ defined by $f(x) = x^2$ is not injective, but restricting the domain to $\N$ would make it injective. Similarly, a function $f\colon \Z\to \Z$ defined by $f(x) = x^2$ is not surjective, but restricting the codomain to $\N$ would make it surjective.
  \end{note}
  \begin{definition}{Composition of Functions}
    If $f\colon X\to Y, g\colon Y\to Z$ are functions, then $g\circ f\colon X\to Z$ is a function defined by $(g\circ f)(x) = g(f(x))$.
  \end{definition}
  \newpage
  \begin{theorem}{Composition of Injective/Surjective Functions is Injective/Surjective}
    Let $f\colon X\to Y$, $g\colon Y\to Z$.
    \begin{itemize}
      \item If $f, g$ are injective, so is $g\circ f$
      \item If $f, g$ are surjective, so is $g\circ f$
    \end{itemize}
  \end{theorem}
  \begin{proof}
    Suppose $f, g$ are injective functions. Let $x, x' \in X$ such that $(g\circ f)(x) = (g\circ f)(x')$. Then
    \begin{align*}
      g(f(x)) &= g(f(x')) \\
      f(x) &= f(x') \tag{Because $g$ is injective} \\
      x &= x' \tag{Because $f$ is injective}
    \end{align*}
    Therefore $g\circ f$ is injective.
  \end{proof}
  \begin{proof}
    Suppose $f, g$ are surjective functions. Let $z\in Z$. Because $g$ is surjective, there exists some $y\in Y$ such that $g(y) = z$. Furthermore, because $f$ is surjective, there exists some $x\in X$ such that $f(x) = y$. Thus, for every $z\in Z$, there exists some $x\in X$ such that $(g\circ f)(x) = g(f(x)) = g(y) = z$, so $g\circ f$ is surjective.
  \end{proof}
  \begin{definition}{Bijectivity}
    If a function is both injective and surjective, then we say that it is \emph{bijective}.
  \end{definition}
  \begin{note}{Cardinality and Bijections}
    If there is a bijection between two sets, they have the same number of elements.
  \end{note}
\end{document}
