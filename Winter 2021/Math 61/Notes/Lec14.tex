\documentclass[class=article, crop=false]{standalone}
% Import packages
\usepackage[margin=1in]{geometry}

\usepackage[many]{tcolorbox}
\usepackage{amssymb, amsthm}
\usepackage{comment}
\usepackage{enumitem}
\usepackage{fancyhdr}
\usepackage{hyperref}
\usepackage{import}
\usepackage{listings}
\usepackage{mathrsfs, mathtools}
\usepackage{pdfpages}
\usepackage{standalone}
\usepackage{transparent}
\usepackage{xcolor}

\usetikzlibrary{decorations.pathreplacing}
\tcbuselibrary{skins}
% Declare math operators
\DeclareMathOperator{\lcm}{lcm}
\DeclareMathOperator{\proj}{proj}
\DeclareMathOperator{\vspan}{span}
\DeclareMathOperator{\im}{im}
\DeclareMathOperator{\range}{range}
\DeclareMathOperator{\Diff}{Diff}
\DeclareMathOperator{\Int}{Int}
\DeclareMathOperator{\fcn}{fcn}
\DeclareMathOperator{\id}{id}
\DeclareMathOperator{\rank}{rank}
\DeclareMathOperator{\tr}{tr}
\DeclareMathOperator{\dive}{div}
\DeclareMathOperator{\row}{row}
\DeclareMathOperator{\col}{col}
% Macros for letters/variables
\renewcommand{\tilde}{\raisebox{0.4ex}{\resizebox{2ex}{!}{\texttildelow}}}
\newcommand{\N}{\ensuremath{\mathbb{N}}}
\newcommand{\Z}{\ensuremath{\mathbb{Z}}}
\newcommand{\Q}{\ensuremath{\mathbb{Q}}}
\newcommand{\R}{\ensuremath{\mathbb{R}}}
\newcommand{\C}{\ensuremath{\mathbb{C}}}
\newcommand{\F}{\ensuremath{\mathbb{F}}}
\newcommand{\M}{\ensuremath{\mathbb{M}}}
\newcommand{\lam}{\ensuremath{\lambda}}
\newcommand{\nab}{\ensuremath{\nabla}}
\newcommand{\eps}{\ensuremath{\varepsilon}}
\newcommand{\es}{\ensuremath{\varnothing}}
% Macros for math symbols
\newcommand{\dx}[1]{\,\mathrm{d}#1}
\newcommand{\inv}{\ensuremath{^{-1}}}
\newcommand{\sm}{\setminus}
\newcommand{\sse}{\subseteq}
\newcommand{\ceq}{\coloneqq}
% Macros for pairs of math symbols
\newcommand{\abs}[1]{\ensuremath{\left\lvert #1 \right\rvert}}
\newcommand{\paren}[1]{\ensuremath{\left( #1 \right)}}
\newcommand{\norm}[1]{\ensuremath{\left\lVert #1\right\rVert}}
\newcommand{\set}[1]{\ensuremath{\left\{#1\right\}}}
\newcommand{\tup}[1]{\ensuremath{\left\langle #1 \right\rangle}}
\newcommand{\floor}[1]{\ensuremath{\left\lfloor #1 \right\rfloor}}
\newcommand{\ceil}[1]{\ensuremath{\left\lceil #1 \right\rceil}}
\newcommand{\eclass}[1]{\ensuremath{\left[ #1 \right]}}

\newcommand{\chapternum}{}
\newcommand{\ex}[1]{\noindent\textbf{Exercise \chapternum.{#1}.}}

\newcommand{\tsub}[1]{\textsubscript{#1}}
\newcommand{\tsup}[1]{\textsuperscript{#1}}

% Include figures
\newcommand{\incfig}[2][1]{%
    \def\svgwidth{#1\columnwidth}
    \import{./figures/}{#2.pdf_tex}
}

\definecolor{problemBackground}{RGB}{212,232,246}

\newenvironment{problem}[1]
  {
    \begin{tcolorbox}[
      boxrule=.5pt,
      titlerule=.5pt,
      sharp corners,
      colback=problemBackground,
      breakable
    ]
    \ifx &#1& \textbf{Problem. }
    \else \textbf{Problem #1.} \fi
  }
  {
    \end{tcolorbox}
  }
\definecolor{exampleBackground}{RGB}{255,249,248}
\definecolor{exampleAccent}{RGB}{158,60,14}
\newenvironment{example}[1]
  {
    \begin{tcolorbox}[
      boxrule=.5pt,
      sharp corners,
      colback=exampleBackground,
      colframe=exampleAccent,
    ]
    \color{exampleAccent}\textbf{Example.} \emph{#1}\color{black}
  }
  {
    \end{tcolorbox}
  }
\definecolor{theoremBackground}{RGB}{234,243,251}
\definecolor{theoremAccent}{RGB}{0,116,183}
\newenvironment{theorem}[1]
  {
    \begin{tcolorbox}[
      boxrule=.5pt,
      titlerule=.5pt,
      sharp corners,
      colback=theoremBackground,
      colframe=theoremAccent,
      breakable
    ]
      \color{theoremAccent}\textbf{Theorem --- }\emph{#1}\\\color{black}
  }
  {
    \end{tcolorbox}
  }
\definecolor{noteBackground}{RGB}{244,249,244}
\definecolor{noteAccent}{RGB}{34,139,34}
\newenvironment{note}[1]
  {
  \begin{tcolorbox}[
    enhanced,
    boxrule=0pt,
    frame hidden,
    sharp corners,
    colback=noteBackground,
    borderline west={3pt}{-1.5pt}{noteAccent},
    breakable
    ]
    \ifx &#1& \color{noteAccent}\textbf{Note. }\color{black}
    \else \color{noteAccent}\textbf{Note (#1). }\color{black} \fi
    }
    {
  \end{tcolorbox}
  }
\definecolor{lemmaBackground}{RGB}{255,247,234}
\definecolor{lemmaAccent}{RGB}{255,153,0}
\newenvironment{lemma}[1]
  {
  \begin{tcolorbox}[
    enhanced,
    boxrule=0pt,
    frame hidden,
    sharp corners,
    colback=lemmaBackground,
    borderline west={3pt}{-1.5pt}{lemmaAccent},
    breakable
    ]
    \ifx &#1& \color{lemmaAccent}\textbf{Lemma. }\color{black}
    \else \color{lemmaAccent}\textbf{Lemma #1. }\color{black} \fi
    }
    {
  \end{tcolorbox}
  }
\definecolor{definitionBackground}{RGB}{246,246,246}
\newenvironment{definition}[1]
  {
    \begin{tcolorbox}[
      enhanced,
      boxrule=0pt,
      frame hidden,
      sharp corners,
      colback=definitionBackground,
      borderline west={3pt}{-1.5pt}{black},
      breakable
    ]
    \textbf{Definition. }\emph{#1}\\
  }
  {
    \end{tcolorbox}
  }

\newenvironment{amatrix}[2]{
    \left[
      \begin{array}{*{#1}{c}|*{#2}c}
  }
  {
      \end{array}
    \right]
  }
\definecolor{codeBackground}{RGB}{253,246,225}
\definecolor{dkgreen}{rgb}{0,0.6,0}
\definecolor{gray}{rgb}{0.5,0.5,0.5}
\definecolor{mauve}{rgb}{0.58,0,0.82}
\lstset{
  language=C++,
  aboveskip=3mm,
  belowskip=3mm,
  backgroundcolor=\color{codeBackground},
  showstringspaces=false,
  columns=flexible,
  basicstyle={\small\ttfamily},
  numbers=none,
  numberstyle=\tiny\color{gray},
  keywordstyle=\color{blue},
  commentstyle=\color{dkgreen},
  stringstyle=\color{mauve},
  breaklines=true,
  breakatwhitespace=true,
  tabsize=2
}

\date{\the\year-\the\month-\the\day}
\author{Kyle Chui}


\fancyhf{}
\lhead{Kyle Chui}
\rhead{Page \thepage}
\pagestyle{fancy}

\begin{document}
  \begin{example}{}
    Suppose you have an $n\times n$ board. How many routes can you take from the lower left corner to the top right corner of the board, only moving to the right and up that don't go above the diagonal? We will denote this number as $C_n$. \\[10pt]
    By convention, we say that $C_0 = 1$. Looking at $1\times 1$ and $2\times 2$ grids, we see that $C_1 = 1$ and $C_2 = 2$. Observe that the ``first move" that we make is fixed because we can't go up with our first move. We ask, what is the number of allowed routes in the $n\times n$ board that first hit the diagonal at $(k, k)$? Observe that this partitions the board into two smaller boards, one of size $k$ and another of size $n-k$. However, because we do not hit the diagonal until $(k,k)$, our first move must be to the right, and our last move must be up. Thus we have that the number of routes from $(0,0)$ to $(k,k)$ is $C_{k-1}$. Then we have $C_{n-k}$ ways to traverse the remaining $(n-k)\times (n-k)$ grid. Letting $k$ be anything between $1$ and $n$, we have
    \[
      C_n = \sum_{k=1}^{n}C_{k-1}C_{n-k}.
    \]
  \end{example}
  \subsection{Solving Recurrence Relations}
  \begin{example}{The Towers of Hanoi} \\
    We have $n$ discs of increasing size, $3$ posts, where each disc rests on a post. We can only move a disc if it rests upon a larger disc. In how few moves can we move all the discs to another post? \\[10pt]
    If we have one disc, then we can move it to another post in one move. With two discs, we need three moves. To solve a tower with $n$ discs, we can move a tower with $n-1$ discs to the middle post, move the bottom disc to the third post, and then move the $n-1$ discs back on top of the largest disc. Thus we have the recurrence relation $h_n = 2h_{n-1} + 1$. \\[10pt]
    How can we get an explicit form for this relation? Observe that
    \begin{align*}
      h_n &= 2h_{n-1} + 1 \\
          &= 2(2h_{n-2} + 1) + 1 \\
          &= 2(2(2h_{n-3} + 1) + 1) + 1 \\
          &\hspace{5pt}\vdots \\
          &= 2^nh_0 + 2^0 + 2^1 + \dotsb + 2^{n-1} \tag*{Prove using induction}\\
          &= 2^nh_0 + 2^n - 1 \\
          &= 2^{n+1} - 1
    \end{align*}
  \end{example}
  \begin{example}{}
    Consider the sequence defined by $s_0 = 2$, $s_1 = 5$, and $s_n = 5s_{n-1} - 6s_{n-2}$ for $n\geq 2$. \\[10pt]
    Ansatz: $s_n$ grows exponentially, for example $s_n = x^n$ for some $x$. Then
    \begin{align*}
      x^n &= 5x^{n-1} - 6x^{n-2} \\
      x^n - 5x^{n-1} - 6x^{n-2} &= 0 \\
      x^{n-2}(x^2 - 5x - 6) &= 0.
    \end{align*}
    Thus we have that $x = 0, 2, 3$. Any sequence $s_n = a 3^n + b 2^n$ for some real numbers $a, b$ will satisfy the recurrence relation. All that's left is to make sure our recurrence relation matches the initial conditions given. We have
    \begin{align*}
      s_0 &= a + b = 2 \\
      s_1 &= 3a + 2b = 5,
    \end{align*}
    so $a = b = 1$, and $s_n = 2^n + 3^n$.
  \end{example}
  \begin{note}{}
    Linear combinations of recurrence relations also satisfy the recurrence relation.
  \end{note}
\end{document}
