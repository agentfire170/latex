\documentclass[class=article, crop=false]{standalone}
% Import packages
\usepackage[margin=1in]{geometry}

\usepackage[many]{tcolorbox}
\usepackage{amssymb, amsthm}
\usepackage{comment}
\usepackage{enumitem}
\usepackage{fancyhdr}
\usepackage{hyperref}
\usepackage{import}
\usepackage{listings}
\usepackage{mathrsfs, mathtools}
\usepackage{pdfpages}
\usepackage{standalone}
\usepackage{transparent}
\usepackage{xcolor}

\usetikzlibrary{decorations.pathreplacing}
\tcbuselibrary{skins}
% Declare math operators
\DeclareMathOperator{\lcm}{lcm}
\DeclareMathOperator{\proj}{proj}
\DeclareMathOperator{\vspan}{span}
\DeclareMathOperator{\im}{im}
\DeclareMathOperator{\range}{range}
\DeclareMathOperator{\Diff}{Diff}
\DeclareMathOperator{\Int}{Int}
\DeclareMathOperator{\fcn}{fcn}
\DeclareMathOperator{\id}{id}
\DeclareMathOperator{\rank}{rank}
\DeclareMathOperator{\tr}{tr}
\DeclareMathOperator{\dive}{div}
\DeclareMathOperator{\row}{row}
\DeclareMathOperator{\col}{col}
% Macros for letters/variables
\renewcommand{\tilde}{\raisebox{0.4ex}{\resizebox{2ex}{!}{\texttildelow}}}
\newcommand{\N}{\ensuremath{\mathbb{N}}}
\newcommand{\Z}{\ensuremath{\mathbb{Z}}}
\newcommand{\Q}{\ensuremath{\mathbb{Q}}}
\newcommand{\R}{\ensuremath{\mathbb{R}}}
\newcommand{\C}{\ensuremath{\mathbb{C}}}
\newcommand{\F}{\ensuremath{\mathbb{F}}}
\newcommand{\M}{\ensuremath{\mathbb{M}}}
\newcommand{\lam}{\ensuremath{\lambda}}
\newcommand{\nab}{\ensuremath{\nabla}}
\newcommand{\eps}{\ensuremath{\varepsilon}}
\newcommand{\es}{\ensuremath{\varnothing}}
% Macros for math symbols
\newcommand{\dx}[1]{\,\mathrm{d}#1}
\newcommand{\inv}{\ensuremath{^{-1}}}
\newcommand{\sm}{\setminus}
\newcommand{\sse}{\subseteq}
\newcommand{\ceq}{\coloneqq}
% Macros for pairs of math symbols
\newcommand{\abs}[1]{\ensuremath{\left\lvert #1 \right\rvert}}
\newcommand{\paren}[1]{\ensuremath{\left( #1 \right)}}
\newcommand{\norm}[1]{\ensuremath{\left\lVert #1\right\rVert}}
\newcommand{\set}[1]{\ensuremath{\left\{#1\right\}}}
\newcommand{\tup}[1]{\ensuremath{\left\langle #1 \right\rangle}}
\newcommand{\floor}[1]{\ensuremath{\left\lfloor #1 \right\rfloor}}
\newcommand{\ceil}[1]{\ensuremath{\left\lceil #1 \right\rceil}}
\newcommand{\eclass}[1]{\ensuremath{\left[ #1 \right]}}

\newcommand{\chapternum}{}
\newcommand{\ex}[1]{\noindent\textbf{Exercise \chapternum.{#1}.}}

\newcommand{\tsub}[1]{\textsubscript{#1}}
\newcommand{\tsup}[1]{\textsuperscript{#1}}

% Include figures
\newcommand{\incfig}[2][1]{%
    \def\svgwidth{#1\columnwidth}
    \import{./figures/}{#2.pdf_tex}
}

\definecolor{problemBackground}{RGB}{212,232,246}

\newenvironment{problem}[1]
  {
    \begin{tcolorbox}[
      boxrule=.5pt,
      titlerule=.5pt,
      sharp corners,
      colback=problemBackground,
      breakable
    ]
    \ifx &#1& \textbf{Problem. }
    \else \textbf{Problem #1.} \fi
  }
  {
    \end{tcolorbox}
  }
\definecolor{exampleBackground}{RGB}{255,249,248}
\definecolor{exampleAccent}{RGB}{158,60,14}
\newenvironment{example}[1]
  {
    \begin{tcolorbox}[
      boxrule=.5pt,
      sharp corners,
      colback=exampleBackground,
      colframe=exampleAccent,
    ]
    \color{exampleAccent}\textbf{Example.} \emph{#1}\color{black}
  }
  {
    \end{tcolorbox}
  }
\definecolor{theoremBackground}{RGB}{234,243,251}
\definecolor{theoremAccent}{RGB}{0,116,183}
\newenvironment{theorem}[1]
  {
    \begin{tcolorbox}[
      boxrule=.5pt,
      titlerule=.5pt,
      sharp corners,
      colback=theoremBackground,
      colframe=theoremAccent,
      breakable
    ]
      \color{theoremAccent}\textbf{Theorem --- }\emph{#1}\\\color{black}
  }
  {
    \end{tcolorbox}
  }
\definecolor{noteBackground}{RGB}{244,249,244}
\definecolor{noteAccent}{RGB}{34,139,34}
\newenvironment{note}[1]
  {
  \begin{tcolorbox}[
    enhanced,
    boxrule=0pt,
    frame hidden,
    sharp corners,
    colback=noteBackground,
    borderline west={3pt}{-1.5pt}{noteAccent},
    breakable
    ]
    \ifx &#1& \color{noteAccent}\textbf{Note. }\color{black}
    \else \color{noteAccent}\textbf{Note (#1). }\color{black} \fi
    }
    {
  \end{tcolorbox}
  }
\definecolor{lemmaBackground}{RGB}{255,247,234}
\definecolor{lemmaAccent}{RGB}{255,153,0}
\newenvironment{lemma}[1]
  {
  \begin{tcolorbox}[
    enhanced,
    boxrule=0pt,
    frame hidden,
    sharp corners,
    colback=lemmaBackground,
    borderline west={3pt}{-1.5pt}{lemmaAccent},
    breakable
    ]
    \ifx &#1& \color{lemmaAccent}\textbf{Lemma. }\color{black}
    \else \color{lemmaAccent}\textbf{Lemma #1. }\color{black} \fi
    }
    {
  \end{tcolorbox}
  }
\definecolor{definitionBackground}{RGB}{246,246,246}
\newenvironment{definition}[1]
  {
    \begin{tcolorbox}[
      enhanced,
      boxrule=0pt,
      frame hidden,
      sharp corners,
      colback=definitionBackground,
      borderline west={3pt}{-1.5pt}{black},
      breakable
    ]
    \textbf{Definition. }\emph{#1}\\
  }
  {
    \end{tcolorbox}
  }

\newenvironment{amatrix}[2]{
    \left[
      \begin{array}{*{#1}{c}|*{#2}c}
  }
  {
      \end{array}
    \right]
  }
\definecolor{codeBackground}{RGB}{253,246,225}
\definecolor{dkgreen}{rgb}{0,0.6,0}
\definecolor{gray}{rgb}{0.5,0.5,0.5}
\definecolor{mauve}{rgb}{0.58,0,0.82}
\lstset{
  language=C++,
  aboveskip=3mm,
  belowskip=3mm,
  backgroundcolor=\color{codeBackground},
  showstringspaces=false,
  columns=flexible,
  basicstyle={\small\ttfamily},
  numbers=none,
  numberstyle=\tiny\color{gray},
  keywordstyle=\color{blue},
  commentstyle=\color{dkgreen},
  stringstyle=\color{mauve},
  breaklines=true,
  breakatwhitespace=true,
  tabsize=2
}

\date{\the\year-\the\month-\the\day}
\author{Kyle Chui}


\fancyhf{}
\lhead{Kyle Chui}
\rhead{Page \thepage}
\pagestyle{fancy}

\begin{document}
  \section{Capacitors}
  \begin{definition}{Capacitor}
    Any two conductors separated by an insulator (or vacuum) for a \emph{capacitor}.
  \end{definition}
  Charges can be stored indefinitely on the conductors, and can be discharged quickly. This is in contrast to batteries, which provide a consistent, steady flow of charges.
  \begin{definition}{Capacitance}
    \emph{Capacitance} is a measure of how much charge a capacitor can store for a given voltage, and is given by the equation
    \[
      C = \frac{Q}{V}.
    \]
    \begin{itemize}
      \item Capacitance is measured in Coulombs per Volt or Farads (F).
      \item $1$ F is huge capacitance.
      \item $Q$ (on the plates) causes $E$ (between the plates), which causes $V$. So $Q\sim E\sim V$.
      \item The actual ratio of $\frac{Q}{V}$ depends only on the shape and material because $Q$ and $V$ are proportional to each other and cancel out. In other words, capacitance depends only on the shape/material of the capacitor.
    \end{itemize}
  \end{definition}
  \subsection{Parallel Plate Capacitor}
  The electric field for a parallel plate capacitor is $E = \frac{\sigma}{\eps_0} = \frac{Q}{\eps_0A}$. We know that
  \[
    V = E\cdot d = \frac{Qd}{\eps_0A},
  \]
  so 
  \[
    \boxed{C = \frac{Q}{V} = \frac{\eps_0A}{d}.}
  \]
  \subsection{Energy Stored in a Plane Capacitor}
  We know that the charge $q$ placed in an electric potential $V$ stores energy: $U = q\cdot V$, so
  \[
    U_C = \int_{0}^{Q}V \dx q = \int_{0}^{Q}\frac{q}{C} \dx q = \frac{1}{C}\int_{0}^{Q}q \dx q = \frac{1}{2}\frac{Q^2}{C}
  \]
  Thus we have
  \[
    \boxed{U_C = \frac{1}{2}CV^2 = \frac{1}{2}QV = \frac{1}{2}\frac{Q^2}{C}.}
  \]
  \subsection{Spherical Capacitor}
  Solving for the voltage difference between two points, we have
  \[
    \Delta V = kQ \paren{\frac{1}{r_a} - \frac{1}{r_b}} = kQ \paren{\frac{r_b - r_a}{r_ar_b}}.
  \]
  Thus the capacitance is
  \[
    C = \frac{Q}{V} = \frac{1}{k}\frac{r_ar_b}{r_b-r_a}
  \]
  \subsection{Cylindrical Capacitor}
  From earlier, we found that the potential difference for a cylinder is
  \[
    \Delta V = 2k\lam\ln \paren{\frac{r_b}{r_a}}.
  \]
  Thus we have
  \[
    C = \frac{Q}{\Delta V} = \frac{\lam L}{2k\lam\ln \paren{\frac{r_b}{r_a}}} = \frac{2\pi \eps_0 L}{\ln \paren{\frac{r_b}{r_a}}}.
  \]
  In other words, we may write
  \[
    \frac{C}{L} = \frac{2\pi\eps_0}{\ln \paren{\frac{r_b}{r_a}}}.
  \]
  \subsection{Capacitors in Series or Parallel}
  \begin{itemize}
    \item Capacitors can be combined in ``series" or ``parallel".
    \item All capacitors then act as one single capacitor with an ``effective capacitance".
    \item Adding capacitors in parallel makes a larger capacitor.
    \item Adding capacitors in series allows you to operate at much higher voltages.
  \end{itemize}
  \textbf{Series Capacitors.} In series, the voltages add up, and the charge is constant across the capacitors. We write $\frac{1}{C_{\text{eq}}} = \sum \frac{1}{C_i}$. Thus the equivalent capacitance is smaller than all of the capacitances of the components. Each capacitor only sees a fraction of the total voltage. Putting capacitors in series essentially increases $d$, while $A$ remains constant. \\[10pt]
  \textbf{Parallel Capacitors.} When in parallel, the capacitance adds up linearly, and the voltage is constant across the capacitors. We write $C_{\text{eq}} = \sum C_i$. The equivalent capacitance is larger than all of the capacitances of the components. Each capacitor only sees a fraction of the total charge. Putting capacitors in parallel essentially increases $A$, while $d$ remains constant.
  \subsection{Dielectrics}
  Most real capacitors have an insulator (dielectric) between the plates, which stores additonal energy via electric dipole alignment. This increased breakdown threshold allows for higher voltages between the plates.
  \begin{definition}{Dielectric Constant}
    The \emph{dielectric constant} $K$ is the ratio of the capacitances before and after you put a dielectric in it. It is given by $K = \frac{C}{C_0}$, and is always greater than or equal to one.
  \end{definition}
  \begin{note}{}
    When you align dipoles inside an electric field, they line up anti-parallel to the original field $E_0$ and create a new, weaker field $E$ that is in the opposite direction of the original. The magnitude of this field is $E = \frac{E_0}{K}$, where $K$ is the dielectric constant. More energy is now stored in the aligned dipoles. 
  \end{note}
  When you fill a capacitor with a dielectric, the opposing electric field decreases the overall electric field. Let the induced charge inside a capacitor be denoted by $\sigma_i$. Then we have that 
  \[
    \sigma_i = \sigma \paren{1 - \frac{1}{K}}.
  \]
  So if $K$ is very large, then $\sigma_i$ is almost as large as $\sigma$. Furthermore, 
  \[
    E = \frac{1}{\eps_0}(\sigma - \sigma_i) = \frac{1}{\eps_0}\paren{\sigma -\sigma + \frac{\sigma}{K}} = \frac{\sigma}{\eps_0K}.
  \]
  \begin{definition}{Permittivity}
    The product $\eps = K\eps_0$ is called the \emph{permittivity} of the dielectric.
  \end{definition}
  Thus we can write that the field $E$ is given by $E = \frac{\sigma}{\eps}$.
  \subsection{Summary}
  For constant charge on a capacitor, adding a dielectric with constant $K$ between the plates will:
  \begin{itemize}
    \item Decrease the electric field by a factor of $K$.
    \item Decrease the voltage between the plates because $V = Ed$.
    \item Increase the capacitance by a factor of $K$ because $C = \frac{Q}{V}$.
  \end{itemize}
  Aligning dipoles in a dielectric increases the energy density:
  \[
    u = \frac{1}{2}K\eps_0 E^2 = \frac{1}{2}\eps E^2.
  \]
  \begin{theorem}{General Form of Gauss's Law}
    If we have charges inside a dielectric, we modify Gauss's Law to be
    \[
      \oint \vec{E}\dx \vec{A} = \frac{Q_{\text{enclosed}-\text{free}}}{K\cdot \eps_0} = \frac{Q_{\text{enclosed}-\text{free}}}{\eps}.
    \]
    We ignore the induced dipole charges when evaluating the formula.
  \end{theorem}
\end{document}
