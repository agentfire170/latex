\documentclass[class=article, crop=false]{standalone}
% Import packages
\usepackage[margin=1in]{geometry}

\usepackage[many]{tcolorbox}
\usepackage{amssymb, amsthm}
\usepackage{comment}
\usepackage{enumitem}
\usepackage{fancyhdr}
\usepackage{hyperref}
\usepackage{import}
\usepackage{listings}
\usepackage{mathrsfs, mathtools}
\usepackage{pdfpages}
\usepackage{standalone}
\usepackage{transparent}
\usepackage{xcolor}

\usetikzlibrary{decorations.pathreplacing}
\tcbuselibrary{skins}
% Declare math operators
\DeclareMathOperator{\lcm}{lcm}
\DeclareMathOperator{\proj}{proj}
\DeclareMathOperator{\vspan}{span}
\DeclareMathOperator{\im}{im}
\DeclareMathOperator{\range}{range}
\DeclareMathOperator{\Diff}{Diff}
\DeclareMathOperator{\Int}{Int}
\DeclareMathOperator{\fcn}{fcn}
\DeclareMathOperator{\id}{id}
\DeclareMathOperator{\rank}{rank}
\DeclareMathOperator{\tr}{tr}
\DeclareMathOperator{\dive}{div}
\DeclareMathOperator{\row}{row}
\DeclareMathOperator{\col}{col}
% Macros for letters/variables
\renewcommand{\tilde}{\raisebox{0.4ex}{\resizebox{2ex}{!}{\texttildelow}}}
\newcommand{\N}{\ensuremath{\mathbb{N}}}
\newcommand{\Z}{\ensuremath{\mathbb{Z}}}
\newcommand{\Q}{\ensuremath{\mathbb{Q}}}
\newcommand{\R}{\ensuremath{\mathbb{R}}}
\newcommand{\C}{\ensuremath{\mathbb{C}}}
\newcommand{\F}{\ensuremath{\mathbb{F}}}
\newcommand{\M}{\ensuremath{\mathbb{M}}}
\newcommand{\lam}{\ensuremath{\lambda}}
\newcommand{\nab}{\ensuremath{\nabla}}
\newcommand{\eps}{\ensuremath{\varepsilon}}
\newcommand{\es}{\ensuremath{\varnothing}}
% Macros for math symbols
\newcommand{\dx}[1]{\,\mathrm{d}#1}
\newcommand{\inv}{\ensuremath{^{-1}}}
\newcommand{\sm}{\setminus}
\newcommand{\sse}{\subseteq}
\newcommand{\ceq}{\coloneqq}
% Macros for pairs of math symbols
\newcommand{\abs}[1]{\ensuremath{\left\lvert #1 \right\rvert}}
\newcommand{\paren}[1]{\ensuremath{\left( #1 \right)}}
\newcommand{\norm}[1]{\ensuremath{\left\lVert #1\right\rVert}}
\newcommand{\set}[1]{\ensuremath{\left\{#1\right\}}}
\newcommand{\tup}[1]{\ensuremath{\left\langle #1 \right\rangle}}
\newcommand{\floor}[1]{\ensuremath{\left\lfloor #1 \right\rfloor}}
\newcommand{\ceil}[1]{\ensuremath{\left\lceil #1 \right\rceil}}
\newcommand{\eclass}[1]{\ensuremath{\left[ #1 \right]}}

\newcommand{\chapternum}{}
\newcommand{\ex}[1]{\noindent\textbf{Exercise \chapternum.{#1}.}}

\newcommand{\tsub}[1]{\textsubscript{#1}}
\newcommand{\tsup}[1]{\textsuperscript{#1}}

% Include figures
\newcommand{\incfig}[2][1]{%
    \def\svgwidth{#1\columnwidth}
    \import{./figures/}{#2.pdf_tex}
}

\definecolor{problemBackground}{RGB}{212,232,246}

\newenvironment{problem}[1]
  {
    \begin{tcolorbox}[
      boxrule=.5pt,
      titlerule=.5pt,
      sharp corners,
      colback=problemBackground,
      breakable
    ]
    \ifx &#1& \textbf{Problem. }
    \else \textbf{Problem #1.} \fi
  }
  {
    \end{tcolorbox}
  }
\definecolor{exampleBackground}{RGB}{255,249,248}
\definecolor{exampleAccent}{RGB}{158,60,14}
\newenvironment{example}[1]
  {
    \begin{tcolorbox}[
      boxrule=.5pt,
      sharp corners,
      colback=exampleBackground,
      colframe=exampleAccent,
    ]
    \color{exampleAccent}\textbf{Example.} \emph{#1}\color{black}
  }
  {
    \end{tcolorbox}
  }
\definecolor{theoremBackground}{RGB}{234,243,251}
\definecolor{theoremAccent}{RGB}{0,116,183}
\newenvironment{theorem}[1]
  {
    \begin{tcolorbox}[
      boxrule=.5pt,
      titlerule=.5pt,
      sharp corners,
      colback=theoremBackground,
      colframe=theoremAccent,
      breakable
    ]
      \color{theoremAccent}\textbf{Theorem --- }\emph{#1}\\\color{black}
  }
  {
    \end{tcolorbox}
  }
\definecolor{noteBackground}{RGB}{244,249,244}
\definecolor{noteAccent}{RGB}{34,139,34}
\newenvironment{note}[1]
  {
  \begin{tcolorbox}[
    enhanced,
    boxrule=0pt,
    frame hidden,
    sharp corners,
    colback=noteBackground,
    borderline west={3pt}{-1.5pt}{noteAccent},
    breakable
    ]
    \ifx &#1& \color{noteAccent}\textbf{Note. }\color{black}
    \else \color{noteAccent}\textbf{Note (#1). }\color{black} \fi
    }
    {
  \end{tcolorbox}
  }
\definecolor{lemmaBackground}{RGB}{255,247,234}
\definecolor{lemmaAccent}{RGB}{255,153,0}
\newenvironment{lemma}[1]
  {
  \begin{tcolorbox}[
    enhanced,
    boxrule=0pt,
    frame hidden,
    sharp corners,
    colback=lemmaBackground,
    borderline west={3pt}{-1.5pt}{lemmaAccent},
    breakable
    ]
    \ifx &#1& \color{lemmaAccent}\textbf{Lemma. }\color{black}
    \else \color{lemmaAccent}\textbf{Lemma #1. }\color{black} \fi
    }
    {
  \end{tcolorbox}
  }
\definecolor{definitionBackground}{RGB}{246,246,246}
\newenvironment{definition}[1]
  {
    \begin{tcolorbox}[
      enhanced,
      boxrule=0pt,
      frame hidden,
      sharp corners,
      colback=definitionBackground,
      borderline west={3pt}{-1.5pt}{black},
      breakable
    ]
    \textbf{Definition. }\emph{#1}\\
  }
  {
    \end{tcolorbox}
  }

\newenvironment{amatrix}[2]{
    \left[
      \begin{array}{*{#1}{c}|*{#2}c}
  }
  {
      \end{array}
    \right]
  }
\definecolor{codeBackground}{RGB}{253,246,225}
\definecolor{dkgreen}{rgb}{0,0.6,0}
\definecolor{gray}{rgb}{0.5,0.5,0.5}
\definecolor{mauve}{rgb}{0.58,0,0.82}
\lstset{
  language=C++,
  aboveskip=3mm,
  belowskip=3mm,
  backgroundcolor=\color{codeBackground},
  showstringspaces=false,
  columns=flexible,
  basicstyle={\small\ttfamily},
  numbers=none,
  numberstyle=\tiny\color{gray},
  keywordstyle=\color{blue},
  commentstyle=\color{dkgreen},
  stringstyle=\color{mauve},
  breaklines=true,
  breakatwhitespace=true,
  tabsize=2
}

\date{\the\year-\the\month-\the\day}
\author{Kyle Chui}


\fancyhf{}
\lhead{Kyle Chui}
\rhead{Page \thepage}
\pagestyle{fancy}

\begin{document}
  \section{Dual Spaces}
  \begin{definition}{Coordinate Function}
    Let $f_{v_0}$ be the unique linear transformation (guaranteed by UPVS) such that
    \begin{align*}
      v_0 &\mapsto 1 \\
      v &\mapsto 0 \tag{For all $v_0\neq v\in \mathcal{B}$}
    \end{align*}
  \end{definition}
  We know that the image of $f_{v_0}$ is a subspace of $F$ and non-zero, so it must be all of $F$. Thus $f_{v_0}$ is an epimorphism. Furthermore, we know that
  \begin{align*}
    \ker(f_{v_0}) &= \set{w\in V\mid w \text{ has $v_0$-coordinate zero}} \\
                  &= \vspan(\mathcal{B}\sm \set{v_0}).
  \end{align*}
  \begin{definition}{Kronecker Delta}
    The Kronecker delta function is the linear transformation $\delta\colon V\to F$ that sends the basis of a vector space to either $0$ or $1$, given by
    \[
      \delta_{v\,v'} = \begin{cases}1 & \text{if }v=v',\\0 & \text{if }v\neq v', v\in \mathcal{B},\end{cases}
    \]
    where $v'\in \mathcal{B}$.
  \end{definition}
  \textbf{Motivation.} Instead of studying abstract vectors in $V$, we want to study the coordinates of vectors in $V$ relative to a basis $\mathcal{B}$.
  \begin{definition}{Dual Space}
    The vector space $V^*\ceq L(V, F)$ is called the \emph{dual space} of $V$.
  \end{definition}
  \begin{note}{}
    The coordinate function will give you the coordinates of a vector $w\in V$, so instead of defining a bunch of alphas, we can just write
    \[
      w = \sum_{\mathcal{B}}f_v(w)v. \tag{$v\in\mathcal{B}$}
    \]
  \end{note}
  Now by the UPVS, we have a unique linear transformation
  \[
    D_{\mathcal{B}}\colon V\to V^*\quad \text{defined by}\quad v\mapsto f_v.\tag{$v\in \mathcal{B}$}
  \]
  We claim that $D_{\mathcal{B}}$ is a monomorphism.
  \begin{proof}
    Suppose $w = \sum_{\mathcal{B}}\alpha_vf_v = 0$. Let $v_0\in \mathcal{B}$, then
    \begin{align*}
      0 &= \paren{\sum_{\mathcal{B}}\alpha_vf_v}(v_0) \\
        &= \sum_{\mathcal{B}}\alpha_vf_v(v_0) \\
        &= \sum_{\mathcal{B}}\alpha_v\delta_{v\,v_0} \\
        &= \alpha_{v_0}. \tag{If $v\neq v_0$, $\alpha_v=0$}
    \end{align*}
    Thus we have $\sum \alpha_vf_v = 0$ implies $\alpha_v = 0$ for all $v\in \mathcal{B}$, so $w = 0$. Thus $D_{\mathcal{B}}$ is a monomorphism.
  \end{proof}
  By the Monomorphism Theorem, we know that $\mathcal{B}^*$ is linearly independent because $\mathcal{B}$ is a basis for $V$.
  \begin{note}{}
    If $V$ is not finite dimensional, then $D_{\mathcal{B}}$ is \emph{not} onto. This is because we have $\abs{V^*} = \abs{F}^{\abs{\mathcal{B}}}$ and $\abs{F} = \abs{V}$ if $F$ is infinite.
  \end{note}
  \begin{note}{}
    The map $D_{\mathcal{B}}\colon V\to V^*$ is \emph{dependent} on your choice of basis $\mathcal{B}$.
  \end{note}
  \begin{definition}{Linear Functional}
    If $V$ is a vector space over $F$, then elements in $V^* = L(V, F)$ are called \emph{linear functionals}.
  \end{definition}
  \textbf{Fact.} (We assume this) If $S$ is a linearly independent set in a vector space over $F$ (even infinite), then $S$ is part of a basis for $V$, i.e. the Extension Theorem holds (needs the Axiom of Choice). This works out nicely for us because every non-zero vector $v\in V$ can be extended into a basis for $V$.
  \begin{note}{}
    There exists a monomorphism $L\colon V\to V^{* *}$ that is \emph{independent} of your choice of bases.
  \end{note}
  \begin{proof}
    For each $v\in V$ define the following linear functional on $V^*$:
    \[
      L_v\colon V^*\to F \quad \text{by}\quad L_v(f)\ceq f(v) \tag{Evaluation at $v$}
    \]
    We now show that $L_v\colon V^*\to F$ is linear, i.e. $L_v\in V^{* *} = (V^*)^*$. Let $f, g\in V^*$, and $\alpha\in F$. Observe that
    \begin{align*}
      L_v(\alpha f + g) &= (\alpha f + g)(v) \\
                        &= \alpha f(v) + g(v) \\
                        &= \alpha L_v(f) + L_v(g).
    \end{align*}
    Thus $L_v$ is linear. We now define a map $L\colon V\to V^{* *}$ by $v\mapsto L_v$ (note that $L_v\in V^{* *}$, $L_v = Lv$). We claim that $L$ is linear. For all $f\in V^*$, $v,v'\in V$, $\alpha\in F$, we have
    \begin{align*}
      L(\alpha v + v')(f) &= L_{\alpha v + v'}(f) \\
                          &= f(\alpha v + v') \\
                          &= \alpha f(v) + f(v') \\
                          &= \alpha L_v(f) + L_{v'}(f) \\
                          &= \alpha L(v)(f) + L(v')(f) \\
                          &= (\alpha Lv + Lv')(f).
    \end{align*}
    We now claim that $L\colon V\to V^{* *}$ is monic. Suppose $v\neq 0$. It suffices to show that there exists some $f\in V^*$ such that $L_v(f) = f(v) \neq 0$. We know from earlier that there exists a $f\in V^*$ such that $L_v(f) = f(v) \neq 0$. Because $L$ is linear, $L$ is a monomorphism. We call $L\colon V\to V^{* *}$ is a \emph{natural} or \emph{canonical} monomorphism---no basis is needed to define it.
  \end{proof}
  \begin{note}{}
    The canonical monomorphism is not an isomorphism in the infinite dimensional case, so we mainly deal with the finite dimensional case.
  \end{note}
  We now assume that $V$ is a finite dimensional vector space over $F$. Let $\mathcal{B} = \set{v_1,\dotsc,v_n}$ be a basis for $V$, and $\mathcal{B}^* = \set{f_1,\dotsc,f_n}\sse V^*$ defined by $f_i(v_j) = \delta_{ij}$ for all $i, j$ ($f_i$ means $f_{v_i}$ in this case). Then, as before, we have a monomorphism
  \[
    D_{\mathcal{B}}\colon V\to V^* \quad \text{determined by}\quad v_i\mapsto f_i.
  \]
  But we also have
  \[
    \dim(V^*) = \dim(L(V, F)) = \dim(V)\dim(F) = \dim(V).
  \]
  Thus $D_{\mathcal{B}}$ is an isomorphism by the Isomorphism Theorem, with $\mathcal{B}^*$ a basis for $V^*$ called the \emph{dual basis} of $\mathcal{B}$. We also have $V\cong V^*\cong V^{* *}$, so $V\cong V^{* *}$. Furthermore, $\mathcal{B}^{* *} = \set{Lv_1,\dotsc,Lv_n}$, where $L_{v_i}\ceq L_{f_i}$, $f_i\in B^*$. Then
  \[
    L_{f_i}(f_j) = L_{v_i}(f_j) = f_j(v_i) = \delta_{ij}.
  \]
  So $\mathcal{B}^{* *}$ is the \emph{dual basis} of $\mathcal{B}^*$. We also know that $L\colon V\to V^{* *}$ is now a \emph{natural isomorphism} by the Isomorphism Theorem. \par
  When $V$ is a finite dimensional vector space over $F$, we can \emph{identify} $L_v$ and $v$ for all $v\in V$ (just like how we can see integers as rational numbers by mapping $n\mapsto \frac{n}{1}$). \par
  So for a finite dimensional vector space $V$ over $F$, we have
  \begin{center}\begin{tabular}{c|c}
    $\mathcal{B} = \set{v_1,\dotsc,v_n}$ & A basis for $V$ \\
    \hline
    $\mathcal{B}^* = \set{f_1,\dotsc,f_n} = \set{f_{v_1},\dotsc,f_{v_n}}$ & The dual basis for $\mathcal{B}$ \\
    \hline
    $\mathcal{B}^{* *} = \set{L_{f_{v_1}},\dotsc,L_{f_{v_n}}} = \set{L_{v_1},\dotsc,L_{v_n}}$ & The dual basis for $\mathcal{B}^*$
  \end{tabular}\end{center}
  In other words, $f_i = f_{v_i}$ and $L_{f_{v_i}} = L_{v_i}$ and these satisfy
  \[
    f_j(v_i) = f_{v_j}(v_i) = \delta_{ij} = L_{f_{v_i}}(f_j) = L_{v_i}(f_j).
  \]
  If $v\in V$, then
  \[
    v = \alpha_1v_1 + \dotsb + \alpha_nv_n = \sum_{i=1}^{n}f_i(v)v_i. \tag{For unique $\alpha_1,\dotsc,\alpha_n\in F$}
  \]
  Similarly, if $f\in V^*$, then
  \[
    f = \beta_1f_1 + \dotsb + \beta_nf_n = \sum_{i=1}^{n}f(v_i)f_i.
  \]
  This is the same argument as before, just decomposing $f$ into the product of scalars and basis vectors.
  \begin{note}{}
    The $f$'s and the $v$'s determine each other.
  \end{note}
  \subsection{The Transpose}
  Let $V,W$ be vector spaces over $F$, with $T\colon V\to W$ linear. If $g\in W^* = L(W, F)$, in other words $g\colon W\to F$ linear, then the composition
  \[
    V\xrightarrow{T}W\xrightarrow{g}F
  \]
  is a linear functional, i.e. $g\circ T\in V^*$.
  \begin{definition}{Transpose}
    Let $V, W$ be vector spaces over a field $F$, $T\colon V\to W$ linear. Define the \emph{transpose} of $T$ by
    \[
      T^t\colon W^*\to V^* \quad \text{where}\quad g\mapsto g\circ T. \tag{$T^tg\ceq g\circ T$ for all $g\in W^*$}
    \]
  \end{definition}
  The following diagrams commute:
  \[\begin{tikzcd}
    V & W \\
    {V^*} & {W^*}
    \arrow["{T^t}", from=2-2, to=2-1]
    \arrow["T", from=1-1, to=1-2]
  \end{tikzcd}
  \hspace{1in}
  \begin{tikzcd}
    V & W \\
    & F
    \arrow["g", from=1-2, to=2-2]
    \arrow["T", from=1-1, to=1-2]
    \arrow["{g\circ T}"', from=1-1, to=2-2]
  \end{tikzcd}\tag{$T^tg = g\circ T$}\]
  \textbf{Claim.} $T^t\colon W^*\to V^*$ is linear.
  \begin{proof}
    If $g, g'\in W^*$, $\alpha\in F$, then
    \begin{align*}
      T^t(\alpha g + g') &= (\alpha g + g')\circ T \\
                         &= \alpha gT + g'T \\
                         &= \alpha T^tg + T^tg'.
    \end{align*}
  \end{proof}
  We call $T^t$ the transpose because of the following.
  \begin{theorem}{Addendum to Matrix Theory Theorem}
    Let $V, W$ be finite dimensional vector spaces over $F$, $\mathcal{B},\mathcal{C}$ ordered bases for $V, W$ respectively, $T\colon V\to W$ linear. Then
    \[
      [T]_{\mathcal{B},\mathcal{C}}^t = [T^t]_{\mathcal{C}^*,\mathcal{B}^*}
    \]
  \end{theorem}
  \begin{proof}
    Let
    \[\begin{array}{cc}\mathcal{B} = \set{v_1,\dotsc,v_n}, & \mathcal{B}^* = \set{f_1,\dotsc,f_n},\\ \mathcal{C} = \set{w_1,\dotsc,w_m}, & \mathcal{C}^* = \set{g_1,\dotsc,g_n},\end{array}\]
    where $\mathcal{B}^*,\mathcal{C^*}$ the ordered dual bases of ordered bases $\mathcal{B},\mathcal{C}$ of $V$ and $W$, respectively. This means that $f_i(v_j) = \delta_{ij}$ and $g_k(w_\ell) = \delta_{k\ell}$. Let
    \[
      [T]_{\mathcal{B},\mathcal{C}} = (\alpha_{ij}) \quad \text{and}\quad [T^t]_{\mathcal{C}^*,\mathcal{B}^*} = (\beta_{ij}).
    \]
    In other words, by definition, we have
    \begin{align*}
      Tv_k &= \sum_{i=1}^{m}\alpha_{ik}w_i \quad\text{in }W\tag{$k = 1,\dotsc,n$}. \\
      T^tg_j &= \sum_{i=1}^{n}\beta_{ij}f_i \quad\text{in }V^* \tag{$j = 1,\dotsc,m$}.
    \end{align*}
    Then computation gives
    \begin{align*}
      (T^tg_j)(v_k) &= g_j(Tv_k) \tag{By definition}\\
                    &= g_j \paren{\sum_{i=1}^{m}\alpha_{ik}w_i} \tag{From above}\\
                    &= \sum_{i=1}^{m}\alpha_{ik}g_j(w_i) \\
                    &= \sum_{i=1}^{m}\alpha_{ik}\delta_{ij} \tag{By definition}\\
                    &= \alpha_{jk}.
    \end{align*}
    and
    \begin{align*}
      (T^tg_j)(v_k) &= \paren{\sum_{i=1}^{n}\beta_{ij}f_i}(v_k) \tag{From above}\\
                    &= \sum_{i=1}^{n}\beta_{ij}f_i(v_k) \\
                    &= \sum_{i=1}^{n}\beta_{ij}\delta_{ik} \\
                    &= \beta_{kj}.
    \end{align*}
    Thus for all $j, k$, we have $\alpha_{jk} = \beta_{kj}$ as needed.
  \end{proof}
  \begin{definition}{Annihilator}
    Let $V$ be a vector space over $F$, $\es\neq S\sse V$ a subset. Then the set
    \[
      S^0 \ceq \set{f\in V^*\mid f|_S = 0} = \set{f\in V^*\mid f(s) = 0, \forall s\in S}
    \]
    is called the \emph{annihilator} of $S$. In other words, the annihilator is the set of all linear functionals, which, when you restrict their domains to $S$, become the zero map.
  \end{definition}
  \textbf{Claim.} The annihilator $S^0\sse V^*$ is a subspace (even if $S$ is not).
  \begin{proof}
    Let $f, g\in S^0$, $\alpha\in F$. To show that $(\alpha f + g)|_S = 0$, we consider some $s\in S$. Then
    \begin{align*}
      (\alpha f + g)(s) &= \alpha f(s) + g(s) \\
                        &= 0 + 0 \\
                        &= 0,
    \end{align*}
    so $\alpha f + g\in S^0$.
  \end{proof}
  \textbf{Observation.} Let $T\colon V\to W$ be linear. Then $\ker(T^t) = (\im(T))^0$ (they both lie in $W^*$).
  \begin{proof}
    Let $g\in \ker(T^t)$. Then $T^tg = 0$, so $(T^tg)(v) = 0$ for all $v\in V$. Thus by definition of transpose, we have $g(Tv) = 0$ for all $v\in V$ so $g\in (\im(T)^0)$.
  \end{proof}
  \begin{theorem}{Annihilator Equation}
    Let $V$ be a finite dimensional vector space over $F$, with $W\sse V$ a subspace. Then
    \[
      \dim(V) = \dim(W) + \dim(W^0).
    \]
  \end{theorem}
  \begin{proof}
    Let $\set{v_1,\dotsc,v_m}$ be a basis for $W$. We extend it to $\mathcal{B} = \set{v_1,\dotsc,v_n}$, a basis for $V$. Let $\mathcal{B}^* = \set{f_1,\dotsc,f_n)}$ be the dual basis of $\mathcal{B}$ so that $f_i(v_j) = \delta_{ij}$ for all $i, j$. \par
    We claim that $\mathcal{C} = \set{f_{k+1},\dotsc,f_n}$ is a basis for $W^0$. Let $f\in W^0$. Then there exist $\beta_1,\dotsc,\beta_n\in F$ such that 
    \[
      f = \sum_{i=1}^{n}\beta_if_i = \sum_{i=1}^{n}f(v_i)f_i = \sum_{i=k+1}^{n}f(v_i)f_i \in \vspan(\mathcal{C}).
    \]
    Because $\mathcal{C}\sse \mathcal{B}^*$ and $\mathcal{B}^*$ is linearly independent, so is $\mathcal{C}$.
  \end{proof}
  \textbf{Corollary.} Let $V$ be a finite dimensional vector space over $F$, $W\sse V$ a subspace. Identifying $V$ and $V^{* *}$ via $v\leftrightarrow L_v$, we have $W = (W^0)^0 = W^{00}$.
  \begin{proof}
    We first that they have the same dimension. Using the Annihilator Equation, we have that
    \begin{align*}
      \dim(V) &= \dim(W) + \dim(W^0) \\
      \dim(V^*) &= \dim(W^0) + \dim(W^{00}).
    \end{align*}
    Because we know that $\dim(V) = \dim(V^*)$, we can see that $\dim(W) = \dim(W^{00})$. It suffices to show that $W\sse W^{00}$ or $W^{00}\sse W$ to complete the proof. \par
    Suppose $w\in W$. Then
    \[
      L_wf = f(w) = 0, \quad \text{for all }f\in W^0.
    \]
    So 
    \[
      w = L_w\in W^{00}, \quad\text{i.e. $W\sse W^{00}$ is a subspace.}
    \]
    Thus we have shown that $W\sse W^{00}$, so $W = W^{00}$.
  \end{proof}
  \begin{theorem}{Rank of Transpose}
    Let $V, W$ be finite dimensional vector spaces over $F$, $T\colon V\to W$ linear. Then
    \[
      \dim(\im(T)) = \dim(\im(T^t)).
    \]
  \end{theorem}
  \begin{proof}
    We have
    \begin{align*}
      \dim(\im(W)) &= \dim(\im(T)) + \dim(\im(T)^0) \tag{Annihilator Equation} \\
      \dim(W^*) &= \dim(\im(T^t)) + \dim(\ker(T^t)). \tag{Dimension Theorem}
    \end{align*}
    Using the observation from before, we have that $\dim(\im(T)^0) = \dim(\im(T))$. Therefore $\dim(\im(T)) = \dim(\im(T^t))$.
  \end{proof}
\end{document}
